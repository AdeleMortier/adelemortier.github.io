%% acknowledgments.tex

% From mitthesis package
% Version: 1.02, 2024/06/19
% Documentation: https://ctan.org/pkg/mitthesis

\chapter*{Acknowledgments}
\pdfbookmark[0]{Acknowledgments}{acknowledgments}


I have been thinking about this piece of text for quite some time -- in fact, my first draft dates back to March 26, 2024, almost a year and a half ago. But despite such early attempts, and many more in my head, I felt I had to wait until the very last minute to truly put this together. I think it's because there are so many people to thank, at so many levels, that I was afraid to produce a version that would be inevitably imperfect. But now is the time, so here goes. I hope I don't lose track along the way.\\

First and foremost, I want to thank my dissertation committee. It was a big one, but I could never have chosen a smaller group: each of you supported me in such unique and complementary ways that together you shaped this work in ways no single subset could have.
To Athulya: you were always there to lift me up when I hit my lowest points; you probably hold the record for the number of times you've seen me cry (and I must admit, the competition was fierce). Beyond that, you were an inspiring mentor from the very beginning, way before I started thinking about my dissertation. You helped me develop my first original research ideas, polish my first abstracts, design my experiments, present my work more effectively, and, eventually, succeed on the job market. I feel incredibly indebted to you. All I can say is that your future students at Yale are very, very lucky. You will always be an ultimate role model for me.

To Amir: I can hardly believe we've been following each other for nearly seven years now. It was such a joy to have you as my TA back in Paris in 2019, and then again as a professor at MIT. Your teaching left a lasting impression on me -- your very first seminar at MIT probably sowed the seeds of this dissertation. As an advisor (and a conference buddy!), you were always there to encourage me whenever I doubted my ideas. The way you manage to get inside someone's head and help them work through their thoughts is truly unique, and it played a major role in shaping key aspects of this dissertation. For that, I am deeply grateful. Even after leaving MIT, I hope I'll continue following in your footsteps, in one way or another.

To Viola: you convinced me that my work and my thinking had real worth in this field. Quite simply, I would not be where I am today had I not met you. Your ``solar'' and inspiring presence, but also your fresh, often cross-linguistic perspectives on my puzzles helped me through what I believe were the most challenging times of my life, professionally and personally. I'll also always treasure the times spent in Brno, Berlin, and Sicily with ``the girls''. MIT is so lucky to get you.


To Danny: you taught me (or at least tried to) to dig into the roots of an issue, to be rigorous, and perhaps more importantly, bold. When I write, I often find myself wondering what you would say. That has helped me not only shape my ongoing ideas, but also uncover new ones. I must say I'm sorry I couldn't always keep up with your train of thought, and that you sometimes had to repeat things ten times before I finally grasped the point! You and Martin gave me plenty of challenging stuff to chew on -- not just for this dissertation, but far beyond it. For all of this, I'm deeply grateful.

To Martin: even though our meetings were not so frequent, I always enjoyed them a lot. Your special way of reflecting on issues -- and of finding depth in things I had thought were mere implementational details -- helped me grow as a researcher. I also appreciated how our conversations always felt like true dialogues, from which I often walked away with something new and unexpected. As I step into a role of professor and mentor myself, I hope I can be as genuine and approachable as you.\\


Second, I want to thank all the mentors and role models I had before my dissertation. To all the strong, remarkable women I met from elementary through high school -- Louisette, Evelyne Roussel, Mélanie Martin, Brigitte Capelle, and Nathalie Renaud-Rodet -- thank you for inspiring in me the desire to play with words, and most importantly, to teach myself. To Jonathan Ginzburg, Paul Égré, and Benjamin Spector, who put their trust in me despite my lack of background in formal linguistics, becoming my advisors in 2018 (Jonathan) and 2019 (Paul and Benjamin). Looking at this dissertation -- especially its appeal to concepts such as granularity and incrementality -- I can clearly see the mark your advising left on my work, even after years of pursuing different topics. I am also grateful to Philippe Schlenker, who encouraged me to apply to MIT and offered invaluable career advice. Without the guidance of this Paris crowd, I most likely would never have found my way to MIT. At MIT, I want to thank David Pesetsky,  who was my first mentor within the department, an incredible source of knowledge as well as a deeply dedicated and understanding advisor. I also want to thank Roger Levy, who gave me the opportunity to TA his course in BCS. That experience taught me a great deal, both as a student and as a teacher, and I know it will serve me well in my future appointment.\\


Third, to my friends and fellow grad students at MIT: thank you. Part of it might have been trauma bonding, but I don't think I've ever formed such rich and deep connections before in my life.
To the amazing members of my cohort: Ido Benbaji-Elhadad, Omri Doron, Eunsun Jou, Yeong-Joon Kim, Annauk Olin, Yash Sinha, and Margaret Wang. You were my very first friends in the US, and your presence was a huge comfort whenever I missed home. I feel like we all learned so much from each other. I'm happy we made it through the pandemic together, and I'm very proud of us for (most likely) all graduating. Ido we are all waiting for you!

To my officemates over the years: Kit Baron, Ido Benbaji-Elhadad, Zachary Feldcamp, Boer Fu, Yizhen Jiang, Filipe Hisao Kobayashi, Haoming Li, Yash Sinha, and Jad Wehbe. Thank you for the interesting, fun, deep, and sometimes emotional conversations -- including the gossip. It was so sweet sharing this space with you all, and caring for each other in times of need. I only hope my mess, plants, and loud headphones weren't too unbearable.

To all the other wonderful people I haven't yet mentioned, but who brightened my days, whether through hallway chats, student jobs done together, conferences, or other outings: Itai Bassi, Agnes Bi, Keny Chatain, Enrico Flor, Janek Guerrini, Nina Haslinger, Alex Hamme, Mina Hirzel, Anton Kukhto, Jéssica Mendes, Elise Newman, Lorenzo Pinton, Mitya Privoznov, Cooper Roberts, Vincent Rouillard, Zhouyi Sun, Anastasia Tsilia, Danfeng Wu, Bingzi Yu, and Cynthia Zhong -- to only name a few. I'm so grateful to have met such an interesting, kind, and fun crowd, and I hope our paths will cross again, one way or another.

Finally, to my closest friends: Margaret, Keely, Omri, Ido, and Jad. I'll keep it short, since I think we've already shared all the important and emotional stuff. Let's make that sitcom someday, when we're old and bored.\\


Fourth, I want to thank all the other people who supported me throughout my PhD, both materially and emotionally. To the staff members of the Linguistics and Philosophy Department: you were incredible, always ready with a solution to whatever issue I brought your way. Just to name a few things off the top of my head. A first-aid kit after I fell off my bike. The elusive bagel guillotine back when Ling-Lunch only had Dunkin. An old school letter scale for shipping MITWPL books. A carefully rehydrated glue stick. A tricky Outlook update on the MITWPL computer. A collection of Ubuntu printer drivers for my own laptop. Countless cheap hostel nights reimbursed. And of course, all the more serious administrative matters; getting those right made me feel safe and welcome as an international student in the US. Your help and support made my life at MIT so much easier and undoubtedly contributed to my success in this department.


To my first US landlady, Susan: thank you for giving me (and my cat Sachou!) a roof, and for trusting me from thousands of kilometers away when I was apartment hunting from France. I will always feel something special when I return to Medford. To my few remaining friends from France: Quentin, Charlène, Sam, Marion, Patrick, Ali, Victor. I've been glad to have you in my life, even if we don't talk that often and things might have gotten tricky for various reasons. To my former partner, Félix: thank you for your support throughout our eight years together. They remain on my mind. To my aunts, Marie-Noëlle and Claudine: thank you for always being there for me, especially during difficult personal times. I hope I can one day offer you the same support in return. Lastly, to my parents, who did everything they could to help me succeed, who taught me the value of merit and hard work, and, I hope, to remain humble while being ambitious, I will always be indebted and grateful. \\

Now, for some lighter stuff. Thanks to old-school hardtrance and related genres, anime songs, \textit{chanson française}, and Margol for keeping me focused or energized when biking daily to MIT, frantically writing an abstract or even working on my dissertation.
Thanks also to the many MIT people who probably have no clue who I am but nonetheless kept me alive with free food fit for surviving WW3: the great Joshua J. Feliciano (for Chabad *leftover* feasts), Nathalie Hill (for the Protestant veggies, and sometimes Little Debbie too!), and of course Jen and Beatriz, along with the Ling-Lunch and Colloquium dinner organizers, for putting together so many memorable catered events. Thanks to you all for keeping me well-fed over the years. To Sandra, our favorite facilities staff: thank you for the late-night hellos, and for your empathy when the crazy Architects ``displaced'' my bike. And to Victor Antonio, to every tagalog person, to every ilocano person, to every filipino person, to every person, to every thing, to every place, to every time...keep it up buddy.


