\chapter{Comparison of the Qtree model to earlier similar approaches}\label{chap:lit-review}

\textbf{Abstract.} 
This Chapter consists in a literature review and compares the model of questions introduced in Chapter \ref{chap:accommodating-quds} to earlier approaches accounting for oddness phenomena \textit{via} theories of questions or alternatives. It is shown that these earlier models differ from the current framework in three possible ways: (i) the core model is technically very similar, but at the conceptual level assertions are not taken to evoke full-fledged questions \citep{Ippolito2019}, or (ii) the machinery proposed \textit{is} based on evoked QuDs but not fully compositional \citep{Zhang2022}, or (iii) question semantics is taken to fully \textit{replace} standard propositional content (the Inquisitive Semantics framework).



STRESSS OPTIONALITY
\section{Inquisitive Semantics}
inquisiyive semantics says that sentences are more or less questions, they raise issues.
but paradox:sentences and qs are the same kind of thing, but then, sentences get impoverished to be made diff from qs eventually


\section{Ippolito's contribution}

\citet{Ippolito2019} proposes a model of alternatives that is very close in its implementation to the Qtree model proposed in the first half of this Chapter. Under \citeauthor{Ippolito2019}'s view, the way alternatives are structured is seen as a source of oddness. But, as a whole, the account will be shown to differ from ours in at least three respects: first, sentences are not taken to evoke full-fledged questions (a mainly conceptual difference); second, it leaves unexplained when, and how, sets of alternatives can be combined, cross-sententially and in biclausal sentences; third, under this view oddness arises from a purely structural constraint (the \textit{Specificity Constraint}), that appears independent from familiar competition-based pragmatic principles. The current section will present the account and outline the first two differences. Chapter \ref{chap:redundancy} will further clarify the third difference, by introducing a new, competition-based \textsc{Redundancy} constraint on LF-Qtree pairs.

\subsection{The data}
\citet{Ippolito2019}'s goal was to provide a unified analysis of a number of seemingly independent instances of pragmatic oddness, taking the form of Sobel sequences (\ref{ex2:sobel}), sequences of superlatives (\ref{ex2:superlative}), and Hurford Disjunctions (\ref{ex2:hurford}). 

\begin{exe}
	\ex \label{ex2:sobel}
	\begin{xlist}
		\ex {If the USA had thrown their nuclear weapons into the sea, there would have been war. But if all the nuclear powers had thrown their weapons into the sea, there would have been peace.}\label{ex2:sobel-ws}
		\ex[\#] {If all the nuclear powers had thrown their nuclear weapons into the sea, there would have been peace. But if the USA had thrown their weapons into the sea, there would have been war.}\label{ex2:sobel-sw}
	\end{xlist}
	\ex \label{ex2:superlative}
	\begin{xlist}
		\ex {The closest gas stations are crummy; but the closest Shell stations are great.}
		\ex[\#] {The closest Shell stations are great; but the closest gas stations are crummy.}
	\end{xlist}	
	\ex \label{ex2:hurford}
	\begin{xlist}
		\ex {John ate some of the cookies or all of them.}\label{ex2:hurford-ws}
		\ex[\#] {John ate all of the cookies or some of them.}\label{ex2:hurford-sw}
	\end{xlist}
\end{exe}

These three families of sentences share commonalities. In all three configurations, two sentences or fragments are being contrasted using connectives like \textit{but} and \textit{or}. For instance, in the Sobel case (\ref{ex2:sobel-ws}), \textit{If the USA had thrown their nuclear weapons into the sea, there would have been war} gets contrasted with \textit{If all the nuclear powers had thrown their nuclear weapons into the sea, there would have been peace}. Additionally, in all three cases, the two sentences being contrasted exhibit some degree of parallelism, in the sense that they each contain a subconstituent $C$/$C^+$, such that $\llbracket C^+ \rrbracket \vdash \llbracket C \rrbracket$. For instance, \textit{all the nuclear powers had thrown their nuclear weapons into the sea}, entails that \textit{the USA had thrown their nuclear weapons into the sea}. Lastly, all configurations are such that the a. examples, which start with the sentences containing the ``weaker'' $C$, appear more felicitous than the b. examples, which start with the sentences containing the ``stronger'' $C^+$. 

\subsection{Structured Sets of Alternatives}
To account for these asymmetries, \citet{Ippolito2019} submits that the alternatives evoked by assertive sentences form ``structured sets'' (henceforth \textbf{SSA}). Such sets are defined in (\ref{ex2:ssa}). The kind of structures generated by this definition are in essence recursive partitions of the CS, or Qtrees, as defined in (\ref{ex2:qtree-def}).\footnote{This is what at least is argued in \citet{Ippolito2019}. It is worth mentioning however, that the definition in (\ref{ex2:ssa}) does not in itself guarantee that any Structured Set of Alternatives should form a tree. Instead, it guarantees that any branching of the form $[_{\alpha} \beta_1 ... \beta_n]$ is s.t. $(\beta_i)_{i\in [1; n]}$ partitions $\alpha$. But nothing in principle guarantees the connectedness of the structure: if specific alternatives happen to be ``missing'' (for relevance/QuD-related reasons, or perhaps due to a missing lexicalization), then, the resulting Structured Set of Alternatives may end up being a forest, instead of a single tree.}

\begin{exe}
	\ex {\textit{Structured Set of Alternatives (SSA) \citep{Ippolito2019}.} ${T}_{\mathcal{A}}$ is a well-formed structured set of alternatives iff the following conditions are met:
		\begin{itemize}
			\item Strength: for any two alternatives $\alpha$, $\beta$ 
			$\in \mathcal{A}$, $\beta$ is the daugther of $\alpha$ in ${T}_{\mathcal{A}}$ just in case $\llbracket\beta\rrbracket \subset \llbracket\alpha\rrbracket$.
			\item Disjointness: for any two alternatives $\beta_1$, $\beta_2$ $\in \mathcal{A}$, if $\beta_1$ and $\beta_2$  are sisters in ${T}_{\mathcal{A}}$, then $\llbracket\beta_1\rrbracket \cap \llbracket\beta_2\rrbracket = \emptyset$
			\item Exhaustivity: for any alternative $\alpha$ with daughters $\beta_1, ... \beta_n$, in ${T}_{\mathcal{A}}$, $\llbracket\beta_1\rrbracket \cup \llbracket\beta_2\rrbracket \cup ... \cup \llbracket\beta_n\rrbracket = \llbracket\alpha\rrbracket$
		\end{itemize}	
	}\label{ex2:ssa}
\end{exe}

Alternatives evoked by an assertion are modeled following \citet{Rooth1992}, i.e. assumed to be obtained by substituting the original sentence's focused material by any expression of the same type. This is spelled out in (\ref{ex2:focus-alternatives}). 

\begin{exe}
	\ex {\textit{Focus alternatives \citep{Rooth1992}.} Let $S$ be a sentence containing a focused element $\alpha$. The set of focus alternatives to $\llbracket S\rrbracket$ is the set of propositions $\llbracket S' \rrbracket$, where $S'$ is obtained from $S$ by substituting $\alpha$ with any element of the same type as $\alpha$.}\label{ex2:focus-alternatives}
\end{exe} 

Figure \ref{fig2:ssa-simplex} illustrates SSAs for simple sentences containing scalar and non-scalar alternatives. It is worth noting that sentences associated with different degrees of granularity (e.g. \textit{Jo grew up in Pairs} vs. \textit{Jo grew up in France}) are not expected to give rise to different SSAs, as shown in Figure \ref{fig2:ssa-non-scalar}. Same holds for scalar sentences in an entailment relation (e.g. \textit{Jo ate some of the cookies} vs. \textit{Jo ate all of the cookies}).

\begin{figure}[H]
	\centering
	\begin{subfigure}[b]{.45\linewidth}
		\centering
		\begin{forest}
			[[France[Paris][Lyon][...]][Germany[Berlin][...]][Italy[...]][...]]
		\end{forest}
		\caption{SSA associated with \textit{Jo grew up in Paris$_F$/France$_F$/Italy$_F$} etc.}\label{fig2:ssa-non-scalar}
	\end{subfigure}\hfill
	\begin{subfigure}[b]{.45\linewidth}
		\centering
		\begin{forest}
			[[some[all][{some but not all}]][none]]
		\end{forest}
		\caption{SSA associated with \textit{Jo ate some$_F$/all$_F$/none$_F$ of the cookies}.}\label{fig2:ssa-scalar}
	\end{subfigure}
	\caption{SSAs for simple focused sentences.}
	\label{fig2:ssa-simplex}
\end{figure}


Additionally, alternatives are assumed to be constrained by ``the'' QuD. This constitutes the first, conceptual difference with our account introduced earlier in this Chapter: under \citeauthor{Ippolito2019}'s view, assertions are not assumed to help determine ``the'' QuD; instead, they are assumed to evoke alternatives, which are themselves constrained by ``the'' QuD. In other words, SSAs are not expected to help determine what ``the'' QuD is--they are partially derived from it. This is far from an esoteric perspective, and appears in line with much past literature. What we want to propose instead, is the reverse perspective: assertions and their alternatives are the primitive, and help \textit{derive} potential QuDs (along with contrasts in pragmatic oddness).

\subsection{The Specificity Constraint}
\citet{Ippolito2019} then proposes that oddness arises from certain SSA configurations. In particular, sequences of sentences belonging to the same SSA are subject to a Specificity Constraint (henceforth \textbf{SC}), spelled out in (\ref{ex2:sc}). The SC states that the two alternatives in the sequence, should be dominated by the same number of nodes in their common SSA. This is equivalent to saying that two alternatives being contrasted should match in terms of their degree of specificity, or granularity.


\begin{exe}
	\ex {\textit{Specificity Condition \citep{Ippolito2019}.} A sequence $\Sigma = < [_{S_i}... \alpha_F ...], [_{S_j}... \beta_F ...] >$, s.t. both $S_i$ and $S_j$ are answers to the same QuD and $\beta$ is in the structured set of alternatives evoked by $\alpha$ ($T_{\mathcal{A}_{\alpha}}$), is felicitous if either:
		\begin{itemize}
			\item $\alpha$ or $\beta$ is the only node on its branch in $T_{\mathcal{A}_{\alpha}}$, or
			\item  $\alpha$ and $\beta$ are dominated by the same number of nodes in $T_{\mathcal{A}_{\alpha}}$.
		\end{itemize}
	}\label{ex2:sc}
\end{exe}

A sentence like (\ref{ex2:hurford-sw}) then violates the SC, because its ``\textit{all}'' and its ``\textit{some}'' disjunct are respectively dominated by $2$, and $1$ node in the corresponding SSA from Figure \ref{fig2:ssa-scalar}. The SC therefore correctly predicts (\ref{ex2:hurford-sw}) to be odd. But, because (\ref{ex2:hurford-ws}) only differs from (\ref{ex2:hurford-sw}) in how the disjuncts are ordered, the SC also incorrectly predicts (\ref{ex2:hurford-ws}) to be odd--at least in the absence of any additional assumptions.

The felicity of (\ref{ex2:hurford-ws}) is captured in \citet{Ippolito2019}'s framework based on the familiar idea that violations of the SC can be avoided by strengthening the weaker alternative \citep{Gazdar1979,Singh2008a,Singh2008b,Chierchia2009,Fox2018}. To retain the \textit{contrast} between  (\ref{ex2:hurford-sw}) and  (\ref{ex2:hurford-ws}), it is assumed that covert strengthening is governed by an economy condition, which disallows it in (\ref{ex2:hurford-sw}). This is shown to generalize to the a. and b. sequences in (\ref{ex2:sobel}-\ref{ex2:superlative}).

Even though the SC appears like a reasonable constraint, the deep reason why contrast alternatives with different degrees of specificity should be disallowed, remains relatively mysterious. In particular, the account does not directly relate the SC to general pragmatic principles based on competition \textit{between} sentences: the SC is a constraint that is only sensitive to the SSA associated with the target sentence, independently of the sentence's competitors and their own SSAs. In that respect, it remains close to Hurford's original constraint. Moreover, the constraint amounts to counting the number of parent nodes for each contrasted alternative, and as such is not sensitive to the relative positions of the two alternatives within their common SSA. This perspective might be slightly reductive, and would not capture the observation that oddness gets stronger if the two alternatives are in a dominance relation, as shown by gradience of the judgments in (\ref{ex2:hurford-gradient}).

\begin{exe}
	\ex\label{ex2:hurford-gradient}
	\begin{xlist}
		\ex[\#] {Jo grew up in Paris or France. \hfill Different specificity, dominance}
		\ex[?] {Jo grew up in Paris or Germany. \hfill Different specificity,no dominance}
		\ex[] {Jo grew up in France or Germany. \hfill Same specificity,no dominance}
	\end{xlist}
\end{exe}


In Chapter \ref{chap:redundancy}, we will propose a constraint akin in effect to the SC, but that will constitute a more direct extension of earlier \textsc{Redundancy}-based constraints used to capture Hurford Disjunctions. We will show how it applies to basic (non-scalar) Hurford Disjunctions and extends to another challenging family of intuitively redundant sentences. Chapters \ref{chap:scalarity} and \ref{chap:economy} will discuss the particular case of scalar Hurford Disjunctions like (\ref{ex2:hurford}), and extend the account to scalar Sobel sequences.


\section{Zhang's }



