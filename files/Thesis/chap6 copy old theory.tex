\chapter[Criss-crossing countries: oddness in non-scalar Hurford Conditionals]{Criss-crossing countries: oddness in non-scalar Hurford Conditionals\footnotemark}\label{chap:hurford-sentences}
\footnotetext{This Chapter constitutes a longer and hopefully more readable adaptation of \citettoappear{HenotMortier2024a}. I would like to thank the audience and reviewers of SuB29 and of the 2024 BerlinBrnoVienna Workshop for relevant questions, datapoints and suggestions regarding earlier iterations of this project.}

This Chapter extends the empirical landscape introduced in Chapter \ref{chap:hurford-disj}, which was mainly interested in Hurford Disjunctions, of the form $p \vee p^+$ or $p^+ \vee p$, with $p^+ \vDash p$. This Chapter is an investigation of Hurford \textit{Conditionals} \citep{Mandelkern2018,Kalomoiros2024}, of the form \textit{If $p$ then $\neg p^+$} or \textit{If $\neg p^+$ then $p$}, with $p^+ \vDash p$. Such conditionals are related to Hurford Disjunctions by the \textit{or}-to-\textit{if} tautology; but unlike their disjunctive counterparts, they exhibit a crisp asymmetry: variants in which the negated stronger proposition is in the antecedent (\textit{If $\neg p^+$ then $p$}) are degraded, while variants in which the negated stronger proposition is in the consequent (\textit{If $p$ then $\neg p^+$}), are fine. This Chapter explains this asymmetry, by building once again on the implicit QuD framework introduced in Chapter \ref{chap:accommodating-quds}, and by proposing a new constraint on Qtree derivation, dubbed \textsc{Incremental Q-Relevance}. This constraint will be shown to relate to Lewis's notion of \textsc{Relevance}, and to predict that the antecedent and consequent of conditionals should be ordered by increasing degrees of specificity. More broadly, this Chapter suggests that Hurford Disjunctions and Conditionals, display different ``flavors of oddness''.




\section{Introducing Hurford Conditionals}

Let us start by reviewing familiar data. As discussed in Chapter \ref{chap:hurford-disj}, Hurford Disjunctions (henceforth \textbf{HD}s), exemplified in (\ref{ex6:hd}), feature contextually entailing disjuncts ($p^+ \vDash_c p$), and are generally odd regardless of the order of their disjuncts \citep{Hurford1974}.\footnote{When the two disjuncts are the same modulo scalar expressions (e.g. $\langle$\textit{some}, \textit{all}$\rangle$) HDs \textit{may} be rescued from infelicity \citenp{Gazdar1979}; \citenp{Singh2008b}; \citenp{Fox2018}; \citenp{HenotMortier2023} i.a.). Chapter \ref{chap:economy} provides an overview of the challenges raised by ``scalar'' Hurford Sentences, and proposes an account elaborating on the framework introduced here.}

\begin{exe}
	\ex \label{ex6:hd}
	\begin{xlist}
		\ex[\#] {SuB29 will take place in Noto or in Italy. \hfill $\pplus\vee \p$}\label{ex6:hd-sw}
		\ex[\#] {SuB29 will take place in Italy or in Noto. \hfill $\p \vee \neg \pplus$}\label{ex6:hd-ws}
	\end{xlist}
\end{exe}

\citet{Mandelkern2018} observed that Hurford \textit{Conditionals} (henceforth \textbf{HC}s), exemplified in (\ref{ex6:hc}), are isomorphic\footnote{Specifically, isomorphy between two logical forms is defined as an equality of parse \textit{modulo} a variable change preserving logical relations (see (\ref{ex6:iso})).
\begin{exe}
	\ex {\textit{Isomorphy ($\approxeq$) between logical forms.} $x \approxeq y$ iff there is a substitution operation $\mathcal{S}$ targeting atomic propositions and preserving the logical relations between the elements in its domain ($aRb \iff \mathcal{S}(a)R\mathcal{S}(b)$, where $R$ denotes entailment, contradiction, or independence) s.t. $x$ and $\mathcal{S}(y)$ have same parse.}\label{ex6:iso}
	\end{exe}} to (\ref{ex6:hd-sw}) granted the \textit{or-to-if} tautology and double-negation elimination (see derivations in (\ref{ex6:hd-hc-derivation})). Yet, despite their isomorphy with (\ref{ex6:hd-sw}), (\ref{ex6:hc-sw}) and (\ref{ex6:hc-ws}) exhibit a crisp oddness asymmetry. Descriptively, it seems that the weaker item must be the antecedent, while the negated stronger item must be the consequent.

\begin{exe}
	\ex \label{ex6:hc}
	\begin{xlist}
		\ex[\#] {If SuB29 will not take place in Noto, it will take place in Italy. \hfill $\neg \pplus\rightarrow \p$}\label{ex6:hc-sw}
		\ex[] {If SuB29 will take place in Italy, it will not take place in Noto. \hfill $\p \rightarrow \neg \pplus$}\label{ex6:hc-ws}
	\end{xlist}
\end{exe}

\begin{exe}
	\ex \textit{Equivalence between the HD (\ref{ex6:hd-sw}) and HCs (assuming implications are material).} \label{ex6:hd-hc-derivation}
	\begin{xlist}
		\ex {(\ref{ex6:hc-sw}) $= \neg \pplus\rightarrow \p$ \\
			\phantom{(\ref{ex6:hc-sw})} $\equiv \neg(\neg \pplus)\vee \p$ \hfill (\textit{or}-to-\textit{if} tautology) \\
			\phantom{(\ref{ex6:hc-sw})} $\equiv \pplus\vee \p$ \hfill (double-$\neg$ elimination) \\
			\phantom{(\ref{ex6:hc-sw})} $=$ (\ref{ex6:hd-sw})}
		\ex {(\ref{ex6:hc-ws}) $= \p \rightarrow \neg \pplus$\\ \phantom{(\ref{ex6:hc-ws})} $\equiv (\neg \p) \vee (\neg \pplus)$ \hfill (\textit{or}-to-\textit{if} tautology)\\
			\phantom{(\ref{ex6:hc-ws})} $\equiv \qplus\vee \q$ \hfill ($\qplus := \neg\p$; $\q := \neg\pplus$; s.t. $\qplus \vDash \q$)\\
			\phantom{(\ref{ex6:hc-ws})} $\approxeq$ (\ref{ex6:hd-sw})}
	\end{xlist}
\end{exe}

As a side note, we do not intend to commit to a material analysis of the HCs explored in this Chapter. Similarly to Chapter \ref{chap:redundancy}, $\rightarrow$ will be used as a shorthand for \textit{if... then...}, throughout this Chapter, i.e. will not imply that the conditionals under consideration are necessarily considered material.



\citet{Kalomoiros2024} proposed the first solution to both the HDs (\ref{ex6:hd}) and  the HCs (\ref{ex6:hc}) (as well as other related datapoints). The approach was based on the idea that overt negation has a special status when it comes to evaluating if a sentence is redundant. This Chapter argues for an alternative view, building on the idea that disjunctions and conditionals evoke distinct QuDs (as already argued by Chapters \ref{chap:accommodating-quds} and \ref{chap:redundancy}). Additionally, this Chapter will assign a central role to the degree of granularity conveyed by propositions (consistent with the discussion in Chapter \ref{chap:accommodating-quds}), and identify the source of pragmatic oddness in HCs with ``granularity'' violations. This will be operationalized in terms of an incremental \textsc{Relevance} constraint. According to this constraint, the HC in (\ref{ex6:hc-sw}) will be predicted to be irrelevant, because its consequent (\textit{Italy}) evokes a by-country partition whose \textit{Italy} cell is not fully inside the \textit{not Noto} domain defined by the antecedent (some \textit{Italy}-worlds being \textit{Noto}-worlds). This is sketched in Figure \ref{tab:not-noto-italy}. The HC in (\ref{ex6:hc-ws}) on the other hand, will be correctly ruled-in, because its consequent (\textit{not Noto}) can evoke a by-city partition whose cells are each fully inside, or fully outside, the \textit{Italy}-domain defined by the antecedent. This is sketched in Figure \ref{tab:italy-not-noto}

\begin{figure}[H]
	\centering
	\begin{subfigure}[t]{.47\linewidth}
		\centering
		\begin{tabular}{|l|
				>{\columncolor{orange!20!white}}l 
				>{\columncolor{orange!20!white}}l ll|}
			\hline
			Noto                     & Rome                          & ...                                           & \cellcolor{orange!20!white}Paris & \cellcolor{orange!20!white}... \\ \hline
			\multicolumn{1}{|l!{\color{red}\vrule\vrule}}{\cellcolor{blue!20!white}} & {\cellcolor{blue!20!white}Italy} & \multicolumn{1}{l|}{\cellcolor{blue!20!white}} & \multicolumn{2}{l|}{...}                                    \\ \hline
		\end{tabular}
		\caption{(\ref{ex6:hc-sw})'s consequent (\textit{Italy}) evokes a by-country partition whose \textit{Italy} cell is not fully inside the \textit{not Noto} domain defined by (\ref{ex6:hc-sw})'s antecedent (see red separation line).}\label{tab:not-noto-italy}
	\end{subfigure}
	\hfill
	\begin{subfigure}[t]{.47\linewidth}
		\centering
		\begin{tabular}{|l|
				>{\columncolor{blue!20!white}}l 
				>{\columncolor{blue!20!white}}l ll|}
			\hline
			\multicolumn{1}{|l}{\cellcolor{blue!20!white}} & Italy                                             &                             & \multicolumn{2}{|l|}{...}                                                         \\ \hline
			Noto                                           & \multicolumn{1}{l|}{\cellcolor{orange!20!white}Rome} & \cellcolor{orange!20!white}... & \multicolumn{1}{|l|}{\cellcolor{orange!20!white}Paris} & \cellcolor{orange!20!white}... \\ \hline
		\end{tabular}
		\caption{(\ref{ex6:hc-ws})'s consequent (\textit{not Noto}) evokes a by-city partition whose cells are each fully inside, or fully outside, the \textit{Italy}-domain defined by (\ref{ex6:hc-ws})'s antecedent.}\label{tab:italy-not-noto}
	\end{subfigure}
\caption{Illustration of the granularity violation in (\ref{ex6:hc-sw}), and of the absence of such a violation in (\ref{ex6:hc-ws}).}
\end{figure}


This Chapter is structured as follows. Section \ref{sec6:prev-approaches} provides an overview of \citeauthor{Kalomoiros2024}'s approach to HCs, and outline some of its limitations. Section \ref{sec6:machinery} introduces the compositional machinery used to derive potential QuDs out of assertions. Section \ref{sec6:constraints} defines \textsc{Incremental Q-Relevance} on LF-QuD pairs and shows how this constraint captures the HCs in (\ref{ex6:hd}) and (\ref{ex6:hc}). Section \ref{sec6:ccl} shows how this approach may capture more elaborate variants of HCs, while outlining remaining issues and questions.


\section{Existing account}\label{sec6:prev-approaches}
\citet{Mandelkern2018} show that HCs are problematic for virtually all accounts of HDs and their variants proposed before \citet{Kalomoiros2024}. Therefore, we will not review these approaches here, and directly jump to  \citeauthor{Kalomoiros2024}'s recent proposal.



\subsection{Super-Redundancy}
%HDs feel redundant; while HCs sound locally irrelevant.
%Talk about repairs: the fact the repairs are differnt suggets the violation stems from a different source.
\citeauthor{Kalomoiros2024}' \textsc{Super-Redundancy}, repeated in (\ref{ex4:sr}) from Chapter \ref{chap:redundancy}, states that a sentence $S$ is super-redundant if it features a binary operation taking a constituent $C$ as argument, and moreover there is no way of strengthening $C$ to $C^+$ that would make the resulting sentence $S^+$ non-redundant (i.e., non-equivalent to its counterpart where $C^+$ got deleted).

\begin{exe}
	\exr{ex4:sr} {\textsc{Super-Redundancy.} A sentence $S$ is infelicitous if it contains $C \ast C'$ or $C' \ast C$, with $\ast$ a binary operation, s.t. $(S)^-_C$ is defined and for all $D$, $(S)^-_C \equiv S_{Str(C, D)}$. In this definition:
		\begin{itemize}
			\item $(S)^-_C$ refers to $S$ where $C$ got deleted;
			\item  $Str(C, D)$ refers to a strengthening of $C$ with $D$, defined inductively and whose key property is that it commutes with negation ($Str(\neg\alpha, D) = \neg (Str(\alpha, D))$), as well as with binary operators ($Str(O(\alpha, \beta), D) = O(Str(\alpha, D), Str(\beta, D))$);
			\item $S_{Str(C, D)}$ refers to $S$ where $C$ is replaced by $Str(C, D)$.
	\end{itemize}}
\end{exe}

%HDs feel redundant; while HCs sound locally irrelevant.
%Talk about repairs: the fact the repairs are differnt suggets the violation stems from a different source.

As already shown in Chapter \ref{chap:hurford-disj}, this constraint can capture HDs (\ref{ex6:hd}). The proof, adapted from \citet{Kalomoiros2024}, is repeated in (\ref{ex5:hd-sr}).

\begin{exe}
	\exr{ex5:hd-sr} {HDs are Super Redundant (SR).\\
		We show (\ref{ex5:hd-sw})=$\pplus\vee\p$ and (\ref{ex5:hd-ws})=$\p\vee\pplus$ are SR.\\
		In either case, take C = \pplus.\\
		We then have (\ref{ex5:hd-sw})$^-_C$ = (\ref{ex5:hd-ws})$^-_C$ = $\p$\\
		$\forall D. \ (\textref{ex5:hd-sw})_{Str(C, D)} = (\textref{ex5:hd-ws})_{Str(C, D)} =  (\pplus \wedge D) \vee \p$\\
		\phantom{$\forall D. \ (\textref{ex5:hd-sw})_{Str(C, D)} = (\textref{ex5:hd-ws})_{Str(C, D)}$} $\equiv (\pplus \vee \p) \wedge (D \vee \p)$\\
		\phantom{$\forall D. \ (\textref{ex5:hd-sw})_{Str(C, D)} = (\textref{ex5:hd-ws})_{Str(C, D)}$} $\equiv \p \wedge (D \vee \p)$\\
		\phantom{$\forall D. \ (\textref{ex5:hd-sw})_{Str(C, D)} = (\textref{ex5:hd-ws})_{Str(C, D)}$} $\equiv (\p \wedge D) \vee \p$\\
		\phantom{$\forall D. \ (\textref{ex5:hd-sw})_{Str(C, D)} = (\textref{ex5:hd-ws})_{Str(C, D)}$} $\equiv \p = (\textref{ex5:hd-sw})^-_C = (\textref{ex5:hd-ws})^-_C$ 
	}
\end{exe}

More interestingly perhaps, (\ref{ex4:sr}) also captures HCs, whether conditionals are assumed to be material, or strict. The proofs assuming material conditionals, are given in (\ref{ex6:hc-sw-sr}) for (\ref{ex6:hc-sw}) and (\ref{ex6:hc-ws-sr}) for (\ref{ex6:hc-ws}). In both cases, it is crucial that the local strengthening of $C=p^+$ be conjunctive \textit{under} negation (and thus, disjunctive after applying De Morgan's law). In (\ref{ex6:hc-sw-sr}), this allows to remove $p^+$ from the chain of logical equivalences, and eventually derive Super-Redundancy. In the second part of (\ref{ex6:hc-ws-sr}) when $C=\neg p^+$, this ensures that the strengthening $D$ \textit{can} be disregarded, and that the equivalence does \textit{not} obtain -- eventually deriving a failure of \textsc{Super-Redundancy}. \citet{Kalomoiros2024} also shows that this account extends to strict (yet not variably strict) conditionals. We omit the proof here for brevity.

\begin{exe}
	\ex\label{ex6:hc-sw-sr} {Assuming implications are material, ``strong-to-weak'' HCs like (\ref{ex6:hc-sw}) are Super Redundant (SR).\\
		We show (\ref{ex6:hc-sw})=$\neg\pplus\rightarrow\p$ is SR.\\
		Take C = $\neg\pplus$.\\
		We then have (\ref{ex6:hc-sw})$^-_C$ = $\p$.\\
		$\forall D. \ (\textref{ex6:hc-sw})_{Str(C, D)} = \neg(\pplus \wedge D) \rightarrow \p$\\
		\phantom{$\forall D. \ (\textref{ex6:hc-sw})_{Str(C, D)}$} $\equiv (\pplus \wedge D) \vee \p$\\
		\phantom{$\forall D. \ (\textref{ex6:hc-sw})_{Str(C, D)}$} $\equiv (\pplus \vee \p) \wedge (D \vee \p)$\\
		\phantom{$\forall D. \ (\textref{ex6:hc-sw})_{Str(C, D)}$} $\equiv \p \wedge (D \vee \p)$\\
		\phantom{$\forall D. \ (\textref{ex6:hc-sw})_{Str(C, D)}$} $\equiv \p \wedge (D \vee \p)$\\
		\phantom{$\forall D. \ (\textref{ex6:hc-sw})_{Str(C, D)}$} $\equiv \p = (\textref{ex6:hc-sw})^-_C$ 
	}
	\ex\label{ex6:hc-ws-sr} {Assuming implications are material, ``weak-to-strong'' HCs like (\ref{ex6:hc-ws}) are not Super Redundant (SR).\\
		We show (\ref{ex6:hc-ws})=$\p\rightarrow\neg\pplus$ is not SR.\\
		Take C = (\ref{ex6:hc-ws})'s antecedent = \p.\\
		We then have (\ref{ex6:hc-ws})$^-_C$ = $\neg\pplus$.\\
		Take $D=\bot$.\\
		$(\textref{ex6:hc-ws})_{Str(C, D)} =  (\p \wedge D) \rightarrow (\neg\pplus) $\\
		\phantom{$(\textref{ex6:hc-ws})_{Str(C, D)}$} $\equiv (\p \wedge \bot) \rightarrow (\neg\pplus)$\\
		\phantom{$(\textref{ex6:hc-ws})_{Str(C, D)}$} $\equiv \bot \rightarrow (\neg\pplus)$\\
		\phantom{$(\textref{ex6:hc-ws})_{Str(C, D)}$} $\equiv \top$\\ \phantom{$(\textref{ex6:hc-ws})_{Str(C, D)}$} $\not\equiv \neg\pplus = (\textref{ex6:hc-ws})^-_C$\\
		Take C = (\ref{ex6:hc-ws})'s consequent = $\neg\pplus$.\\
		We then have (\ref{ex6:hc-ws})$^-_C$ = $\p$.\\
		Take $D=\top$.\\
		$(\textref{ex6:hc-ws})_{Str(C, D)} =  \p \rightarrow (\neg(\pplus \wedge D)) $\\
		\phantom{$(\textref{ex6:hc-ws})_{Str(C, D)}$} $\equiv \p \rightarrow (\neg(\pplus \wedge \top))$\\
		\phantom{$(\textref{ex6:hc-ws})_{Str(C, D)}$} $\equiv \p \rightarrow (\neg\pplus)$\\
		\phantom{$(\textref{ex6:hc-ws})_{Str(C, D)}$} $\equiv (\neg\p) \vee (\neg\pplus)$\\
		\phantom{$(\textref{ex6:hc-ws})_{Str(C, D)}$} $\equiv \neg\pplus$\\
		\phantom{$(\textref{ex6:hc-ws})_{Str(C, D)}$} $\not\equiv \p = (\textref{ex6:hc-ws})^-_C$\\
		
	}
\end{exe}


This approach is compelling regarding its empirical coverage, but raises one conceptual interrogation. While earlier approaches to \textsc{Redundancy} (\citenp{Meyer2013}; \citenp{Katzir2014}; \citenp{Mayr2016} i.a.) link it to the concept of \textsc{Brevity} in the sense of \citet{Grice1975}, it remains unclear, under the \textsc{Super-Redundancy} view, why the notion of local strengthening is defined the way it is (in particular when it comes to its commuting with negation), and why it should be so central in deriving oddness. 

The next Section adds to this an empirical concern, by presenting data suggesting that overt negation may not be the only source of the contrast in (\ref{ex6:hc}).



\subsection{Is overt negation really the culprit in HCs?}

\textsc{Super-Redundancy} was originally motivated by the observation that ``disjunctwise'' negated HDs, like (\ref{ex6:hd-neg}), appear felicitous. 

\begin{exe}
	\ex {Context (taken from \citenp{Kalomoiros2024}): we go into John's office and see a full pack of Marlboros in the dustbin. We are entertaining hypotheses about what's going on.\\
		John either doesn't smoke or he doesn't smoke Marlboros. \hfill $(\neg\p) \vee (\neg\pplus)$}\label{ex6:hd-neg}
\end{exe}

In this Section, we show that slight variations of (\ref{ex6:hd-neg}) display unexpected downgrades in felicity. Moreover, such downgrades can in turn be mitigated by certain operators or expressions. We suggest that the entire paradigm may be better explained by assuming that (\ref{ex6:hd-neg}) should in principle be deemed odd, but also that additional pragmatic processes, should be able to rescue it and its variants, only under certain conditions.\\

First, let us double-check that (\ref{ex4:sr}) predicts (\ref{ex6:hd-neg}) to be fine. This is done in (\ref{ex6:hd-neg-sr}).

\begin{exe}
	\ex\label{ex6:hd-neg-sr} {``Disjunctwise'' negated HDs are not Super Redundant (SR).\\
		We show (\ref{ex6:hd-neg})=$(\neg\p)\vee(\neg\pplus)$ is SR.\\
		Take C = $\neg\p$.\\
		We then have (\ref{ex6:hd-neg})$^-_C$ = $\neg\pplus$.\\
		Take $D=\bot$.\\
		$(\textref{ex6:hd-neg})_{Str(C, D)} = (\neg(\p \wedge D)) \vee (\neg\pplus)$\\
		\phantom{$(\textref{ex6:hd-neg})_{Str(C, D)}$} $\equiv (\neg(\p \wedge \bot)) \vee (\neg\pplus)$\\
		\phantom{$(\textref{ex6:hd-neg})_{Str(C, D)}$} $\equiv (\neg\bot) \vee (\neg\pplus)$\\
		\phantom{$(\textref{ex6:hd-neg})_{Str(C, D)}$} $\equiv \top \vee (\neg\pplus)$\\
		\phantom{$(\textref{ex6:hd-neg})_{Str(C, D)}$} $\equiv \top$\\
		\phantom{$(\textref{ex6:hd-neg})_{Str(C, D)}$} $\not\equiv \neg\pplus = (\textref{ex6:hd-neg})^-_C$\\
		Take C = $\neg\pplus$.\\
		We then have (\ref{ex6:hd-neg})$^-_C$ = $\neg\p$.\\
		Take $D=\top$.\\
		$(\textref{ex6:hd-neg})_{Str(C, D)} = (\neg(\pplus \wedge D)) \vee (\neg\p)$\\
		\phantom{$(\textref{ex6:hd-neg})_{Str(C, D)}$} $\equiv (\neg(\pplus \wedge \top)) \vee (\neg\p)$\\
		\phantom{$(\textref{ex6:hd-neg})_{Str(C, D)}$} $\equiv (\neg\pplus) \vee (\neg\p)$\\
		\phantom{$(\textref{ex6:hd-neg})_{Str(C, D)}$} $\not\equiv \neg\p = (\textref{ex6:hd-neg})^-_C$
	}
\end{exe}

This fact should be unsurprising given that the HD in (\ref{ex6:hd-neg}) is equivalent to the HC (\ref{ex6:hc-ws}) granted to \textit{or}-to-\textit{if} tautology, and that we already showed in (\ref{ex6:hc-ws-sr}) that (\ref{ex6:hc-ws}) is not Super-Redundant assuming implications are material.\\

We now show that slight variations of (\ref{ex6:hd-neg}) displays felicity downgrades that can be mitigated by certain operators or expressions. First, (\ref{ex6:hd-neg}) becomes degraded if its two disjuncts get swapped, as done in (\ref{ex6:hd-neg-swap-no-at-all}). This is \textit{not} expected under \textsc{Super-Redundancy}, whose predictions are insensitive to the order of the disjuncts. Interestingly, adding \textit{at all} to the second disjunct of (\ref{ex6:hd-neg-swap-no-at-all}), recovers felicity, as shown in (\ref{ex6:hd-neg-swap-at-all}). 

\begin{exe}
	\ex \label{ex6:hd-neg-swap}
	\begin{xlist}
		\ex[\#] {John either doesn't smoke Marlboros or he doesn't smoke.}\label{ex6:hd-neg-swap-no-at-all}
		\ex[] {John either doesn't smoke Marlboros or he doesn't smoke at all.}\label{ex6:hd-neg-swap-at-all}
	\end{xlist}
\end{exe}

Although we do intend not provide a full-fledged account of the effect of \textit{at all} here,\footnote{Here is an intuition however. \textit{At all} seems to make the question whether John smokes ($p$ vs. $\neg p$) more salient, and as such may force the $\neg p$ alternative to be considered when attempting exhaustification on the first disjunct $\neg p^+$. This would eventually make the two disjuncts contradictory, and rescue (\ref{ex6:hd-neg-swap-no-at-all}) from oddness. See footnote \ref{fn:exh-marlboros} for a more in-depth discussion of the role of covert exhaustification in the sentence at stake, specifically in the subsequent focused variant (\ref{ex6:hd-neg-focus}).}
we believe the paradigm formed by (\ref{ex6:hd-neg}) and (\ref{ex6:hd-neg-swap}), indicates that some additional incremental pragmatic mechanism is at play in disjunctwise negated HDs -- meaning, (\ref{ex6:hd-neg}) may be deemed deviant \textit{a priori}, but may end up being rescued by some extra pragmatic mechanism made unavailable in e.g. (\ref{ex6:hd-neg-swap-no-at-all}). This in turn, suggests that the felicity of (\ref{ex6:hd-neg}) should not necessarily be accounted for by a \textsc{Non-Redundancy} constraint.

Another observation in line with this hypothesis, is that (\ref{ex6:hd-neg}) is significantly improved by focus, as shown in (\ref{ex6:hd-neg-focus}). 


\begin{exe}
	\ex {(Either) John doesn't smoke or doesn't smoke MARLBOROS.}\label{ex6:hd-neg-focus}
\end{exe}

Our pre-theoretical understanding of the effect of focus in (\ref{ex6:hd-neg-focus}), is that not smoking MALRBOROS implies smoking cigarettes different from Marlboros, i.e. smoking still.\footnote{\label{fn:exh-marlboros}This may be backed by the theory, as well, assuming that focus forces covert exhaustification \textit{via} the operator \textit{exh} \citep{Fox2007,Chierchia2009}, and that $\neg p$ (\textit{John does not smoke}) is a salient alternative to $\neg p^+$ (\textit{John does not smoke Marlboros}) in (\ref{ex6:hd-neg-focus}). In that case, the enriched meaning of $\neg p^+$ would end up being $\neg p^+ \wedge \neg\neg p \equiv \neg p^+ \wedge p$, i.e. that \textit{John does not smoke Marlboros, but does smoke}. Interestingly, this kind of inference licensed by \textit{exh}, should be unavailable in (\ref{ex6:hd-neg-swap-no-at-all}) -- even when \textit{Marlboros} gets focused -- in order to capture the infelicity of this sentence. This could be ensured by assuming that relevant alternatives are somehow incrementally computed, and that $\neg p$ is not a relevant alternative to $\neg p^+$ ``out-of-the-blue'' i.e. if $\neg p^+$ is not preceded by $\neg p$ (and, e.g. appears in the first disjunct of a disjunction).} If this is indeed the case, (\ref{ex6:hd-neg-focus}) would end up meaning $\neg p \vee q$ with $q \equiv \neg p^+ \wedge p$. Descriptively, this disjunction features incompatible disjuncts, so does not violate \citeauthor{Hurford1974}'s original condition. It is also predicted by most if not all accounts of oddness to be fine. Assuming that whatever focus achieves in (\ref{ex6:hd-neg-focus}), may also be achieved covertly and without explicit focus in (\ref{ex6:hd-neg}), would explain (\ref{ex6:hd-neg})'s felicity independently of \textsc{Super-Redundancy}. We believe the pattern described here gets even crisper if disjuncts are picked to be more parallel\footnote{For instance, instead of having V and V+NP as disjuncts, we can have V+NP and V+NP$^+$, with $\llbracket \text{NP}^+\rrbracket \subset \llbracket \text{NP}\rrbracket$).
	\begin{exe}
		\ex {Analogs of (\ref{ex6:hd-neg}) (original sentence), (\ref{ex6:hd-neg-swap-no-at-all}) (swapped disjuncts), and (\ref{ex6:hd-neg-swap-at-all}) (swapped disjuncts, plus \textit{at all}), respectively.}
		\begin{xlist}
			\ex[?] {John either doesn't own a dog or he doesn't own a lab.}\label{ex6:hd-neg-original}
			\ex[\#] {John either doesn't own a lab or he doesn't own a dog.}\label{ex6:hd-neg-swapped}
			\ex[] {John either doesn't own a lab or he doesn't own a dog at all.}\label{ex6:hd-neg-swapped-at-all}
		\end{xlist}
		\ex Analog of (\ref{ex6:hd-neg-focus}) (focused $p^+$).
		\begin{xlist}
			\ex {John either doesn't own a dog or doesn't own a LAB.}
		\end{xlist}
\end{exe}}\\


Second, we note that (\ref{ex6:hd-neg}) is made worse by removing \textit{either}; see (\ref{ex6:hd-neg-no-either}). (\ref{ex6:hd-neg-no-either-at-all}) shows that adding \textit{at all} to the stronger disjunct ($\neg p$) restores felicity.

\begin{exe}
	\ex
	\begin{xlist}	
		\ex[??] {John doesn't smoke or doesn't smoke Marlboros.}\label{ex6:hd-neg-no-either}
		\ex[] {John doesn't smoke at all or doesn't smoke Marlboros.}\label{ex6:hd-neg-no-either-at-all}
	\end{xlist}
\end{exe}

Again, we do not wish to propose a full-fledged account of the effect of \textit{either} in (\ref{ex6:hd-neg}). But let us just observe that removing \textit{either} in other sentences leads to the same kind of degradation. Such sentences, dubbed \textit{bathroom sentences} \citep{Evans1977} and attributed to Barbara Partee, are exemplified in (\ref{ex6:bathroom-either}). 

\begin{exe}
	\ex \label{ex6:bathroom}
	\begin{xlist}
		\ex[] {Either there is no bathroom or it's upstairs.}\label{ex6:bathroom-either}
		\ex[??] {There is no bathroom or it's upstairs.}\label{ex6:bathroom-no-either}
		\ex[] {Either there is no bathroom or there is a bathroom and it's upstairs.}\label{ex6:bathroom-either-overt}
	\end{xlist}
\end{exe}
Roughly, (\ref{ex6:bathroom-either}) requires its second disjunct to be interpreted given the negation of its first disjunct to be felicitous. This is because the pronoun \textit{it} in (\ref{ex6:bathroom-either})'s second disjunct, requires an antecedent, which is not overtly introduced in (\ref{ex6:bathroom-either}), but could be provided by an existential statement of the form \textit{there is a bathroom}, which correspond to the negation of (\ref{ex6:bathroom-either})'s first disjunct. So, very roughly, (\ref{ex6:bathroom-either}) could be felicitous, if understood as (\ref{ex6:bathroom-either-overt}), which has the form $\neg p \vee (p \wedge q_p)$, where $q_p$ means that $q$ presupposes $p$. This is quite similar to the possible pragmatic strengthening of (\ref{ex6:hd-neg})'s second disjunct with $p$, which we argued made this sentence felicitous. Removing \textit{either} in both disjunctwise negated HDs and bathroom sentences, could be argued to prevent this rescue mechanism \footnote{This in fact would be in line with the idea that \textit{either} somehow forces exclusivity between disjuncts (\citenp{Nicolae2025} i.a.).} -- leading to a degradation, as shown in (\ref{ex6:hd-neg-no-either}) and (\ref{ex6:bathroom-no-either}) respectively.\\


Let us now take stock and review the implications for HCs. We have just seen that disjunctwise negated HDs like (\ref{ex6:hd-neg}), which constitute the basis of the argument supporting the \textsc{Super-Redundancy} approach, may be felicitous reasons independent of \textsc{Non-Redundancy}. Specifically, we provided additional data suggesting that independent pragmatic mechanism(s) may force the weaker disjunct of (\ref{ex6:hd-neg}) to contradict the stronger one, and that such mechanisms may be blocked or forced, when considering specific variants of (\ref{ex6:hd-neg}). In fact, even ignoring such variants, we can observe that the felicity of (\ref{ex6:hd-neg}) seems only guaranteed when a precise context is set up; but doing so may force a specific kind of QuD, and such a move was shown to improve non-negated HDs just as well \citep{Haslinger2023}. Besides, we can note that \cite{Kalomoiros2024}'s \textsc{Super-Redundancy} is challenged outside of the domain of HDs and HCs, by other varieties of redundant sentences obtained from the structure $p\vee p \vee q$ \textit{via} the \textit{or-to-if} tautology (see Chapter \ref{chap:redundancy}).\\

While the tentative explanations laid out here do not fully explain the complex patterns reported, we believe the overall data to be more in line with an analysis which, unlike \textsc{Super-Redundancy}, would not assign a key role to overt negation in HDs and HCs, but instead, would interact with pragmatic processes themselves influenced by negation and incrementality. In our alternative proposal, we will in fact suggest that \textit{granularity} differences (e.g., \textit{Paris} being finer-grained than \textit{France}, \textit{smoking Marlboros} being more fine-grained than \textit{smoking}), drive the contrast in (\ref{ex6:hc}).



\section{QuD evoked by Hurford Conditionals}\label{sec6:machinery}


Building on the Qtree model presented in Chapter \ref{chap:accommodating-quds}, we compute Qtrees evoked by sentences like (\ref{ex6:hc-sw}) and (\ref{ex6:hc-ws}), repeated below. 

\begin{exe}
	\exr{ex6:hc}
	\begin{xlist}
		\ex[\#] {If SuB29 will not take place in Noto, it will take place in Italy. \hfill $\neg \pplus\rightarrow \p$}
		\ex[] {If SuB29 will take place in Italy, it will not take place in Noto. \hfill $\p \rightarrow \neg \pplus$}
	\end{xlist}
\end{exe}

Crucial for this Section will be the idea that conditional evoke Qtrees which assign asymmetric roles to antecedent and consequent: a conditional Qtree is a Qtree for the antecedent whose verifying nodes are replaced by their intersection with a Qtree for the consequent. We will see towards the end of this Section that this asymmetry may be exploited to derive a descriptive generalization about conditionals, namely that their consequent should evoke some Qtree that is finer-grained than some Qtree evoked by the antecedent. 

\subsection{Qtrees for the antecedent and consequent of HCs}

We first compute the Qtrees compatible with $S_p = $ \textit{SuB29 will take place in Italy}, and $S_{p^+} = $ \textit{SuB29 will take place in Noto}. The Qtrees for $\neg S_{p^+} = $ \textit{SuB29 will not take place in Noto} will be subsequently derived from those evoked by $S_{p^+}$.

Chapter \ref{chap:accommodating-quds} extensively discussed how to derive Qtrees from simplex sentences like $S_p$ and $S_{p^+}$. And Chapter \ref{chap:hurford-disj} in fact discussed the Qtrees evoked by these exact sentences. Here, it is enough to say that such sentences may evoke three kinds of Qtrees: ``polar'' ones, splitting the Context Set (henceforth \textbf{CS}) into $p$ and $\neg p$ worlds; ``\textit{wh}'' ones, splitting the CS according to the Hamblin partition generated by same-granularity alternatives to the prejacent; and ``\textit{wh}-articulated'' ones, whereby each layer corresponds to a Hamblin partition of increasing granularity from the top down, the last layer matching the granularity of the prejacent. In each case, leaves entailed by the prejacent are \setlength{\fboxsep}{1pt}\fbox{flagged} as ``verifying'', and as such keep track of at-issue content. In this Chapter, and just like in Chapter \ref{chap:hurford-disj}, we will only consider two levels of granularity for $S_p$ and $S_{p^+}$: by-city and by-country. This gives rise to the Qtrees in Figure \ref{fig6:qtrees-noto} (for $S_{p^+}$) and Figure \ref{fig6:qtrees-italy} (for $S_{p}$). 

\begin{figure}[H]
	\centering
	\begin{subfigure}[b]{.2\linewidth}
		\centering
		\scalebox{1}{
			\begin{forest}
				[CS [\fbox{\textcolor{orange}{Noto}}] [$\neg$\textcolor{orange}{Noto}]]
			\end{forest}
		}\caption{``Polar''}\label{fig6:qtree-noto-polar}
	\end{subfigure}\hfill
	\begin{subfigure}[b]{.37\linewidth}
		\centering
		\scalebox{1}{
			\begin{forest}
				[CS [\fbox{\textcolor{orange}{Noto}}] [\textcolor{orange}{Rome}][\textcolor{orange}{Paris}][\textcolor{orange}{...}]]
			\end{forest}
		}\caption{``\textit{Wh}''}\label{fig6:qtree-noto-wh}
	\end{subfigure}\hfill
	\begin{subfigure}[b]{.37\linewidth}
		\centering
		\scalebox{1}{
			\begin{forest}
				[CS[\textcolor{blue}{Italy} [\fbox{\textcolor{orange}{Noto}}] [\textcolor{orange}{Rome}] [\textcolor{orange}{...}]] [\textcolor{blue}{France}[\textcolor{orange}{Paris}][\textcolor{orange}{...}]] [\textcolor{blue}{...}]]
			\end{forest}
		}\caption{``\textit{Wh}-articulated''}\label{fig6:qtree-noto-tiered}
	\end{subfigure}
	\caption[]{Qtrees evoked by $S_{\pplus}$ = \textit{SuB29 will take place in Noto.}}\label{fig6:qtrees-noto}
\end{figure}
\begin{figure}[H]
	\centering
	
	\begin{subfigure}[b]{.45\linewidth}
		\centering
		\scalebox{1}{
			\begin{forest}
				[CS [\fbox{\textcolor{blue}{Italy}}] [$\neg$\textcolor{blue}{Italy}]]
			\end{forest}
		}\caption{``Polar''}\label{fig6:qtree-italy-polar}
	\end{subfigure}\hfill
	\begin{subfigure}[b]{.45\linewidth}
		\centering
		\scalebox{1}{
			\begin{forest}
				[CS [\fbox{\textcolor{blue}{Italy}}] [\textcolor{blue}{France}][\textcolor{blue}{...}]]
			\end{forest}
		}\caption{``\textit{Wh}''}\label{fig6:qtree-italy-wh}
	\end{subfigure}
	\caption[]{Qtrees evoked by $S_{\p}$ = \textit{SuB29 will take place in Italy.}}\label{fig6:qtrees-italy}
\end{figure}

We can already note that Figures \ref{fig6:qtree-noto-tiered} and \ref{fig6:qtree-italy-wh}, introduce consistent partitionings: Figure \ref{fig6:qtree-noto-tiered} can be in fact be seen as a refinement of Figure \ref{fig6:qtree-italy-wh}, as per (\ref{ex2:qtree-refinement}), repeated below.

\begin{exe}
	\exr{ex2:qtree-refinement} {\textit{Qtree refinement.} Let $T$ and $T'$ be Qtrees. $T$ is a refinement of $T'$ (or: $T$ is finer-grained than $T'$), iff $T'$ can be obtained from $T$ by removing a subset $\mathcal{T}$ of $T$'s subtrees, s.t., if $\mathcal{T}$ contains a subtree rooted in $N$, then, for each node $N'$ that is a sibling of $N$ in $T$, the subtree of $T$ rooted in $N'$, is also in $\mathcal{T}$.}
\end{exe}

Figures \ref{fig5:qtree-noto-tiered} and \ref{fig5:qtree-italy-wh} thus structurally capture the intuition that $S_{p^+}$ answers a finer-grained question than $S_p$. More generally, Figures \ref{fig6:qtrees-noto} and \ref{fig6:qtrees-italy} show that some Qtree obtained for $S_{p^+}$ (namely Figure \ref{fig5:qtree-noto-tiered}), refines some Qtree for $S_p$ (namely, Figure \ref{fig5:qtree-italy-wh}); while no Qtree obtained for $S_p$ refines a Qtree for $S_{p^+}$.\\

Before moving on to computing the conditional Qtrees for (\ref{ex6:hc-sw}) and (\ref{ex6:hc-ws}), let us compute the Qtree corresponding to the negation of $S_{p^+}$, namely $\neg S_{p^+} = $ \textit{SuB29 will not take place in Noto}, which constitutes the antecedent of (\ref{ex6:hc-sw}) and the consequent of (\ref{ex6:hc-ws}). As discussed in Chapter \ref{chap:accommodating-quds}, a negated LF is assumed to evoke the same kind of question as its positive counterpart, but flags a disjoint set of verifying nodes. More specifically, given an LF $X$, evoking a Qtree $T$, a Qtree $T'$ for $\neg X$ is obtained by retaining $T$'s structure (nodes and edges), and ``swapping'' $T$'s verifying nodes, by replacing any set of same-level verifying nodes in $T$ by the set of non-verifying nodes at the same level in $T$. If the verifying nodes are all leaves, this operation simply corresponds to set complementation in the domain of leaves. This is done for $\neg S_{p^+}$ in Figure \ref{fig6:qtrees-not-noto}.

\begin{figure}[H]
	\centering
	\begin{subfigure}[b]{.2\linewidth}
		\centering
		\scalebox{1}{
			\begin{forest}
				[CS [{\textcolor{orange}{Noto}}] [\fbox{$\neg$\textcolor{orange}{Noto}}]]
			\end{forest}
		}\caption{``Polar''}\label{fig6:qtree-not-noto-polar}
	\end{subfigure}\hfill
	\begin{subfigure}[b]{.37\linewidth}
		\centering
		\scalebox{1}{
			\begin{forest}
				[CS [{\textcolor{orange}{Noto}}] [\fbox{\textcolor{orange}{Rome}}][\fbox{\textcolor{orange}{Paris}}][\fbox{\textcolor{orange}{...}}]]
			\end{forest}
		}\caption{``\textit{Wh}''}\label{fig6:qtree-not-noto-wh}
	\end{subfigure}\hfill
	\begin{subfigure}[b]{.37\linewidth}
		\centering
		\scalebox{1}{
			\begin{forest}
				[CS[\textcolor{blue}{Italy} [{\textcolor{orange}{Noto}}] [\fbox{\textcolor{orange}{Rome}}] [\fbox{\textcolor{orange}{...}}]] [\textcolor{blue}{France}[\fbox{\textcolor{orange}{Paris}}][\fbox{\textcolor{orange}{...}}]] [\textcolor{blue}{...}]]
			\end{forest}
		}\caption{``\textit{Wh}-articulated''}\label{fig6:qtree-not-noto-tiered}
	\end{subfigure}
	\caption[]{Qtrees evoked by $\neg S_{\pplus}$ = \textit{SuB29 will not take place in Noto.}}\label{fig6:qtrees-not-noto}
\end{figure}

Because negation preserves Qtree structure and only affects verifying nodes, Figure \ref{fig6:qtree-not-noto-tiered}, just like Figure \ref{fig6:qtree-noto-tiered}, constitutes a refinement of Figure \ref{fig6:qtree-italy-wh}. More broadly, our observation about $S_p$ and $S_{p^+}$ extends to $S_p$ and $\neg S_{p^+}$: some Qtree obtained for $\neg S_{p^+}$ (namely Figure \ref{fig6:qtree-not-noto-tiered}), refines some Qtree for $S_p$ (namely, Figure \ref{fig6:qtree-italy-wh}); while no Qtree obtained for $S_p$ refines a Qtree for $\neg S_{p^+}$.\\

This double observation will be crucial for our approach to HCs: felicitous HCs like (\ref{ex6:hc-ws}) are the ones whose antecedent evokes a question that is coarser-grained than that of their consequent (i.e. s.t. the antecedent Qtree \textit{can} be refined by a consequent Qtree); odd HCs like (\ref{ex6:hc-sw}) are the ones whose antecedent evokes a question that is finer-grained than that of their consequent (i.e. s.t. the antecedent Qtree \textit{cannot} be refined by any consequent Qtree).

%\footnote{This approach is perhaps a bit naive; uttering $p$ vs. $\neg p$, does not seem to preferentially answer the same kind of question, i.e. evoke the same kind of Qtree structure. More specifically, it seems that uttering negative statements in general conveys the idea that the original question was a polar question of the form \textit{whether p?} -- more than a \textit{wh} kind of question. This observation can be related to informativity: uttering $\neg p$ when the question is \textit{whether p?}, is maximally informative, because it identifies one single cell -- the $\neg p$-cell. Uttering $\neg p$ when the questions is e.g. \textit{p, q, or r?}, is underinformative, because it does \textit{not} identify a single cell. To account for this, one might want to say that Qtrees ar ranked according to how well they are addressed by the assertion evoking them -- Qtree with smaller sets of verifying nodes should be preferred.}



\subsection{General recipe for conditional Qtrees, and one useful result}

Let us now turn to the Qtrees evoked by the HCs (\ref{ex6:hc-sw}) = $\neg S_{p^+} \rightarrow S_{p}$  and (\ref{ex6:hc-ws}) = $S_p \rightarrow \neg S_{p^+}$. Following Chapters \ref{chap:accommodating-quds} and \ref{chap:redundancy}, we assume that the ``inquisitive'' contribution of \textit{if ... then ...} (glossed $\rightarrow$) is \textit{not} material, meaning, a conditional Qtree is not derived by disjoining the negation of its antecedent Qtrees, with its consequent Qtrees.\\


Chapter \ref{chap:accommodating-quds} instead proposed that conditionals evoke questions pertaining to their consequent, set in the domain(s) of the CS where the antecedent holds. This was modeled by assuming that conditional Qtrees are derived by ``plugging'' a consequent Qtree $T_C$ into the verifying nodes of antecedent Qtrees $T_A$. More concretely, for each verifying node $N$ of $T_A$, $N$ gets replaced by $N \cap T_C$, where $\cap$ refers to tree-node intersection. This operation is repeated in (\ref{ex2:nodewise-inter}). From an algorithmic perspective, the tree-node intersection between $T$ and $N$ can be achieved by (i) intersecting all nodes of $T$ with $N$; (ii) removing resulting empty nodes; (iii) removing resulting dangling and unary edges.\footnote{This operation is structurally idle if $N$ entails a leaf of $T_C$, because in this case, $N \cap T_C$ reduces to a root $N$, and replacing $N$ by $N$ in $T_A$ is idle. However, it might still affect verifying nodes.} (\ref{ex2:cond-qtree}) then builds on this definition to define conditional Qtrees.

\begin{exe}
	\exr{ex2:nodewise-inter} {\textit{Tree-node intersection.} Let $T=(\mathcal{N}, \mathcal{E}, R)$ be a Qtree. Let $p$ be a proposition. The tree-node intersection between $T$ and $p$, noted $T \cap p$, is defined iff $R \cap p \neq \emptyset$ and, if so, is the Qtree $T'=(\mathcal{N}', \mathcal{E}', R')$ s.t.:
		\begin{itemize}
			\item $\mathcal{N}' = \lbrace N \cap p \ | \ N \in \mathcal{N} \wedge N \cap p \neq \emptyset\rbrace$
			\item $\mathcal{E}' = \lbrace \lbrace N_1\cap p, N_2\cap p\rbrace \ | \ \lbrace N_1, N_2\rbrace \in \mathcal{E} \wedge (N_1\cap p) \neq (N_2\cap p) \wedge N_1\cap p \neq \emptyset \wedge N_2\cap p \neq \emptyset \rbrace$
			\item $R' = R\cap p$
	\end{itemize}}
\end{exe} 

\begin{exe}
	\exr{ex2:cond-qtree} {\textit{Qtrees for conditional LFs.} A Qtree $T$ for $X \rightarrow Y$ is obtained from a Qtree $T_X$ for $X$ and a Qtree $T_Y$ for $Y$ by:
		\begin{itemize}
			\item replacing each node $N$ of $T_X$ that is in $\mathcal{N}^+(T_X)$ with $N \cap T_Y$ (see (\ref{ex2:n-t-inter}));
			\item returning the result only if it is a Qtree.
		\end{itemize}
		In other words, $Qtrees(X \rightarrow Y) = \lbrace T_X \cup \bigcup_{N\in \mathcal{N}^+(T_X)}(N\cap T_Y) | (T_X, T_Y) \in Qtrees(X) \times Qtrees(Y) \wedge T_X \cup \bigcup_{N\in \mathcal{N}^+(T_X)}(N\cap T_Y) \text{verifies (\ref{ex2:qtree-def})}  \rbrace$, and $\mathcal{N}^+(T_X \rightarrow T_Y) = \lbrace N\cap N' | (N, N') \in \mathcal{N}^+(T_X) \times \mathcal{N}^+(T_Y) \wedge N\cap N' \neq \emptyset \rbrace$.}
\end{exe}


Additionally, Chapter \ref{chap:accommodating-quds} assumed that only the consequent of a conditional contributes verifying nodes in the resulting conditional Qtree. In particular, nodes falsifying the antecedent are not considered verifying in the resulting conditional Qtree. The core idea behind this operation is that conditionals introduce a hierarchy between antecedent (backgrounded) and consequent (at-issue): the consequent Qtree gets \textit{restricted} by the antecedent Qtree.\\


This definition comes with one useful prediction when it comes to HCs, namely that intersecting a city-level node with a country-level Qtree does not have any effect. This is consistent with the intuition that answering a question about cities automatically answers question about countries, and corresponds to the generalization in (\ref{ex2:vacuous-tree-node}), repeated below.

\begin{exe}
	\exr{ex2:vacuous-tree-node} {\textit{Vacuous tree-node intersection.} Let $T$ be a Qtree whose leaves are $\mathcal{L}(T)$, and $N$ a (non-empty) node (set of worlds). $T\cap N = N$ iff $\exists N' \mathcal{L}(T). \ N \vDash N'$.}
\end{exe}

Figure \ref{fig6:city-country-qtree-intersection} illustrates this result, considering two possible Qtrees for $S_p$ = \textit{SuB29 will take place in Italy}, and their intersection with a city-level node like \textit{Noto}.


\begin{figure}[H]
	\centering
	\begin{subfigure}[b]{\linewidth}
	\centering
	\vspace{-8mm}
	\begin{tabular}{ccccc}
		\scalebox{1}{\begin{forest}
				[{CS$\cap$\textcolor{orange}{Noto}\\=\textcolor{orange}{Noto}} [{\textcolor{blue}{Italy}$\cap$\textcolor{orange}{Noto}\\=\textcolor{orange}{Noto}}] [{$\neg$\textcolor{blue}{Italy}$\cap$\textcolor{orange}{Noto}\\=$\emptyset$}]]
		\end{forest}} &\begin{tabular}{c}
			~\\~\\~\\\textit{empty node}\\\textit{deletion}\\ $\overrightarrow{~~~~~~}$
		\end{tabular}&
		\begin{forest}
			[\textcolor{orange}{Noto} [\textcolor{orange}{Noto}]]
		\end{forest}&\begin{tabular}{c}
			~\\~\\~\\\textit{trivial link}\\\textit{deletion}\\ $\overrightarrow{~~~~~~}$
		\end{tabular}&
		\begin{tabular}{c}
			\\ \\
			\textcolor{orange}{Noto}\\
		\end{tabular}
	\end{tabular}
		\caption{Derivation of \textit{Noto}$\cap$Tree \ref{fig6:qtree-italy-polar}=the \textit{Noto}-node}
\end{subfigure}\vspace{-5mm}

\begin{subfigure}[b]{\linewidth}
	\centering
	\begin{tabular}{ccccc}
		\scalebox{1}{\begin{forest}
				[{CS$\cap$\textcolor{orange}{Noto}\\=\textcolor{orange}{Noto}} [{\textcolor{blue}{Italy}$\cap$\textcolor{orange}{Noto}\\=\textcolor{orange}{Noto}}] [{\textcolor{blue}{France}$\cap$\textcolor{orange}{Noto}\\=$\emptyset$}] [{\textcolor{blue}{UK}$\cap$\textcolor{orange}{Noto}\\=$\emptyset$}]]
		\end{forest}} &\begin{tabular}{c}
			~\\~\\~\\\textit{empty node}\\\textit{deletion}\\ $\overrightarrow{~~~~~~}$
		\end{tabular}&
		\begin{forest}
			[\textcolor{orange}{Noto} [\textcolor{orange}{Noto}]]
		\end{forest}&\begin{tabular}{c}
			~\\~\\~\\\textit{trivial link}\\\textit{deletion}\\ $\overrightarrow{~~~~~~}$
		\end{tabular}&
			\begin{tabular}{c}
				\\ \\
				\textcolor{orange}{Noto}\\
			\end{tabular}
	\end{tabular}
	\caption{Derivation of \textit{Noto}$\cap$Tree \ref{fig6:qtree-italy-wh}=the \textit{Noto}-node}
\end{subfigure}

\caption{Intersecting a city-level node and a country-level tree yields the input city-level node.}
\label{fig6:city-country-qtree-intersection}
\end{figure}

A consequence of this result, is that if an antecedent Qtree $T_X$ and a consequent Qtree $T_Y$ are such that each verifying node of $T_X$ entails some leaf in $T_Y$, the conditional Qtree resulting from their composition, will have the same structure as $T_X$, and its verifying nodes will be exactly the verifying nodes in $T_X$ that entail some verifying node in $T_Y$. We will use this observation to justify the derivation of conditional Qtrees for HCs in the next two Sections.

\subsection{Qtrees for the HCs in (\ref{ex6:hc})}

Let us start by computing the candidate Qtrees for the infelicitous HD (\ref{ex6:hc-sw}) = $\neg S_{p^+} \rightarrow S_p$. Applying (\ref{ex2:cond-qtree}) to this LF, using the Qtrees for $\neg S_{p^+}$ from Figure \ref{fig6:qtrees-not-noto} as antecedent Qtrees, and the Qtrees for $S_p$ from Figure \ref{fig6:qtrees-italy} as consequent Qtrees, leads to the conditional Qtrees in Figure \ref{fig6:qtrees-hc-sw}. 

\begin{figure}[H]
\centering
\begin{subfigure}[b]{.45\linewidth}
	\centering
	\scalebox{1}
	{\begin{forest}
			[CS [\textcolor{orange}{Noto}] [{$\neg$\textcolor{orange}{Noto}} [\fbox{\textcolor{blue}{Italy}$\cap\neg$\textcolor{orange}{Noto}}][$\neg$\textcolor{blue}{Italy}]]]
	\end{forest}}
	\caption{Tree \ref{fig6:qtree-not-noto-polar} $\rightarrow$ Tree \ref{fig6:qtree-italy-polar}.}\label{fig6:tree-hc-sw-polar-polar}
\end{subfigure}\hfill
\begin{subfigure}[b]{.45\linewidth}
	\centering
	\scalebox{1}
	{\begin{forest}
			[CS [\textcolor{orange}{Noto}] [{$\neg$\textcolor{orange}{Noto}} [\fbox{\textcolor{blue}{Italy}$\cap\neg$\textcolor{orange}{Noto}}][\textcolor{blue}{France}] [\textcolor{blue}{...}]]]
	\end{forest}}
	\caption{Tree \ref{fig6:qtree-not-noto-polar} $\rightarrow$ Tree \ref{fig6:qtree-italy-wh}.}\label{fig6:tree-hc-sw-polar-wh}
\end{subfigure}

\begin{subfigure}[b]{.45\linewidth}
	\centering
	\scalebox{1}
	{\begin{forest}
			[CS [\textcolor{orange}{Noto}] [\fbox{\textcolor{orange}{Rome}}] [\fbox{\textcolor{orange}{...}}] [\textcolor{orange}{Paris}] [\textcolor{orange}{...}]]
	\end{forest}}
	\caption{Tree \ref{fig6:qtree-not-noto-wh} $\rightarrow$ Tree \ref{fig6:qtree-italy-polar}/\ref{fig6:qtree-italy-wh}.}\label{fig6:tree-hc-sw-wh}
\end{subfigure}\hfill
\begin{subfigure}[b]{.45\linewidth}
	\centering
	\scalebox{1}
	{\begin{forest}
			[CS [\textcolor{blue}{Italy}[\textcolor{orange}{Noto}] [\fbox{\textcolor{orange}{Rome}}] [\fbox{\textcolor{orange}{...}}]]  [\textcolor{blue}{France}[\textcolor{orange}{Paris}][\textcolor{orange}{...}]] [\textcolor{blue}{...}]]
	\end{forest}}
	\caption{Tree \ref{fig6:qtree-not-noto-tiered} $\rightarrow$ Tree \ref{fig6:qtree-italy-polar}/\ref{fig6:qtree-italy-wh}.}\label{fig6:tree-hc-sw-wh-wh}
\end{subfigure}
\caption{Qtrees for (\ref{ex6:hc-sw})=\#\textit{If SuB29 will not take place in Noto, it will take place in Italy.}}
\label{fig6:qtrees-hc-sw}
\end{figure}

Figures \ref{fig6:tree-hc-sw-polar-polar} and \ref{fig6:tree-hc-sw-polar-wh} are obtained by replacing the verifying \textit{not Noto} node of the ``polar'' antecedent Qtree from Figure \ref{fig6:qtree-not-noto-polar}, with the intersection between this node and a Qtree for \textit{Italy} (either from Figure \ref{fig6:qtree-italy-polar}, or from Figure \ref{fig6:qtree-italy-wh}). Because \textit{not Noto}, does not entail any leaf in the consequent Qtrees for \textit{Italy} (it does not entail any particular city), the whole operation is \textit{not} structurally vacuous, and the output Qtrees are of depth $2$. Verifying nodes are inherited from the consequent Qtree after intersection, i.e. correspond to \textit{Italy but not Noto}.

By contrast, the Qtrees in Figures \ref{fig6:tree-hc-sw-wh} and Figure \ref{fig6:tree-hc-sw-wh-wh} are obtained by replacing each leaf different from \textit{Noto} in the non-``polar'' antecedent Qtrees (from Figures \ref{fig6:qtree-not-noto-wh} and \ref{fig6:qtree-not-noto-tiered} respectively), with the intersection between this leaf, and a Qtree for \textit{Italy} (from Figure \ref{fig6:qtree-italy-polar} or \ref{fig6:qtree-italy-wh}). Because each node different from \textit{Noto}, is a city-node, it will entail some leaf in the Qtrees evoked by \textit{Italy}. Therefore, the formation of a conditional Qtree based on these inputs, will be structurally vacuous. This explains why the Qtrees in Figures \ref{fig6:tree-hc-sw-wh} and Figure \ref{fig6:tree-hc-sw-wh-wh} appear structurally similar to the antecedent Qtrees used to form them, in Figure \ref{fig6:qtree-not-noto-wh} and Figure \ref{fig6:qtree-not-noto-tiered} respectively. The only difference between these inputs, and the outputs, lies in the verifying nodes, which are inherited from the consequent Qtrees, and correspond to Italian cities different from Noto. Figure \ref{fig6:qtree-if-not-noto-italy-wh-breakdown} further details the derivation of the Qtree in Figure \ref{fig6:tree-hc-sw-wh}.

\begin{figure}[H]
	\centering
	\begin{forest}
		[CS [\textcolor{orange}{Noto}][\sout{\fbox{\textcolor{orange}{Rome}}}][\fbox{\textcolor{orange}{...}}][\sout{\fbox{\textcolor{orange}{Paris}}}][...]]
		\draw[<-, dashed] (-.3, -1.3) to[out=north east, in=south east] (1.5,-.5) to[out=north west, in=west] (3, 0);
		\draw[<-, dashed] (1.6, -1.7) to[out=south east, in=north east] (1,-2.2) to[out=south west, in=west] (2, -3);
	\end{forest}
	\dbox{
		\begin{forest}
			[{CS$\cap$\textcolor{orange}{Rome} = \textcolor{orange}{Rome}} [\fbox{\textcolor{blue}{Italy}$\cap$\textcolor{orange}{Rome}=\textcolor{orange}{Rome}}][{$\neg$\textcolor{blue}{Italy}$\cap$\textcolor{orange}{Rome}=$\emptyset$}]]
		\end{forest} = \fbox{\textcolor{orange}{Rome}}\hspace*{3mm}}\\
	\hspace*{10mm}\dbox{
		\begin{forest}
			[{CS$\cap$\textcolor{orange}{Paris} = \textcolor{orange}{Paris}} [\fbox{\textcolor{blue}{Italy}$\cap$\textcolor{orange}{Paris}=$\emptyset$}][{$\neg$\textcolor{blue}{Italy}$\cap$\textcolor{orange}{Paris}=\textcolor{orange}{Paris}}]]
		\end{forest} = {\textcolor{orange}{Paris}}\hspace*{3mm}}	
	\caption{Breakdown of the derivation of Figure \ref{fig6:tree-hc-sw-wh}, assuming Figure \ref{fig6:qtree-italy-polar} is the consequent Qtree. The end result is unchanged if Figure \ref{fig6:qtree-italy-wh} is considered instead.}\label{fig6:qtree-if-not-noto-italy-wh-breakdown}
\end{figure}


Let us now turn to the candidate Qtrees for the felicitous HD (\ref{ex6:hc-ws}) = $S_p \rightarrow \neg S_{p^+}$. Applying (\ref{ex2:cond-qtree}) to this LF, using now the Qtrees for $S_p$ from Figure \ref{fig6:qtrees-italy} as antecedent Qtrees, and  the Qtrees for $\neg S_{p^+}$ from Figure \ref{fig6:qtrees-not-noto} as consequent Qtrees, leads to the conditional Qtrees in Figure \ref{fig6:qtrees-hc-ws}. 
.
\begin{figure}[H]
\centering
\begin{subfigure}[b]{.45\linewidth}
	\centering
	\scalebox{1}
	{\begin{forest}
			[CS [\textcolor{blue}{Italy} [\textcolor{orange}{Noto}][\fbox{$\neg$\textcolor{orange}{Noto}$\cap$\textcolor{blue}{Italy}}]] [{$\neg$\textcolor{blue}{Italy}} ]]
	\end{forest}}
	\caption{Tree \ref{fig6:qtree-italy-polar} $\rightarrow$ Tree \ref{fig6:qtree-not-noto-polar}.}\label{fig6:tree-hc-ws-polar-polar}
\end{subfigure}\hfill
\begin{subfigure}[b]{.45\linewidth}
	\centering
	\scalebox{1}
	{\begin{forest}
			[CS [\textcolor{blue}{Italy} [\textcolor{orange}{Noto}] [\fbox{\textcolor{orange}{Rome}}] [\fbox{\textcolor{orange}{...}}]] [{$\neg$\textcolor{blue}{Italy}} ]]
	\end{forest}}
	\caption{Tree \ref{fig6:qtree-italy-polar} $\rightarrow$ Tree \ref{fig6:qtree-not-noto-wh}/\ref{fig6:qtree-not-noto-tiered}.}\label{fig6:tree-hc-ws-polar-wh}
\end{subfigure}

\begin{subfigure}[b]{.45\linewidth}
	\centering
	\scalebox{1}
	{\begin{forest}
			[CS [\textcolor{blue}{Italy}[\textcolor{orange}{Noto}][\fbox{$\neg$\textcolor{orange}{Noto}$\cap$\textcolor{blue}{Italy}}]] [\textcolor{blue}{France}] [\textcolor{blue}{...}]]
	\end{forest}}
	\caption{Tree \ref{fig6:qtree-italy-wh} $\rightarrow$ Tree \ref{fig6:qtree-not-noto-polar}.}\label{fig6:tree-hc-ws-wh-polar}
\end{subfigure}\hfill
\begin{subfigure}[b]{.45\linewidth}
	\centering
	\scalebox{1}
	{\begin{forest}
			[CS [\textcolor{blue}{Italy}[\textcolor{orange}{Noto}] [\fbox{\textcolor{orange}{Rome}}] [\fbox{\textcolor{orange}{...}}]] [\textcolor{blue}{France}] [\textcolor{blue}{...}] ]
	\end{forest}}
	\caption{Tree \ref{fig6:qtree-italy-wh} $\rightarrow$ Tree \ref{fig6:qtree-not-noto-wh}/\ref{fig6:qtree-not-noto-tiered}.}\label{fig6:tree-hc-ws-wh-wh}
\end{subfigure}
\caption{Qtrees for (\ref{ex6:hc-ws})=\textit{If SuB29 will take place in Italy, it will not take place in Noto.}}
\label{fig6:qtrees-hc-ws}
\end{figure}


All the Qtrees in Figure \ref{fig6:qtrees-hc-ws} are obtained by replacing the verifying \textit{Italy} node of the antecedent Qtree (from Figure \ref{fig6:qtree-not-noto-polar} or \ref{fig6:qtree-not-noto-wh}), with the intersection between this node and a Qtree for \textit{\textit{not Noto}}. Because \textit{Italy}, does not entail any leaf in the consequent Qtrees for \textit{not Noto} (it entails neither \textit{not Noto}, nor any specific city), the whole operation is never structurally vacuous, and the output Qtrees are all of depth $2$. Verifying nodes are inherited from the consequent Qtree after intersection, i.e. correspond to \textit{Italy but not Noto}.\\



At this point, it seems that many Qtrees are available, for both the felicitous variant (\ref{ex6:hc-ws}) and the odd variant (\ref{ex6:hc-sw}). What is the difference between these two sets of Qtrees? It appears that \textit{some} Qtrees compatible with (\ref{ex6:hc-ws}), namely those in Figure \ref{fig6:tree-hc-ws-polar-wh} and \ref{fig6:tree-hc-ws-wh-wh},  \textit{still} feature a by-city partition at the leaf level, as evoked by the consequent: their lower layer partitions the CS, taking \textit{Italy} for granted, according to Italian cities. By contrast, none of the Qtrees evoked by (\ref{ex6:hc-sw}) still feature the by-country partition contributed by their consequent. Such Qtrees either feature by-city partitions at the leaf level (Figures \ref{fig6:tree-hc-sw-wh} and \ref{fig6:tree-hc-sw-wh-wh}), or feature ``mixed'' partitions where some country nodes (namely, \textit{Italy}-nodes) are cut-out (Figure \ref{fig6:tree-hc-sw-polar-polar} and \ref{fig6:tree-hc-sw-polar-wh}). In other words, it seems that the consequent of (\ref{ex6:hc-ws}), \textit{not Noto} \textit{can} be taken to be ``relevant'' to the global question evoked by this sentence, while the consequent of (\ref{ex6:hc-sw}), \textit{Italy}, \textit{cannot}. We will formalize this intuition in the next Section, in the form of an incremental \textsc{Relevance} constraint.


\section{Hurford Conditionals and Relevance}
\subsection{Do we need an extra constraint?}
First, let us double-check that the previous constraints on Qtrees (and LFs) defined in Chapters \ref{chap:accommodating-quds} and \ref{chap:hurford-disj}, are insufficient to capture the contrast between the two HCs in (\ref{ex6:hc}). The first constraint to check is the \textsc{Empty Labeling} constraint, which states that well-formed Qtree should flag at least one node as verifying (see (\ref{ex2:vacuous-flagging}))

\begin{exe}
	\exr{ex2:vacuous-flagging} {\textit{Empty labeling of verifying nodes.} If a sentence $S$ evokes a Qtree $T$ but does not flag any node as verifying on $T$, then $T$ is deemed odd given $S$.}
\end{exe}

It is easy to see that none of the Qtrees in Figure \ref{fig6:qtrees-hc-sw} (corresponding to the infelicitous HC (\ref{ex6:hc-sw})) or Figure \ref{fig6:qtrees-hc-ws} (corresponding to the felicitous HC (\ref{ex6:hc-ws})), violate \textsc{Empty Labeling}: all these Qtrees flag at least one node.\\


The second constraint to check, is \textsc{Q-Non-Redundancy}, which states that a Qtree evoked by an LF should not be equivalent to a Qtree evoked by the simplification of that LF (see (\ref{ex5:q-non-redundancy}) and (\ref{ex5:q-equivalence})).
\begin{exe}
	\exr{ex5:q-non-redundancy} {{\textsc{Q-Non-Redundancy} (final version)}. Let $X$ be a LF and let $Qtrees(X)$ be the set of Qtrees evoked by $X$. For any $T \in Qtrees(X)$, $T$ is deemed \textsc{Q-Redundant} given $X$, iff there exists a formal simplification of $X$, $X'$, and $T' \in Qtrees(X')$, such that $T\equiv T'$.}
	\exr{ex5:q-equivalence} {Qtree equivalence relation $\equiv$. $T \equiv T'$ iff $T$ and $T'$ have same structure and same set of minimal verifying paths.}
\end{exe}


We can show that none of the Qtrees in Figures \ref{fig6:qtrees-hc-sw} and \ref{fig6:qtrees-hc-ws} violate (\ref{ex5:q-non-redundancy}). To see this, we need to review the Qtrees associated to the simplifications of (\ref{ex6:hc-sw}) and (\ref{ex6:hc-ws}). Let us use $p$ and $p^+$ as shorthands for $S_p$ = \textit{SuB29 will take place in Italy}, and $S_{p^+}$ = \textit{SuB29 will take place in Noto}. The possible simplifications of (\ref{ex6:hc-sw}) and (\ref{ex6:hc-ws}) are summarized in Table \ref{tab6:hc-simplifications}.

\begin{table}[H]
	\centering
	\begin{tabular}{ll}
		\toprule
		Sentence & Simplifications                                             \\ \midrule
		(\ref{ex6:hc-sw})=$\neg\pplus\rightarrow\p$     & $\neg\pplus$, \pplus, \p, $\pplus\rightarrow\p$\\
		(\ref{ex6:hc-ws})=$\p\rightarrow\neg\pplus$     & $\neg\pplus$, \pplus, \p, $\p\rightarrow\pplus$\\\bottomrule            
	\end{tabular}
	\caption[]{Gathering the formal simplifications of (\ref{ex6:hc-sw}) and (\ref{ex6:hc-ws})}\label{tab6:hc-simplifications}
\end{table}


Let us start by evaluating the simplifications of (\ref{ex6:hc-sw}). The Qtrees for (\ref{ex6:hc-sw}) are given in Figure \ref{fig6:qtrees-hc-sw}. For (\ref{ex6:hc-sw}) to be deemed deviant, all these Qtrees must be equivalent to \textit{some} Qtree evoked by \textit{some} simplification of (\ref{ex6:hc-sw}). For clarity, we proceed simplification-by-simplification.

First, Qtrees for the simplification $p$, shown in Figure \ref{fig6:qtrees-italy}, have a different structure altogether from all the Qtrees in Figure \ref{fig6:qtrees-hc-sw}. So there is no way Qtree equivalence holds.

Second, among the Qtrees for the simplification $p^+$ shown in Figure \ref{fig6:qtrees-noto}, two Qtrees are structurally identical to two Qtrees from Figure \ref{fig6:qtrees-hc-sw}. The first pair is made of the Qtree in Figure \ref{fig6:qtree-noto-wh} and the one in Figure \ref{fig6:tree-hc-sw-wh}. These two Qtrees, though structurally identical, are not equivalent, because they each flag different sets of leaves as verifying. The second pair is made of the Qtree in Figure \ref{fig6:qtree-noto-tiered} and the one in Figure \ref{fig6:tree-hc-sw-wh-wh}. Again, these two Qtrees, though structurally identical, are not equivalent, because they each flag different sets of leaves as verifying.

Third, the reasoning about the simplification $p^+$, extends to the simplification $\neg p^+$: there are two Qtrees evoked by $\neg p^+$ (in Figures \ref{fig6:qtree-not-noto-wh} and \ref{fig6:qtree-not-noto-tiered}) that are structurally identical to two Qtrees in Figure \ref{fig6:qtrees-hc-sw}, but these pairs, though structurally identical, are not equivalent, because they flag different sets of leaves as verifying.

Lastly, Qtrees for the simplification $p^+\rightarrow p$ are the same as the Qtrees evoked by $p^+$. This is because, the only verifying node of the antecedent Qtree, \textit{Noto}, always entails a leaf of the consequent Qtree (namely, \textit{Italy}); therefore, the intersection operation is vacuous (as per (\ref{ex2:vacuous-tree-node})), and the resulting conditional Qtree are just the same as the antecedent Qtree used to form them. Since we have already shown that none of the Qtree evoked by $p^+$ make the Qtrees in Figure \ref{fig6:qtrees-hc-sw} \textsc{Q-Redundant}, none of the Qtree evoked by $p^+\rightarrow p$ make the Qtrees in Figure \ref{fig6:qtrees-hc-sw} \textsc{Q-Redundant}, either.

We have just gone through all the possible simplifications of the infelicitous HC (\ref{ex6:hc-sw}), and shown that none of these simplification evoke Qtrees triggering \textsc{Q-Non-Redundancy}. Therefore, \textsc{Q-Non-Redundancy} does not rule out (\ref{ex6:hc-sw}).\\

Let us now turn to the simplifications of (\ref{ex6:hc-ws}). The Qtrees for (\ref{ex6:hc-ws}) are given in Figure \ref{fig6:qtrees-hc-ws}. For (\ref{ex6:hc-ws}) to be deemed deviant, all these Qtrees must be equivalent to \textit{some} Qtree evoked by \textit{some} simplification of (\ref{ex6:hc-ws}).

First, Qtrees for the simplification $p$, shown in Figure \ref{fig6:qtrees-italy}, have a different structure altogether from all the Qtrees in Figure \ref{fig6:qtrees-hc-ws}. So there is no way Qtree equivalence holds.

Second, Qtrees for the simplification $p^+$, shown in Figure \ref{fig6:qtrees-noto}, also have a different structure altogether from all the Qtrees in Figure \ref{fig6:qtrees-hc-ws}. So there is no way Qtree equivalence holds. This extends to the simplification $\neg p^+$.


Lastly, Qtrees for the simplification $p\rightarrow p^+$ are structurally identical as the Qtree in Figure \ref{fig6:qtrees-hc-ws}. This is because both sets of Qtrees are built using input Qtrees (antecedent, consequent) that are themselves pairwise structurally identical. Still, Qtrees for the simplification $p\rightarrow p^+$ are \textit{not} pairwise equivalent to the Qtrees in Figure \ref{fig6:qtrees-hc-ws}, because each pair of structurally identical Qtrees, ends up flagging disjoint sets of verifying nodes. The Qtrees in Figure \ref{fig6:qtrees-hc-ws}, flag nodes that are Italian cities different from Noto; while the Qtrees evoked by $p\rightarrow p^+$, flag nodes that are Italian cities, and in fact Noto. Thus, there is no way Qtree equivalence holds.

We have just gone through all the possible simplifications of the felicitous HC (\ref{ex6:hc-ws}), and shown that none of these simplification evoke Qtrees triggering \textsc{Q-Non-Redundancy}. Therefore, \textsc{Q-Non-Redundancy} does not incorrectly rule out (\ref{ex6:hc-ws}).\\




In the Section, we have seen that the constraints on Qtrees and LFs posited so far (\textsc{Empty Labeling}, \textsc{Q-Non-Redundancy}), if they do not incorrectly rule out felicitous HCs like (\ref{ex6:hc-ws}), also cannot rule out \textit{in}felicitous HCs like (\ref{ex6:hc-sw}). In other words both HCs are so far predicted to be felicitous. To account for the infelicity of (\ref{ex6:hc-sw}), while retaining the felicity of (\ref{ex6:hc-ws}), we will appeal to an updated definition of \textsc{Relevance}. The next Section will first motivate the use of a new \textsc{Relevance} constraint, by showing how previous approaches to \textsc{Relevance} are not directly sensitive to the granularity considerations that appear crucial in the context of HCs.

\subsection{The shortcomings of earlier notions of Relevance}

We have previously suggested that the contrast between felicitous and infelicitous HCs may be a matter of relevance. Chapter \ref{chap:introduction} already defined ways in which a proposition could be understood as relevant to a question, seen as a partition of the CS. Adapting insights from \citet{Lewis1988} to the QuD framework, we stated that a proposition is \textsc{Lewis-Relevant} to a QuD, if it coincides with a union of cells. This is repeated in (\ref{ex1:lewis-relevance}). 

\begin{exe}
	\exr{ex1:lewis-relevance} {\textsc{Lewis's Relevance (rephrased in the QuD framework).} Let $\mathcal{C}$ be a conversation, $Q$ a QuD defined as a partition of $CS(\mathcal{C})$. Let $p$ be a proposition. $p$ is Lewis-relevant to $Q$, iff $\exists C \subseteq Q. \ p \cap CS(\mathcal{C}) = C$}
\end{exe}

We also mentioned the view from \citet{Roberts2012}, according to which a proposition is relevant if it rules-out a cell (see (\ref{ex1:roberts-relevance})). 

\begin{exe}
	\exr{ex1:roberts-relevance} {\textsc{Roberts's Relevance \citep{Roberts2012}.} Let $\mathcal{C}$ be a conversation, $Q$ a (non-trivial) QuD defined as a partition of $CS(\mathcal{C})$. Let $p$ be a proposition. $p$ is Roberts-relevant to $Q$, if $\exists c \in Q. \ p \cap c = \emptyset$.
	}
\end{exe}


%\begin{figure}[H]
%	\centering
%	\begin{subfigure}[t]{.27\linewidth}
%		\centering
%		\scalebox{.8}{
%			\begin{tikzpicture}
%				\draw [draw=black] (3,1) rectangle (0,0);
%				\draw [draw=black,fill=gray] (2,1) rectangle (0,0);
%				\draw [] (1,1) -- (1,0);
%				\draw [] (2,1) -- (2,0);
%		\end{tikzpicture}}
%		\caption{A relevant proposition.}\label{fig6:relevant-p}
%	\end{subfigure}
%	\hfill
%	\begin{subfigure}[t]{.65\linewidth}
%		\centering\scalebox{.8}{
%			\begin{tikzpicture}
%				\draw [draw=black] (3,1) rectangle (0,0);
%				\draw [draw=black,fill=gray] (.8,.6) rectangle (.2,.4);
%				\draw [] (1,1) -- (1,0);
%				\draw [] (2,1) -- (2,0);
%		\end{tikzpicture}}
%		\hfill\scalebox{.8}{
%			\begin{tikzpicture}
%				\draw [draw=black] (3,1) rectangle (0,0);
%				\draw [draw=black,fill=gray] (1.5,.6) rectangle (.2,.4);
%				\draw [] (1,1) -- (1,0);
%				\draw [] (2,1) -- (2,0);
%		\end{tikzpicture}}\hfill\scalebox{.8}{
%			\begin{tikzpicture}
%				\draw [draw=black] (3,1) rectangle (0,0);
%				\draw [draw=black,fill=gray] (2,1) rectangle (1,0);
%				\draw [draw=black,fill=gray] (1,.6) rectangle (.2,.4);
%				\draw [] (1,1) -- (1,0);
%				\draw [] (2,1) -- (2,0);
%		\end{tikzpicture}}
%		\caption{Irrelevant propositions.}\label{fig6:over-p}
%	\end{subfigure}
%	\caption{Various QuD-proposition configurations (proposition defined by the gray area).}\label{fig6:qud-p-diagrams}
%\end{figure}


Ideally, we would like to reuse either (\ref{ex1:lewis-relevance}) or (\ref{ex1:roberts-relevance}) in the context of compositionally derived Qtrees, and derive, for instance, that two LFs $X$ and $Y$ can form a conditional $X \rightarrow Y$ only if the proposition denoted by $Y$, is relevant to a Qtree evoked by $X$.\footnote{This would be the most intuitive direction, because $X$, as antecedent, would be understood as ``setting'' the QuD, and $Y$, would be understood as some relevant answer to it.} This (stipulative) idea is summarized in (\ref{ex6:incremental-relevance-naive}).

\begin{exe}
	\ex {\textit{Incremental Relevance (naive version).} Let $X$ and $Y$ be two LFs. Then $X \rightarrow Y$ is deviant if none of the questions $X$ evokes (partitions of the CS; e.g. leaves of $X$'s Qtrees), make the proposition $Y$ denotes relevant. Relevance may be understood as (\ref{ex1:lewis-relevance}) or (\ref{ex1:roberts-relevance}).}\label{ex6:incremental-relevance-naive}
\end{exe}

This however, would not quite work on the HCs at stake. In particular, (\ref{ex6:incremental-relevance-naive}) predicts an infelicitous HC like (\ref{ex6:hc-sw}) to be fine. Indeed, (\ref{ex6:hc-sw}) has its antecedent evoke Qtrees whose leaves either partition the CS into cities, or partition the CS into \textit{Noto} vs. \textit{not Noto}-worlds. These partitions are represented in Figures \ref{tab:not-noto-italy-wh-rel} and \ref{tab:not-noto-italy-polar-rel}. Additionally, (\ref{ex6:hc-sw})'s consequent denotes \textit{Italy}. Is \textit{Italy} relevant to any of the partitions evoked by (\ref{ex6:hc-sw})'s antecedent? Figure \ref{tab:not-noto-italy-wh-rel} shows that \textit{Italy} is in fact both \textsc{Lewis-} and \textsc{Roberts-Relevant} to the antecedent's implicit ``\textit{wh}'' question: it corresponds to a collection of Italian cities (hence \textsc{Lewis-Relevant}), and rules out the non Italian cities (hence \textsc{Roberts-Relevant}). According to (\ref{ex6:incremental-relevance-naive}), this is enough to predict that (\ref{ex6:hc-sw}) should be fine. Note however that, if a polar partition is considered itself for the antecedent, the consequent is neither \textsc{Lewis-} not \textsc{Roberts-Relevant}.

\begin{figure}[H]
	\centering
	\begin{subfigure}[t]{.47\linewidth}
		\centering
		\begin{tabular}{|lll|ll}
			\hline
			\multicolumn{1}{|l|}{Noto} & \multicolumn{1}{l|}{Rome} & ... & \multicolumn{1}{l|}{Paris} & \multicolumn{1}{l|}{...} \\ \hline
			\multicolumn{3}{|l|}{\cellcolor{blue!20!white}Italy}                             & \multicolumn{2}{l}{}                                  \\ \cline{1-3}
		\end{tabular}
		\caption{(\ref{ex6:hc-sw})'s consequent is both \textsc{Lewis-} and \textsc{Roberts-Relevant} to the ``\textit{wh}''-partition evoked by (\ref{ex6:hc-sw})'s antecedent.}\label{tab:not-noto-italy-wh-rel}
	\end{subfigure}
	\hfill
	\begin{subfigure}[t]{.47\linewidth}
		\centering
		\begin{tabular}{|lllll}
			\hline
			\multicolumn{1}{|l|}{Noto} & \multicolumn{4}{l|}{not Noto} \\ \hline
			\multicolumn{3}{|l|}{\cellcolor{blue!20!white}~~~Italy~~~}                &       &       \\ \cline{1-3}
		\end{tabular}
		\caption{(\ref{ex6:hc-sw})'s consequent is neither \textsc{Lewis-} nor \textsc{Roberts-Relevant} to the ``polar''-partition evoked by (\ref{ex6:hc-sw})'s antecedent.}\label{tab:not-noto-italy-polar-rel}
	\end{subfigure}
	\caption{How (\ref{ex6:hc-sw})'s consequent interacts with (\ref{ex6:hc-sw})'s antecedent's evoked questions.}
\end{figure}

Additionally, we can show that (\ref{ex6:incremental-relevance-naive}) predicts the felicitous HC in (\ref{ex6:hc-ws}), to be deviant.  Indeed, (\ref{ex6:hc-ws}) has its antecedent evoke Qtrees whose leaves either partition the CS into countries, or partition the CS into \textit{Italy} vs. \textit{not Italy}-worlds. These partitions are represented in Figures \ref{tab:italy-not-noto-wh-rel} and \ref{tab:italy-not-noto-polar-rel}. Additionally, (\ref{ex6:hc-ws})'s consequent denotes \textit{not Noto}. Is \textit{not Noto} relevant to any of the partitions evoked by (\ref{ex6:hc-sw})'s antecedent? Figure \ref{tab:italy-not-noto-wh-rel} shows that \textit{not Noto} is neither \textsc{Lewis-} nor \textsc{Roberts-Relevant} to the antecedent's implicit ``\textit{wh}'' question: it does not correspond to a collection of countries (hence not \textsc{Lewis-Relevant}), and does not rule out any country (hence not \textsc{Roberts-Relevant}). The same holds for Figure \ref{tab:italy-not-noto-wh-rel}: \textit{not Noto} does not correspond to a union of cells (hence not \textsc{Lewis-Relevant}), and does not rule out any cell (hence not \textsc{Roberts-Relevant}). 

\begin{figure}[H]
	\centering
	\begin{subfigure}[t]{.47\linewidth}
		\centering
			\begin{tabular}{llll|}
				\hline
				\multicolumn{2}{|l|}{Italy} & \multicolumn{1}{l|}{France} & ... \\ \hline
				\multicolumn{1}{l|}{}   & \multicolumn{3}{l|}{\cellcolor{orange!20!white}not Noto}         \\ \cline{2-4} 
			\end{tabular}
		\caption{(\ref{ex6:hc-ws})'s consequent is neither \textsc{Lewis-} nor \textsc{Roberts-Relevant} to the ``\textit{wh}''-partition evoked by (\ref{ex6:hc-ws})'s antecedent.}\label{tab:italy-not-noto-wh-rel}
	\end{subfigure}
	\hfill
	\begin{subfigure}[t]{.47\linewidth}
		\centering
		\begin{tabular}{llll|}
			\hline
			\multicolumn{2}{|l|}{Italy} & \multicolumn{2}{l|}{not Italy} \\ \hline
			\multicolumn{1}{l|}{}   & \multicolumn{3}{l|}{\cellcolor{orange!20!white}not Noto}      \\ \cline{2-4} 
		\end{tabular}
		\caption{(\ref{ex6:hc-ws})'s consequent is neither \textsc{Lewis-} nor \textsc{Roberts-Relevant} to the ``polar''-partition evoked by (\ref{ex6:hc-ws})'s antecedent.}\label{tab:italy-not-noto-polar-rel}
	\end{subfigure}
	\caption{How (\ref{ex6:hc-ws})'s consequent interacts with (\ref{ex6:hc-ws})'s antecedent's evoked questions.}
\end{figure}

Of course, reversing the directionality of the principle in (\ref{ex6:incremental-relevance-naive}), i.e. stating that the antecedent should be relevant to one of the consequent's implicit questions (see (\ref{ex6:incremental-relevance-reversed})), would in turn reverse the above predictions, and capture the case HCs. 


\begin{exe}
	\ex {\textit{Incremental Relevance (reversed version).} Let $X$ and $Y$ be two LFs. Then $X \rightarrow Y$ is deviant if none of the questions $Y$ evokes (partitions of the CS; e.g. leaves of $Y$'s Qtrees), make the proposition $X$ denotes relevant. Relevance may be understood as (\ref{ex1:lewis-relevance}) or (\ref{ex1:roberts-relevance}).}\label{ex6:incremental-relevance-reversed}
\end{exe}

We think there are still two issues with 


  However, we have argued the contrast in HCs should be explained by a constraint sensitive to the level of granularity conveyed by the antecedent and consequent. But neither \textsc{Lewis's Relevance} nor \textsc{Roberts's Relevance} are sensitive to how the proposition at stake packages information: the only relevant(!) factor is the set-theoretic relation between the proposition as a whole, and the QuD, seen as a partition of the CS. This implies that the level of granularity conveyed by the proposition is not directly taken into account when assessing its relevance to the question (although, of course, the strength of the proposition is). Looking beyond the case of HCs, this does not seem to match intuitions about what a relevant answer to a question is.\\

The question-answer pairs in (\ref{ex6:relevant-answers}) for instance, show that the answer's conveyed granularity should be taken into account when assessing \textsc{Relevance}. (\ref{ex6:relevant-answers})'s QuD strongly suggests a by-country partition of the CS. This predicts both (\ref{ex6:relevant}) and (\ref{ex6:relevant-coarse}) to be relevant under both views, because both sentences refer to a proper subset of all countries. Yet, (\ref{ex6:relevant-coarse}) does not appear felicitous without hedging (e.g., using \textit{for all I know}). Its relative oddness in this context is however intuitive: \textit{Europe} feels ``coarser-grained'' than, e.g. \textit{France or Belgium}. This can be modeled by saying \textit{Europe} can less straightforwardly be partitioned into country-level cells, than \textit{France or Belgium}. 

\begin{exe}
	\ex {In which country did Jo grow up?}
	\begin{xlist}
		\ex[] {-- They grew up in France or Belgium.\hfill (\ref{ex1:lewis-relevance}) \cmark{} (\ref{ex6:roberts-relevance}) \cmark }\label{ex6:relevant}
		\ex[??] {-- They grew up in Europe. \hfill (\ref{ex1:lewis-relevance}) \cmark{} (\ref{ex6:roberts-relevance}) \cmark}\label{ex6:relevant-coarse}
		\ex[?] {-- They grew up in Paris (or Brussels). \hfill (\ref{ex1:lewis-relevance}) \xmark{} (\ref{ex6:roberts-relevance}) \cmark}\label{ex6:overinformative}
		\ex[??] {-- They speak French natively \hfill (\ref{ex1:lewis-relevance}) \xmark{} (\ref{ex6:roberts-relevance}) \xmark}\label{ex6:irrelevant}
	\end{xlist}
	\label{ex6:relevant-answers}
\end{exe}

In fact, this is exactly the kind of intuition that Qtrees incorporate:\footnote{See also \cite{Benbaji2024} for a ``dynamic'' view of \textsc{Relevance} along the same lines.} a Qtree for (\ref{ex6:relevant}) ould feature a country-level terminal layer, properly coinciding with the QuD; while a Qtree for (\ref{ex6:relevant-coarse}) would ``stop'' at the continent-level. But no continent properly ``fits'' within a country-cell of the QuD. Making \textsc{Relevance} sensitive to distinctions in evoked Qtrees could thus explain the contrast in (\ref{ex6:relevant}-\ref{ex6:relevant-coarse}).


 Having a granularity-sensitive notion of Relevance may also help explain why finer-grained, overinformative answers like (\ref{ex6:overinformative}), are not so infelicitous:\footnote{We however note that (\ref{ex1:roberts-relevance}) achieves this, too.} they suggest a partition of the CS whose cells (city-level) can all be mapped to a single (country-level) cell of the partition provided by the QuD. (\ref{ex6:irrelevant}), which is also overinformative and predicted to be irrelevant, sounds more odd than (\ref{ex6:overinformative}) and (\ref{ex6:overinformative-2}), because the partition it suggests (\textit{What's Jo's level in French?}) cannot be properly mapped to the partition set by the QuD. For instance, Jo could very well be fluent in French, without having grown up in France. This suggests that granularity-sensitive notion of \textsc{Relevance} should state that a proposition is relevant if it can be partitioned into more specific ``sub-propositions'' (=verifying nodes), s.t. each sub-proposition ``fits'' a cell of the question, i.e., entails it.


This may also capture the contrast between (\ref{ex6:overinformative}) and (\ref{ex6:irrelevant})-- (\ref{ex6:overinformative}) turns out not so odd, because it typically evokes a Qtree featuring a terminal city-layer, in which each node can be ``fitted'' within single country (i.e. is relevant to the QuD according to (\ref{ex6:roberts-relevance}-\ref{ex6:relevance-p})). (\ref{ex1:lewis-relevance}) on the other hand, feels worse because the kind of question it evokes (\textit{What's Jo's proficiency in French?}), features cells that cannot be properly mapped to the partition set by the QuD. These data overall suggest that a proposition is relevant if it evokes a Qtree whose nodes all entail some cell of the overt QuD. This constitutes an extension (and a strengthening) of (\ref{ex6:roberts-relevance}-\ref{ex6:relevance-q}), 

\subsection{Incremental Q-Relevance}

In the previous Section, we have argued that \textsc{Relevance} should be constraint between questions, whether overt or evoked by a proposition. In fact, both \citet{Lewis1988} and \citet{Roberts2012} discussed notions of \textsc{Relevance} between questions. According to \citet{Lewis1988}, relevance between to propositions, can be cashed out in terms of relevance between their subject matters. If we see subject matters as evoked questions (as suggested by Lewis) the notion of relevance at stake amounts to inclusion (meaning, refinement) or connection between the questions evoked by the two propositions. This is shown by Lewis to be equivalent to connection, and at least one proposition being non-contingent. Two questions are said to be connected if one can find a pair of cells, one from each question, whose overlap is empty. Lewis additionally shows an interesting result, which is that, under a strict analysis of conditionals, \textit{if $p$ then $q$} holds iff $p$ and $q$ are relevant to each other. This however, still fails to predict an asymmetry in HCs, because relevance, defined as connection between evoked questions, and non-contingency, is a symmetric concept, that is also very weak. \\

Considering a stronger notion of relevance could help, but would come at the cost of additional stipulations. For instance, stipulating that whatever the antecedent of a conditional denotes, should be \textsc{Lewis-Relevant} to some question evoked by the consequent, may very well account for the data. But this stipulation raises many questions: why should the antecedent act as the proposition and the consequent, as the background question, and not vice-versa? Why would this only apply to conditionals, and not disjunctions? (see paris or germany / germany or paris)
france or not paris
nor paris or france

We add to this another principle, also from \citet{Roberts2012}, according to which a \textit{follow-up question} is relevant to a QuD, if each of the alternatives the follow-up denotes is relevant to the QuD (in the sense of (\ref{ex6:roberts-relevance-p})). This is rephrased in (\ref{ex6:roberts-relevance-q}).

\begin{exe}
	\ex\label{ex6:roberts-relevance} {\textsc{Roberts's Relevance \citep{Roberts2012}.} Let $\mathcal{C}$ be a conversation, $Q$ a (non-trivial) QuD defined as a partition of $CS(\mathcal{C})$. 
		\begin{enumerate}[(i)]
			\item\label{ex6:roberts-relevance-p} Let $p$ be a proposition. $p$ is Roberts-relevant to $Q$, if $\exists c \in Q. \ p \cap c = \emptyset$.
			\item\label{ex6:roberts-relevance-q} Let $Q'$ be a follow-up question, seen as a set of alternatives. $Q'$ is Roberts-relevant to $Q$, if each alternative of $Q'$ is Roberts-relevant to $Q$.
		\end{enumerate}
	}
\end{exe}






We now introduce \textsc{Q-Relevance}, to incorporate a sightly weaker variant of this intuition. Our principle applies incrementally at each step of the Qtree computation process. \textsc{Q-Relevance} targets the verifying nodes of an input Qtree (which play the role of a ``structured'' proposition), and evaluates how these nodes ``fit'' into the output Qtree. A verifying node ``fits'' a Qtree iff it is not cut-out in this Qtree, i.e. iff there is no node of the Qtree that overlaps with it without fully containing it. In other words, no node of the output Qtree should \textit{strictly entail} a verifying node of the input Qtree. This is formalized in (\ref{ex6:q-relevance}).

(\ref{ex6:q-relevance}) introduces \textsc{Q-Relevance}, which targets Qtree derivation instead of QuD-answer pairs. It applies incrementally at each step of the Qtree derivation process. The interesting case is the binary one: if two Qtrees combine together, the first Qtree can be seen as an ``incremental'' QuD, while the second Qtree (``input'') can be seen as a follow-up question subject to \textsc{Relevance}. The result of this incremental combination creates an ``output'' Qtree, against which \textsc{Relevance} is assessed. Specifically, \textsc{Q-Relevance} targets the verifying nodes of the input Qtree, and evaluates how they ``fit'' into the output Qtree. A verifying node $N$ ``fits'' a Qtree $T$ iff it is not cut-out in $T$, i.e. no node in $T$ overlaps with $N$ without fully containing it. A correlate of this, is that no node of the output Qtree should \textit{strictly entail} a verifying node of the input Qtree. (\ref{ex6:q-relevance}) therefore involves a sightly weaker version of the previous intuition: only some nodes of the input Qtree (the verifying ones) will need to ``fit'' in the QuD/ouput Qtree.

\begin{exe}
	\ex {\textit{\textsc{Q-Relevance}}.  Let $X$ and $Y$ be LFs and let $Qtrees(X)$ and $Qtrees(Y)$ be the sets of Qtrees compatible with $X$ and $Y$. Let $\circ$ be a Qtree-level operation, e.g. $\neg$, $\vee$, or $\rightarrow$. Let $C$ be a non-empty partial LF (incremental context). Two cases:
		\begin{itemize}
			\item $C=\circ$, with $\circ$ a unary operation. For any $T \in Qtrees(X)$, $\circ T$ is \textsc{Q-Relevant} with respect to $\circ X$ iff $\forall N \in \mathbb{N}^+(T). \ \neg\exists N' \in \mathbb{N}(\circ T). \ N' \subset N$.
			\item $C = X \circ$, with $\circ$ a binary operation. For any $T \in Qtrees(Y)$, $T_X \circ T_Y$ is \textsc{Q-Relevant} with respect to $X \circ Y$ iff $\forall N \in \mathbb{N}^+(T_Y). \ \neg\exists N' \in \mathbb{N}(T_x \circ T_Y). \ N' \subset N$
	\end{itemize}}\label{ex6:q-relevance}
\end{exe}

\begin{exe}
	\ex {{\textsc{Q-Relevance}}.  Let $X$ and $Y$ be LFs and let $Qtrees(X)$ and $Qtrees(Y)$ be the sets of Qtrees compatible with $X$ and $Y$. Let $\circ$ be a Qtree-level operation, e.g. $\neg$, $\vee$, or $\rightarrow$. Let $C$ be a non-empty partial LF (incremental context). Two cases:
		\begin{itemize}
			\item $C=\circ$, with $\circ$ a unary operation. For any $T \in Qtrees(X)$, $\circ T$ is \textsc{Q-Relevant} with respect to $\circ X$ iff $\forall N \in \mathbb{N}^+(T). \ \neg\exists N' \in \mathbb{N}(\circ T). \ N' \subset N$.
			\item $C = X \circ$, with $\circ$ a binary operation. For any $T \in Qtrees(Y)$, $T_X \circ T_Y$ is \textsc{Q-Relevant} with respect to $X \circ Y$ iff $\forall N \in \mathbb{N}^+(T_Y). \ \neg\exists N' \in \mathbb{N}(T_x \circ T_Y). \ N' \subset N$.
	\end{itemize}}\label{ex6:q-relevance}
\end{exe}


\textsc{Q-Relevance} issues arise in conditional Qtrees like those evoked by the HCs in (\ref{ex6:hc}). This is because such Qtrees are formed \textit{via} operations of the form $N \cap T$, with $N$ a verifying node of an antecedent Qtree, and $T$ a consequent Qtree. This may affect the verifying nodes of $T$ in ways that violate \textsc{Q-Relevance}. Specifically, (\ref{ex6:q-relevance}) predicts that, in a conditional Qtree, each verifying node of the consequent Qtree compatible with \textit{some} verifying node of the antecedent Qtree, should be fully preserved in the output Qtree.\footnote{To see this, consider a Qtree $T_X\rightarrow T_Y$ compatible with an LF $X \rightarrow Y$. Within $T_X\rightarrow T_Y$, we focus on a subtree $N\cap T_Y$, with $N \in \mathbb{N}^+(T_X)$. (\ref{ex6:q-relevance}) imposes that for each verifying node in $T_Y$, no node of $N\cap T_Y$ strictly entails it: $\forall N' \in \mathbb{N}^+(T_Y). \neg \exists N'' \in \mathbb{N}(N\cap T_Y). \ N'' \subset N'$. If $N'$ in this formula is incompatible with $N$, then for sure no node $N''$ will strictly entail it. If $N'$ is compatible with $N$, then a violation of (\ref{ex6:q-relevance}) arises as soon as $N \cap N' \subset N'$, i.e. if intersecting $N'$ with $N$ ``shrinks'' $N'$. This holds for any subtree $N\cap T_Y$ of $T_X\rightarrow T_Y$, with $N \in \mathbb{N}^+(T_X)$.} We now detail how this applies to HCs.





The predictions of \textsc{Q-Relevance} for our $\lbrace \neg, \vee, \rightarrow\rbrace$ fragment of the language are the following. First, because $\neg$ is structure preserving, we can be sure that all the input verifying nodes are part of the negated output Qtree, satisfying \textsc{Q-Relevance}. Second, because $\vee$ forces well-formed unions of Qtrees (in terms of both structure and verifying nodes), we can be sure that all the input verifying nodes are part of the disjunctive output Qtree, again, satisfying \textsc{Q-Relevance}. The interesting cases arise with $\rightarrow$, because this operation involves intersecting verifying nodes and Qtrees, i.e. performing operations of the form $N \cap T$, with $N$ a verifying node of an antecedent Qtree, and $T$ a consequent Qtree. This kind of operation may affect the nodes of $T$ (in particular, $T$'s verifying nodes) in ways that may violate \textsc{Q-Relevance}. To see this, let us consider a Qtree $T_X\rightarrow T_Y$ compatible with an LF $X \rightarrow Y$. Within $T_X\rightarrow T_Y$, let us focus on a subtree $N\cap T_Y$, with $N \in \mathbb{N}^+(T_X)$. \textsc{Q-Relevance} imposes that for each verifying node in $T_Y$, no node of $N\cap T_Y$ strictly entails it: $\forall N' \in \mathbb{N}^+(T_Y). \neg \exists N'' \in \mathbb{N}(N\cap T_Y). \ N'' \subset N'$. If $N'$ in this formula is incompatible with $N$, then for sure no node $N''$ will strictly entail it. If $N'$ is compatible with $N$, then a violation of \textsc{Q-Relevance} arises as soon as $N \cap N' \subset N'$, i.e. if intersecting $N'$ with $N$ ``shrinks'' $N'$. This holds for any subtree $N\cap T_Y$ of $T_X\rightarrow T_Y$, with $N \in \mathbb{N}^+(T_X)$. Zooming out, this implies that each verifying node of the consequent Qtree compatible with some verifying node of the antecedent Qtree, should be fully preserved in the output.\\


\subsection{Capturing the contrast in Hurford Conditionals}

We are now equipped to explain the oddness pattern of the HCs in (\ref{ex6:hc}), whose Qtrees are repeated below. \textsc{Q-Relevance} predicts all the Qtrees compatible with the infelicitous HC (\ref{ex6:hc-sw}) (see Figure \ref{fig6:qtrees-hc-sw}) to be \textsc{Irrelevant}. This is because in each case, the output Qtree contains nodes that strictly entail the verifying \textit{Italy} node of the consequent Qtree: nodes of the form \textit{Italy and not Noto} (in Trees \ref{fig6:tree-hc-sw-polar-polar-2} and \ref{fig6:tree-hc-sw-polar-wh-2}), or city-level nodes (in Trees \ref{fig6:tree-hc-sw-wh-2} and \ref{fig6:qtree-if-not-noto-italy-tiered-wh-2}). (\ref{ex6:hc-sw}) is thus predicted to be odd, in a way that is consistent with the intuition that raising \textit{Italy} after assuming \textit{not Noto} feels irrelevant to the general question the sentence is trying to answer.

\begin{figure}[H]
\centering
\setlength{\fboxsep}{1pt}
\renewcommand\thefigure{\ref{fig6:qtrees-hc-sw}}
\begin{subfigure}[b]{.22\linewidth}
	\centering
	\scalebox{.7}
	{\begin{forest}for tree={s sep=2mm, inner sep=0, l=0}
			[CS [{Noto}] [{$\neg$Noto} [\fbox{\begin{tabular}{c}
					Italy$\cap\neg$Noto\\
					\textbf{\textcolor{red}{$\subset$ Italy}}
			\end{tabular}}][$\neg$Italy]]]
	\end{forest}}
	\caption{Tree \ref{fig6:qtree-not-noto-polar} $\rightarrow$ Tree \ref{fig6:qtree-italy-polar}.}\label{fig6:tree-hc-sw-polar-polar-2}
\end{subfigure}\hfill
\begin{subfigure}[b]{.22\linewidth}
	\centering
	\scalebox{.7}
	{\begin{forest}for tree={s sep=2mm, inner sep=0, l=0}
			[CS [{Noto}] [{$\neg$Noto} [\fbox{\begin{tabular}{c}
					Italy$\cap\neg$Noto\\
					\textbf{\textcolor{red}{$\subset$ Italy}}
			\end{tabular}}][France] [...]]]
	\end{forest}}
	\caption{Tree \ref{fig6:qtree-not-noto-polar} $\rightarrow$ Tree \ref{fig6:qtree-italy-wh}.}\label{fig6:tree-hc-sw-polar-wh-2}
\end{subfigure}\hfill
\begin{subfigure}[b]{.3\linewidth}
	\centering
	\scalebox{.7}
	{\begin{forest}for tree={s sep=2mm, inner sep=0, l=0}
			[CS [{Noto}] [\fbox{\begin{tabular}{c}
					Rome\\
					\textbf{\textcolor{red}{$\subset$ Italy}}
			\end{tabular}}] [\fbox{\begin{tabular}{c}
					...\\
					\textbf{\textcolor{red}{$\subset$ Italy}}
			\end{tabular}}] [{Paris}] [...]]
	\end{forest}}
	\caption{Tree \ref{fig6:qtree-not-noto-wh} $\rightarrow$ Tree \ref{fig6:qtree-italy-wh}/\ref{fig6:qtree-italy-polar}.}\label{fig6:qtree-if-not-noto-italy-wh-wh-2}\label{fig6:tree-hc-sw-wh-2}
\end{subfigure}\hfill
\begin{subfigure}[b]{.25\linewidth}
	\centering
	\scalebox{.7}
	{\begin{forest}for tree={s sep=2mm, inner sep=0, l=0}
			[CS [Italy[{Noto}] [\fbox{\begin{tabular}{c}
					Rome\\
					\textbf{\textcolor{red}{$\subset$ Italy}}
			\end{tabular}}] [\fbox{\begin{tabular}{c}
					...\\
					\textbf{\textcolor{red}{$\subset$ Italy}}
			\end{tabular}}]]  [...]]
	\end{forest}}
	\caption{Tree \ref{fig6:qtree-not-noto-tiered} $\rightarrow$ Tree \ref{fig6:qtree-italy-wh}/\ref{fig6:qtree-italy-polar}.}\label{fig6:qtree-if-not-noto-italy-tiered-wh-2}
\end{subfigure}
\caption{Qtrees for (\ref{ex6:hc-sw})=\#\textit{If SuB29 will not take place in Noto, it will take place in Italy.}}
\end{figure}

Regarding the felicitous HC (\ref{ex6:hc-ws}), \textsc{Q-Relevance} predicts some, but crucially not all, of the Qtrees in Figure \ref{fig6:qtrees-hc-ws} to be \textsc{Irrelevant}. Trees \ref{ex6:qtree-if-italy-not-noto-polar-polar-2} and \ref{ex6:qtree-if-italy-not-noto-wh-polar-2} are \textsc{Irrelevant}, because in both cases, the output Qtree contains nodes of the form \textit{Italy and not Noto}, which strictly entail the verifying \textit{not Noto} node of the consequent Qtree. Trees \ref{ex6:qtree-if-italy-not-noto-polar-wh-2} and \ref{ex6:qtree-if-italy-not-noto-wh-wh-2} however, do not violate \textsc{Q-Relevance}, because they are constructed from a consequent Qtree for \textit{not Noto} that introduces a city-level partition that perfectly ``fits'' the restriction to \textit{Italy} introduced by the antecedent Qtree, so that no city-node verifying \textit{not Noto} is strictly entailed by a node in the output Qtree. This correctly predicts (\ref{ex6:hc-ws}) to be felicitous, in a way that is consistent with the intuition that it is possible to make sense of \textit{not Noto} after assuming \textit{Italy}, granted that \textit{not Noto} discusses a \textit{which city?} kind of question.



\begin{figure}[H]
\setlength{\fboxsep}{1pt}
\renewcommand\thefigure{\ref{fig6:qtrees-hc-ws}}
\centering
\begin{subfigure}[b]{.23\linewidth}
	\centering
	\scalebox{.8}
	{\begin{forest}for tree={s sep=2mm, inner sep=0, l=0}
			[CS [{Italy} [Noto][\fbox{\begin{tabular}{c}
					$\neg$Noto$\cap$Italy\\
					\textbf{\textcolor{red}{$\subset \neg$Noto}}
			\end{tabular}}]] [{$\neg$Italy} ]]
	\end{forest}}
	\caption{Tree \ref{fig6:qtree-italy-polar} $\rightarrow$ Tree \ref{fig6:qtree-not-noto-polar}.}\label{ex6:qtree-if-italy-not-noto-polar-polar-2}
\end{subfigure}\hfill
\begin{subfigure}[b]{.25\linewidth}
	\centering
	\scalebox{.8}
	{\begin{forest}for tree={s sep=2mm, inner sep=0, l=0}
			[CS [{Italy} [{Noto}] [\fbox{Rome}] [\fbox{...}]] [{$\neg$Italy} ]]
	\end{forest}}
	\caption{Tree \ref{fig6:qtree-italy-polar} $\rightarrow$ Tree \ref{fig6:qtree-not-noto-wh}/\ref{fig6:qtree-not-noto-tiered}.}\label{ex6:qtree-if-italy-not-noto-polar-wh-2}
\end{subfigure}
\hfill
\begin{subfigure}[b]{.25\linewidth}
	\centering
	\scalebox{.8}
	{\begin{forest}for tree={s sep=2mm, inner sep=0, l=0}
			[CS [Italy[Noto][\fbox{\begin{tabular}{c}
					$\neg$Noto$\cap$Italy\\
					\textbf{\textcolor{red}{$\subset \neg$Noto}}
			\end{tabular}}]] [France] [...]]
	\end{forest}}
	\caption{Tree \ref{fig6:qtree-italy-wh} $\rightarrow$ Tree \ref{fig6:qtree-not-noto-polar}.}\label{ex6:qtree-if-italy-not-noto-wh-polar-2}
\end{subfigure}\hfill
\begin{subfigure}[b]{.25\linewidth}
	\centering
	\scalebox{.8}
	{\begin{forest}for tree={s sep=2mm, inner sep=0, l=0}
			[CS [Italy[{Noto}] [\fbox{Rome}] [\fbox{...}]] [France] [...] ]
	\end{forest}}
	\caption{Tree \ref{fig6:qtree-italy-wh} $\rightarrow$ Tree \ref{fig6:qtree-not-noto-wh}/\ref{fig6:qtree-not-noto-tiered}.}\label{ex6:qtree-if-italy-not-noto-wh-wh-2}
\end{subfigure}
\caption{Qtrees for (\ref{ex6:hc-ws})=\textit{If SuB29 will take place in Italy, it will not take place in Noto.}}\label{ex6:qtree-if-italy-not-noto-2}
\end{figure}

\section{Extension to ``varieties'' of Hurford Conditionals}


\subsection{``Long-Distance'' HCs}
In this section, we explore how our QuD-informed view of redundancy captures two kinds of ``Long-Distance'' variants of HCs: those derived from LDHDs \textit{via} the \textit{or-to-if} tautology; and those derived by applying the ``Long Distance'' recipe directly to HCs. 


\subsubsection{LDHCs derived from LDHDs}

Recall that ``Long Distance'' HDs (\textbf{LDHDs}, \citenp{Marty2022}) are obtained from standard HDs of the form \p{} $\vee$ \pplus{} by further disjoining \pplus{} with a proposition \r{}, s.t. \pplus{} $\vee$ \r{} is not redundant, and no longer entails \p. This can be done by choosing \r{} to contradict \pplus{} and \p{}. Such constructions have been argued to be deviant, despite the fact that none of their disjuncts are in an entailment relation (and thus satisfy the traditional, non-explanatory version of Hurford's Constraint). (\ref{ex5:ldhd-pos}) illustrates a classic LDHD: \textit{Europe} entails \textit{France}, and \textit{France} is further disjoined with \textit{New York}, that is incompatible with \textit{Europe} -- making the two main disjuncts compatible, but non-entailing.  (\ref{ex5:ldhd-neg}) illustrates a perhaps more degenerate case of LDHD: the weak disjunct, \textit{not France}, takes the form of the negation of the stronger disjunct of (\ref{ex5:ldhd-pos}); the stronger disjunct, \textit{not Europe}, takes the form of the negation of the weak disjunct of (\ref{ex5:ldhd-pos}); and the extra proposition disjoined with it (\textit{Paris}), is chosen to be incompatible with \textit{not France} and \textit{not Europe}. Note that the kind of variable change we described here, was already used to construct simple infelicitous HCs like (\ref{ex5:hc-ns-w}).

\begin{exe}
	\exr{ex5:ldhd}
	\begin{xlist}
		\ex[\#] {Jo either studied in \textcolor{blue}{Europe}, or in \textcolor{orange}{France} or in \textcolor{green}{New York}.
			\hfill \p{} $\vee$ (\pplus{} $\vee$ \r)}\label{ex5:ldhd-pos}
		\ex[\#] {Jo either does\textcolor{blue}{n't study in France}, or \textcolor{orange}{not in Europe} or in \textcolor{green}{Paris}.	\hfill \textbf{\textcolor{blue}{q}} $\vee$ (\textcolor{orange}{\textbf{q}$^+$} $\vee$ \r)\\
			with: \textbf{\textcolor{blue}{q}} := $\neg$\pplus; \textcolor{orange}{\textbf{q}$^+$} := $\neg$\p }\label{ex5:ldhd-neg}
	\end{xlist}
\end{exe}

We think that both (\ref{ex5:ldhd-pos}) and (\ref{ex5:ldhd-neg}) are degraded -- which is in itself interesting because SR predicts (\ref{ex5:ldhd-neg}) to be fine.

\begin{exe}
	\ex 
	\begin{xlist}
		\ex {(\ref{ex5:ldhd-pos})=\p{} $\vee$ (\pplus{} $\vee$ \r) is SR.\\
			C = \pplus. $\forall D. $ \p{} $\vee$ ((\pplus{}$\wedge$ $D$) $\vee$ \r) $\equiv$ \p{} $\vee$ ((\pplus{}$\vee$ \r) $\wedge$ ($D$ $\vee$ \r)) $\equiv$ (\p{} $\vee$ \r) $\wedge$ (\p{} $\vee$ $D$ $\vee$ \r) $\equiv$ \p{} $\vee$ \r}
		\ex {(\ref{ex5:ldhd-neg}) = $\neg$\pplus{} $\vee$ ($\neg$\p{} $\vee$ \r) is not SR.\\
			C = $\neg$\pplus. Take $D = \top$.\\
			$\neg$(\pplus{} $\wedge$ $D$) $\vee$ ($\neg$\p{} $\vee$ \r) $\equiv$ $\neg$\pplus{} $\vee$ ($\neg$\p{} $\vee$ \r) $\equiv$ $\neg$\pplus{} $\vee$ \r{} $\not\equiv$ $\neg$\p{} $\vee$ \r\\
			C = $\neg$\p. Take $D = \bot$.\\
			$\neg$\pplus{} $\vee$ ($\neg$(\p{} $\wedge$ $D$) $\vee$ \r) $\equiv$ $\neg$\pplus{} $\vee$ ($\top$ $\vee$ \r) $\equiv$ $\top$ $\not\equiv$ $\neg$\pplus{} $\vee$ \r\\
			C = \r. Take $D = \top$.\\
			$\neg$\pplus{} $\vee$ ($\neg$\p{} $\vee$ (\r{} $\wedge$ $D$)) $\equiv$ $\neg$\pplus{} $\vee$ \r{} $\not\equiv$ $\neg$\pplus $\equiv$ $\neg$\pplus{} $\vee$ $\neg$\p{}\\
			C = ($\neg$\p{} $\vee$ \r). Take $D = \top$.\\
			$\neg$\pplus{} $\vee$ (($\neg$\p{} $\wedge$ $D$) $\vee$ (\r{} $\wedge$ $D$)) $\equiv$ $\neg$\pplus{} $\vee$ ($\neg$\p{} $\vee$ \r) $\equiv$ $\neg$\pplus{} $\vee$ \r{} $\not\equiv$ $\neg$\pplus
			
		}
	\end{xlist}
	
\end{exe}

Interestingly, turning the LDHDs in (\ref{ex5:ldhd}) into conditionals maintains infelicity, as shown in (\ref{ex5:ldhc}).

\begin{exe}
	\exr{ex5:ldhc}
	\begin{xlist}
		\ex[\#] {If Jo did not study in \textcolor{blue}{Europe}, she studied in \textcolor{orange}{France} or in \textcolor{green}{New York}.}
		\ex[\#] {If Jo studied in \textcolor{blue}{France}, she did not study \textcolor{orange}{in Europe} or she studied in \textcolor{green}{Paris}.}
	\end{xlist}
\end{exe}

We proceed to show that both (\ref{ex5:ldhc-pos}) and (\ref{ex5:ldhc-neg}) are \textsc{Q-Redundant}. \textsc{Q-Redundancy}, as defined in Chapter \ref{chap:hurford-sentences}, states that a Qtree $T$ paired with a specific LF $X$ is deviant iff the $T$ is equivalent to a tree $T'$ compatible with a simplification of $X$. (\ref{ex5:ldhc-pos}) is then \textsc{Q-Redundant} because any Qtree it evokes is also evoked by (\ref{ex5:ldhc-pos-simplification}); and (\ref{ex5:ldhc-neg}) is \textsc{Q-Redundant} as well, because any Qtree it evokes is also evoked by (\ref{ex5:ldhc-neg-simplification}).

\begin{exe}
	\ex\label{ex5:ldhc-simplifications}
	\begin{xlist}
		\ex[] {If Jo did not study in \textcolor{blue}{Europe}, she studied in \textcolor{green}{New York}.}\label{ex5:ldhc-pos-simplification}
		\ex[] {If Jo studied in \textcolor{blue}{France}, she studied in \textcolor{green}{Paris}.}\label{ex5:ldhc-neg-simplification}
	\end{xlist}
\end{exe}

We start with (\ref{ex5:ldhc-pos}). The possible Qtrees evoked by its antecedent ($\neg$\textit{Europe}) are depth-1 Qtrees whose leaves are $\lbrace$\textit{Europe}, $\neg$\textit{Europe}$\rbrace$, or continent-level alternatives $\lbrace$\textit{Europe}, \textit{Africa}, \textit{Asia}, ...$\rbrace$. ``Tiered'' Qtrees are unlikely, since coarser-grained alternatives to continent-level locations are themselves unlikely to be salient. In each possible Qtree, the leaves incompatible with \textit{Europe} are flagged as verifying (effect of negation). This is summarized in Figure \ref{trees:not-Europe}.

\begin{figure}[H]
	\centering
	\begin{subfigure}[b]{.3\linewidth}
		\centering
		\scalebox{1}{
			\begin{forest}
				[CS[Europe][\fbox{$\neg$Europe}]]
			\end{forest}
		}
		\caption{}\label{tree:not-europe-polar}
	\end{subfigure}
	\qquad
	\begin{subfigure}[b]{.3\linewidth}
		\centering
		\scalebox{1}{
			\begin{forest}
				[CS[Europe][\fbox{America}][\fbox{Africa}][\fbox{...}]]
			\end{forest}
		}
		\caption{}\label{tree:not-europe-wh}
	\end{subfigure}
	\caption{Qtrees for \textit{Jo did not study in \textcolor{blue}{\textbf{Europe}}.}}\label{trees:not-Europe}
\end{figure}

The consequent of (\ref{ex5:ldhc-pos}) is disjunctive, and therefore results from the well-formed union of Qtrees evoked by its two disjuncts (\textit{France} and \textit{New York}). Possible Qtrees for \textit{France} (already computed in section \ref{sec6:qtrees-scalemates-non-scalemates}), are repeated in Figure \ref{trees:france-2}; possible Qtrees for \textit{New York} can be computed in a way that is strictly analogous to Qtrees for \textit{Paris} (as done in section \ref{sec6:qtrees-scalemates-non-scalemates}), and are thus given in Figure \ref{trees:new-york}.
\begin{figure}[H]
	\centering
	\begin{subfigure}[b]{.3\linewidth}
		\centering
		\begin{forest}
			[CS[\bfbox{France}][$\neg$France]]
		\end{forest}
		\caption{}\label{tree:france-polar-2}
	\end{subfigure}
	\qquad
	\begin{subfigure}[b]{.3\linewidth}
		\centering
		\begin{forest}
			[CS[\bfbox{France}][US][...]]
		\end{forest}
		\caption{}\label{tree:france-wh-2}
	\end{subfigure}
	\caption{Qtrees for \textit{Jo studied in \textbf{\textcolor{blue}{France}}}.} \label{trees:france-2}
\end{figure}

\begin{figure}[H]
	\centering
	\begin{subfigure}[b]{.3\linewidth}
		\centering
		\scalebox{1}{
			\begin{forest}
				[CS[\gfbox{New York}][$\neg$New York]]
		\end{forest}}
		\caption{}\label{tree:new-york-polar}
	\end{subfigure}
	\hfill
	\begin{subfigure}[b]{.3\linewidth}
		\centering
		\scalebox{1}{
			\begin{forest}
				[CS[\gfbox{New York}][Boston][Paris][...]]
		\end{forest}}
		\caption{}\label{tree:new-york-wh}
	\end{subfigure}\hfill
	\begin{subfigure}[b]{.3\linewidth}
		\centering
		\scalebox{1}{
			\begin{forest}
				[CS[France[{Paris}][...]][US[\gfbox{New York}][Boston]][...]]
		\end{forest}}
		\caption{}\label{tree:new-york-tiered}
	\end{subfigure}
	
	
	\caption{Qtrees for \textit{Jo studied in \textbf{\textcolor{green}{New York}}.}}\label{trees:new-york}
\end{figure}

Unioning Qtrees from Figures \ref{trees:france-2} and \ref{trees:new-york} can be done in only one way: using Tree \ref{tree:france-wh-2} for \textit{France}, and Tree \ref{tree:new-york-tiered}, for \textit{New York}. This yields the Qtree in Figure \ref{tree:france-or-new-york}.

\begin{figure}[H]
	\centering
	\scalebox{1}{
		\begin{forest}
			[CS[\fbox{France}[{Paris}][...]][US[\fbox{New York}][Boston][...]][...]]
	\end{forest}}
	\caption{Qtree for \textit{Jo studied in \textcolor{orange}{\textbf{France}} or in \textcolor{green}{\textbf{New York}}}.}\label{tree:france-or-new-york}
\end{figure}

Now that Qtrees for the antecedent and the consequent of (\ref{ex5:ldhc-pos}) are derived, we can combine them to form a Qtree for the whole conditional, by taking the antecedent Qtrees from Figure \ref{trees:not-Europe} and replacing their verifying nodes (anything non Europe) by their intersection with the consequent Qtree from Figure \ref{tree:france-or-new-york}. This yields the Qtrees in Figure \ref{trees:if-not-Europe-then-france-or-new-york}. 

\begin{figure}[H]
	\centering
	\begin{subfigure}[b]{.45\linewidth}
		\centering
		\scalebox{1}{
			\begin{forest}
				[CS[Europe][{$\neg$Europe} [US[\fbox{New York}][Boston][...]][Japan[...]][...]]]
			\end{forest}
		}
		\caption{}
	\end{subfigure}
	\hfill
	\begin{subfigure}[b]{.45\linewidth}
		\centering
		\scalebox{1}{
			\begin{forest}
				[CS[Europe][{America}[US[\fbox{New York}][Boston][...]][...]][{Africa}[...]][{...}]]
			\end{forest}
		}
		\caption{}
	\end{subfigure}
	\caption{Qtrees for (\ref{ex5:ldhc-pos}) = \textit{If Jo did not study in \textcolor{blue}{\textbf{Europe}}, she studied in \textcolor{orange}{\textbf{France}} or in \textcolor{green}{\textbf{New York}}.}}\label{trees:if-not-Europe-then-france-or-new-york}
\end{figure}

Crucially, the intersection operation between the consequent Qtree in Figure \ref{tree:france-or-new-york}, and the non-Europe nodes in the Qtrees from Figure \ref{trees:not-Europe}, discarded the sections of the consequent Qtree that had to do with \textit{France}. This does not violate \textsc{Q-Relevance}, but instead, causes the output Qtrees to be identical (and hence, equivalent) to Qtrees evoked by a formal simplification of (\ref{ex5:ldhc-pos}) that does not feature \textit{France} as a disjunct in its consequent -- namely, (\ref{ex5:ldhc-pos-simplification}). Any Qtree derived from (\ref{ex5:ldhc-pos}) is thus \textsc{Q-Redundant}, and as a result, we correctly predict (\ref{ex5:ldhc-pos}) to be infelicitous.\\

We can now move on to the case of (\ref{ex5:ldhc-neg}). The possible Qtrees evoked by its antecedent (France), are already given in Figure \ref{trees:france-2}.


The consequent of (\ref{ex5:ldhc-neg}) is disjunctive, and therefore results from the well-formed union of Qtrees evoked by its two disjuncts ($\neg$\textit{Europe} and \textit{Paris}). Possible Qtrees for $\neg$\textit{Europe} are already given in Figure \ref{trees:not-Europe}. Possible Qtrees for \textit{Paris} are given in Figure \ref{trees:paris}.

\begin{figure}[H]
	\centering
	\begin{subfigure}[b]{.3\linewidth}
		\centering
		\scalebox{1}{
			\begin{forest}
				[CS[\ofbox{Paris}][$\neg$Paris]]
		\end{forest}}
		\caption{}\label{tree:paris-polar}
	\end{subfigure}
	\hfill
	\begin{subfigure}[b]{.3\linewidth}
		\centering
		\scalebox{1}{
			\begin{forest}
				[CS[\ofbox{Paris}][Nice][London][...]]
		\end{forest}}
		\caption{}\label{tree:paris-wh}
	\end{subfigure}\hfill
	\begin{subfigure}[b]{.3\linewidth}
		\centering
		\scalebox{1}{
			\begin{forest}
				[CS[France[\ofbox{Paris}][Nice][...]][UK[London][...]][...]]
		\end{forest}}
		\caption{}\label{tree:paris-tiered}
	\end{subfigure}
	\begin{subfigure}[b]{.3\linewidth}
		\centering
		\scalebox{1}{
			\begin{forest}
				[CS[Europe[France[\fbox{Paris}][Nice][...]][UK[London][...]]][America[...]][...]]
		\end{forest}}
		\caption{}\label{tree:paris-tiered-tiered}
	\end{subfigure}
	
	
	\caption{Qtrees for \textit{Jo studied in \textbf{\textcolor{orange}{Paris}}.}}\label{trees:paris}
\end{figure}

Unioning Qtrees from Figures \ref{trees:not-Europe} and \ref{trees:paris} can be done in only one way: using Tree \ref{tree:not-europe-wh} for $\neg$\textit{Europe}, and Tree \ref{tree:paris-tiered-tiered}, for \textit{Paris}. This yields the Qtree in Figure \ref{tree:not-europe-or-paris}.

\begin{figure}[H]
	\centering
	\scalebox{1}{
		\begin{forest}
			[CS[Europe[France[\fbox{Paris}][Nice][...]][UK[London][...]]][\fbox{America}[{...}]][\fbox{...}]]
	\end{forest}}
	\caption{Qtrees for \textit{Jo did not study in \textbf{\textcolor{blue}{Europe}} or studied in \textbf{\textcolor{orange}{Paris}}.}}\label{tree:not-europe-or-paris}
\end{figure}

Now that Qtrees for the antecedent and the consequent of (\ref{ex5:ldhc-neg}) are derived, we can combine them to form a Qtree for the whole conditional, by taking the antecedent Qtrees from Figure \ref{trees:france-2} and replacing their verifying nodes (\textit{France} nodes) by their intersection with the consequent Qtree from Figure \ref{tree:not-europe-or-paris}. This yields the Qtrees in Figure \ref{trees:if-france-then-not-europe-or-paris}. 

\begin{figure}[H]
	\centering
	\begin{subfigure}[b]{.3\linewidth}
		\centering
		\begin{forest}
			[CS[{France}[\fbox{Paris}][Nice][...]][$\neg$France]]
		\end{forest}
		\caption{}
	\end{subfigure}
	\qquad
	\begin{subfigure}[b]{.3\linewidth}
		\centering
		\begin{forest}
			[CS[{France}[\fbox{Paris}][Nice][...]][US][...]]
		\end{forest}
		\caption{}
	\end{subfigure}
	\caption{Qtrees for (\ref{ex5:ldhc-neg}) = \textit{If Jo studied in \textcolor{blue}{\textbf{France}}, she did not study in \textcolor{orange}{\textbf{Europe}} or she studied in \textcolor{green}{\textbf{Paris}}.}} \label{trees:if-france-then-not-europe-or-paris}
\end{figure}

Crucially, the intersection operation between the consequent Qtree in Figure \ref{tree:not-europe-or-paris}, and the \textit{France} nodes in the Qtrees from Figure \ref{trees:france-2}, discarded the sections of the consequent Qtree that had to do with $\neg$\textit{Europe}. This does not violate \textsc{Q-Relevance}, but instead, causes the output Qtrees to be identical (and hence, equivalent) to Qtrees evoked by a formal simplification of (\ref{ex5:ldhc-neg}) that does not feature $\neg$\textit{Europe} as a disjunct in its consequent -- namely, (\ref{ex5:ldhc-neg-simplification}). Any Qtree derived from (\ref{ex5:ldhc-neg}) is thus \textsc{Q-Redundant}, and as a result, we correctly predict (\ref{ex5:ldhc-neg}) to be infelicitous.\\

In sum, \textsc{Q-Redundancy} predicts both kinds of LDHCs derived from LDHDs via the \textit{or-to-if} tautology to be infelicitous, in line with pre-theoretical intuitions about these sentences.




\subsubsection{LDHCs derived from HCs}

This reasoning extends to the LDHCs (\ref{ex5:ldhc-w-ns}) and (\ref{ex5:ldhc-ns-w}), whose trees are given in Fig. \ref{fig5:ldhc-non-scalar-w-ns} and \ref{fig5:ldhc-non-scalar-ns-w}. For (\ref{ex5:ldhc-w-ns}), \textsc{Q-Relevance} is satisfied because none of the verifying leaves from the consequent Qtree (cities different from Paris and Brussels) are cut across in the output Qtree. The feeling of redundancy in (\ref{ex5:ldhc-w-ns}) may come from the fact removing \textit{Brussels} from the sentence leads to the same overall meaning and Qtree.\footnote{Following \citet{HenotMortier2024a,HenotMortier2024b}, this implies that (\ref{ex5:ldhc-w-ns}) is \textsc{Q-Redundant}.} For (\ref{ex5:ldhc-ns-w}), \textsc{Q-Relevance} is violated because the compatible \textit{France} leaf from the consequent Qtree cannot ``fit'' into the city-level nodes introduced by the antecedent.


\begin{minipage}{.45\linewidth}
	\centering
	\begin{figure}[H]
		\centering
		\scalebox{.7}{
			\begin{forest}
				[CS[FR[Paris][\fbox{Nice}][\fbox{...}]][$\neg$FR/UK]]
		\end{forest}}
		\caption{Qtree for (\ref{ex5:ldhc-w-ns}).}\label{fig5:ldhc-non-scalar-w-ns}
	\end{figure}
\end{minipage}
\hfill
\begin{minipage}{.45\linewidth}
	\centering
	\begin{figure}[H]
		\centering
		\scalebox{.7}{
			\begin{forest}
				[CS[Paris][Brussels][\fbox{\begin{tabular}{c}
						FR$\wedge$Nice\\
						\textcolor{red}{$\subset$ FR}
				\end{tabular}}][...]]
		\end{forest}}
		\caption{Qtree for (\ref{ex5:ldhc-ns-w}).}\label{fig5:ldhc-non-scalar-ns-w}
	\end{figure}
\end{minipage}





An argument against this view comes from Long-Distance HCs (henceforth LDHC). Such structures, inspired from Long-Distance HDs (\cite{Marty2022}) and exemplified in (\ref{ex5:ldhc}), are derived from (\ref{ex5:hc-s}) by further disjoining \pplus{} with a proposition \r{} taken to be incompatible with \p. 


\begin{exe}
	\ex\label{ex5:ldhc-surface}
	\begin{xlist}
		\ex[?] {If Jo studied in \textcolor{blue}{France} she didn't study in \textcolor{orange}{Paris} or \textcolor{green}{Brussels}.\hfill \p$\rightarrow\neg$(\pplus$\vee$\r)}\label{ex5:ldhc-w-ns}
		\ex[\#] {If Jo didn't study in \textcolor{orange}{Paris} or \textcolor{green}{Brussels} she studied in \textcolor{blue}{France}.\hfill $\neg$(\pplus$\vee$\r)$\rightarrow$\p}\label{ex5:ldhc-ns-w}
	\end{xlist}
\end{exe}

Felicity-wise, LDHCs seem quite degraded: (\ref{ex5:ldhc-w-ns}) is felt to convey the same kind of information twice (because having studied in France contextually entails not having studied in Brussels), while (\ref{ex5:ldhc-ns-w})'s consequent feels independent from the subject matter raised by the antecedent. Yet neither (\ref{ex5:ldhc-w-ns}) nor (\ref{ex5:ldhc-ns-w}) are predicted to be SR, due to the presence of \r{}.

\begin{exe}
	\ex (\ref{ex5:ldhc-w-ns}) is not SR
	\begin{xlist}
		\ex {C = \pplus{} $\vee$\r. Take $D = \top$.\\\p{} $\rightarrow$ $\neg$((\pplus{} $\vee$ \r) $\wedge$ $D$) $\equiv$ \p{} $\rightarrow$ $\neg$(\pplus{} $\vee$ \r) $\not\equiv$ \p}
		\ex {C = \p. Take $D = \bot$.\\
			(\p{} $\wedge$ $D$) $\rightarrow$ $\neg$(\pplus{}$\vee$ \r) $\equiv$ (\p{} $\wedge$ $\bot$) $\rightarrow$ $\neg$(\pplus{}$\vee$ \r) $\equiv$ $\bot$ $\rightarrow$ $\neg$(\pplus{}$\vee$ \r) $\equiv$ $\top$ $\not\equiv$ $\neg$(\pplus{}$\vee$\r)} 
		\ex {C = \pplus. Take $D = \top$.\\
			\p{} $\rightarrow$ $\neg$((\pplus{}$\wedge$ $D$)$\vee$ \r) $\equiv$ \p{} $\rightarrow$ $\neg$(\pplus{}$\vee$ \r) $\equiv$ $\neg$\p{} $\vee$ ($\neg$\pplus{}$\wedge$$\neg$\r) $\equiv$ $\neg$\pplus{}$\wedge$(\p{}$\rightarrow$$\neg$\r) $\not\equiv$ \p{}$\rightarrow$$\neg$\r}
		\ex {C = \r. Take $D = \top$.\\
			\p{} $\rightarrow$ $\neg$(\pplus{}$\vee$(\r{} $\wedge$ $D$)) $\equiv$ \p{} $\rightarrow$ $\neg$(\pplus{}$\vee$ \r) $\equiv$ $\neg$\p{} $\vee$ ($\neg$\pplus{}$\wedge$$\neg$\r) $\equiv$ $\neg$\pplus{}$\wedge$(\p{}$\rightarrow$$\neg$\r) $\not\equiv$ \p{}$\rightarrow$$\neg$\pplus}
		\label{ex5:sr-ldhc-ws}
	\end{xlist}
	\ex (\ref{ex5:ldhc-ns-w}) is not SR
	\begin{xlist}
		\ex {C = \pplus{} $\vee$ \r. Take $D = \top$.\\ $\neg$((\pplus{} $\vee$ \r) $\wedge$ $D$) $\rightarrow$ \p{} $\equiv$ $\neg$(\pplus{} $\vee$ \r) $\rightarrow$ \p{} $\equiv$ \r{} $\vee$ \p{} $\not\equiv$ \p}
		\ex {C = \p. Take $D = \bot$.\\
			$\neg$(\pplus{}$\vee$ \r) $\rightarrow$ (\p{} $\wedge$ $D$) $\equiv$  $\neg$(\pplus{}$\vee$ \r)$\rightarrow$(\p{} $\wedge$ $\bot$)  $\equiv$ \pplus{}$\vee$ \r{} $\vee$ $\bot$ $\equiv$ \pplus{}$\vee$ \r{} $\not\equiv$ $\neg$(\pplus{}$\vee$\r)}
		\ex {C = \pplus. Take $D = \bot$.\\
			$\neg$((\pplus{} $\wedge$ $D$)$\vee$ \r) $\rightarrow$ \p{}  $\equiv$  $\neg$(\pplus{}$\vee$ \r)$\rightarrow$ \p{} $\equiv$ \pplus{}$\vee$ \r{} $\vee$ $\bot$ $\equiv$ \pplus{}$\vee$ \r{} $\not\equiv$ $\neg$(\pplus{}$\vee$\r)} \label{ex5:sr-ldhc-sw}
	\end{xlist}
\end{exe}

\begin{exe}
	\ex 
	\begin{xlist}
		\ex[\#] {Jo studied in \textcolor{orange}{Paris} or \textcolor{green}{Brussels} or she studied in \textcolor{blue}{France}. \hfill (\pplus$\vee$\r) $\vee$ \p}\label{ex5:ldhd-s-w}
		\ex[\#] {Jo studied in \textcolor{blue}{France} or she studied in \textcolor{orange}{Paris} or \textcolor{green}{Brussels}.\hfill \p{} $\vee$ (\pplus$\vee$\r)}\label{ex5:ldhd-w-s}
	\end{xlist}
\end{exe}

\section{``Compatible'' Hurford Conditionals}
However, HC-variants of (\ref{ex5:hd-compatible}) are predicted by our model to be odd regardless of the ordering of the antecedent and consequent, because Qtrees evoked by \textit{France} and Qtrees evoked by \textit{the Basque country} are not in any kind of refinement relation. This prediction is not empirically borne out, as shown in (\ref{ex5:hc-compatible}).

\begin{exe}
	\ex \label{ex5:hc-compatible}
	\begin{xlist}
		\ex[?] {If SuB29 will take place in France, it will take place in the Basque country.}\label{ex5:hc-compatible-ws}
		\ex[\#] {If SuB29 will take place in the Basque country, it will take place in France.}\label{ex5:hc-compatible-sw}
	\end{xlist}
\end{exe}

Within our framework, the contrast in (\ref{ex5:hc-compatible}) suggests that it is somehow easier to split the set of \textit{Basque country} worlds into French and Spanish subsets (in order to satisfy \textsc{Q-Relevance} in \ref{ex5:hc-compatible-ws}), than to split the set of \textit{France} worlds into Basque and non-Basque subsets (in order to satisfy \textsc{Q-Relevance} in \ref{ex5:hc-compatible-sw}). Future work should investigate when, and how such coercion operations on Qtrees can take place.\\


\section{Outlook}

Beyond the Hurford Sentences used as a case study for this chapter, our model of implicit QuDs raises a variety of questions: what distinguishes non-scalar from scalar alternatives when it comes to Qtree computation? How do Qtrees interact with presuppositions, and with operators such as \textit{exh}, \textit{only}, \textit{at least}? Can we extend the model to (incrementally) predict the shape of the follow-up questions \textit{raised} by sentences? Chapters 4 and 5 will deal with some of these questions. The next Chapter presents another use case for our model of QuDs and the concept of \textsc{Q-Redudancy}. Crucially, the data presented in the next Chapter (variations of the structure $p \vee p \vee q$ based on the \textit{or-to-if} tautology), represent a challenge for \citeauthor{Kalomoiros2024}'s account, as well as other accounts of pragmatic oddness, \textsc{Redundancy}-based, or not.

An issue remains with ``conditional'' counterparts of (\ref{ex6:chd}), shown in (\ref{ex6:hc-compatible}). (\ref{ex6:hc-compatible-ws}-\ref{ex6:hc-compatible-sw}) are \textit{both} predicted to be odd, roughly because Qtrees evoked by \textit{France} and Qtrees evoked by \textit{the Basque country} convey incomparable degrees of granularity, and thus are not in any kind of refinement relation.

\begin{exe}
	\ex \label{ex6:hc-compatible}
	\begin{xlist}
		\ex[?] {If SuB29 will take place in France, it will take place in the Basque country.}\label{ex6:hc-compatible-ws}
		\ex[\#] {If SuB29 will take place in the Basque country, it will take place in France.}\label{ex6:hc-compatible-sw}
	\end{xlist}
\end{exe}

In our framework, the contrast in (\ref{ex6:hc-compatible}) suggests that it is somehow easier to ``split'' \textit{Basque country} nodes into French and Spanish subsets (to satisfy \textsc{Q-Relevance} in \ref{ex6:hc-compatible-ws}), than to ``split'' \textit{France} nodes into Basque and non-Basque subsets (to satisfy \textsc{Q-Relevance} in \ref{ex6:hc-compatible-sw}). Future work will determine when and how such coercion operations on Qtrees can take place.


