% From mitthesis package
% Version: 1.01, 2023/06/19
% Documentation: https://ctan.org/pkg/mitthesis
%
% The abstract environment creates all the required headers and footnote. 
% You only need to add the text of the abstract itself.
%
% Approximately 500 words or less; try not to use formulas or special characters
% If you don't want an initial indentation, do \noindent at the start of the abstract

At a broad level, this dissertation's main claim is that many cases of pragmatic oddness do not stem from assertions alone, but rather from their interaction with the questions they implicitly  evoke. Felicitous assertions, must evoke felicitous questions. To operationalize this claim, a model of compositionally derived implicit question is devised, along with conditions of their well-formedness, drawing from familiar concepts in pragmatics, such as \textsc{Redundancy} and \textsc{Relevance}. This model assigns a central role to the degree of specificity, or granularity, conveyed by assertions.\\

At a more narrow level, this dissertation argues that disjunctions and conditionals fundamentally differ in terms of the questions they evoke, and that this difference has direct consequences on the oddness/felicity profiles of sentences involving these operators. Disjunctions are shown to be prone to \textsc{Redundancy} issues, while conditionals are shown to be prone to \textsc{Relevance} issues. In other words, disjunctions and conditionals typically display distinct flavors of oddness. This is supported by three main classes of sentences. First, sentences that can be seen as equivalent, but which combine conditionals and disjunctions in distinct ways, display varying felicity profiles. Second, ``pure'' disjunctions and conditionals that can be seen as isomorphic, if not equivalent, display varying felicity profiles. Third, some differences between these disjunctions and conditionals remain when additional pragmatic phenomena, in particular scalar implicatures, are at play, and such differences shift in a way predicted by our approach.\\

This dissertation therefore justifies the appeal to a more elaborate model of (implicit) questions, which, when fed to the pragmatic module, is characterized by a better empirical accuracy on challenging data, than previous model solely based on assertive content.
