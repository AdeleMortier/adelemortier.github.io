\sloppy
	

\chapter{Scalarity, information structure and Relevance in Hurford Conditionals}\label{chap:scalarity}

\begin{center}
	\textbf{Abstract}
\end{center}

\begin{small}
	Hurford Conditionals (HCs) involving scalemates appear felicitous, despite the fact that \textit{exh} is not predicted to rescue such structures from redundancy constraints previously introduced in the literature. We show that \textsc{Q-Relevance}, as introduced in Chapter \ref{chap:hurford-sentences}, can explain this pattern, modulo the intuitive assumption that scalar items can evoke fine-grained enough questions (generated by their scalemates) out-of-the-blue, while non-scalar items conveying different degrees of granularity cannot.
\end{small}
	\section{Hurford Sentences and scalar implicatures}
	In Chapter \ref{chap:hurford-sentences} we investigated Hurford Conditionals such as \textit{\# If Jo did not study in Paris, she studied in France}. In this Chapter, we investigate Hurford Conditionals involving exhaustifiable scalemates, and compare them to related disjunctions.
	\subsection{Hurford Disjunctions}
	
	
	Recall that Hurford Disjunctions (henceforth \textbf{HDs}, \cite{Hurford1974}), already introduced in Chapter \ref{chap:hurford-sentences}, typically involve entailing disjuncts and appear infelicitous regardless of the linear order of the disjuncts. This is shown in (\ref{ex:hd-ns}).
	
	\begin{exe}
		\ex\label{ex:hd-ns}
		\begin{xlist}
			\ex[\#] {Jo studied in \textcolor{blue}{France} or \textcolor{orange}{Paris}. \hfill \p{} $\vee$ \pplus}\label{ex:hd-w-s}
			\ex[\#] {Jo studied in \textcolor{orange}{Paris} or \textcolor{blue}{France}. \hfill \pplus{} $\vee$ \p}\label{ex:hd-s-w}
		\end{xlist}
	\end{exe}
	
	
	\citet{Gazdar1979} observed that infelicity disappears when (i) the Hurford disjuncts are scalemates, and (ii) the \textcolor{blue}{\textbf{weak}} disjunct precedes the \textcolor{orange}{\textbf{stronger}} one, as in (\ref{ex:hd-scalar-w-s}). However, when the order of the two disjuncts is reversed, as in (\ref{ex:hd-scalar-s-w}), infelicity tends to remain (\cite{Singh2008a, Singh2008b}). We call such disjunctions \textbf{scalar HDs}.
	
	
	
	\begin{exe}
		\ex\label{ex:hd-s}
		\begin{xlist}
			\ex[] {Jo read \textcolor{blue}{some} or \textcolor{orange}{all} of the books. \hfill \s{} $\vee$ \splus}\label{ex:hd-scalar-w-s}
			\ex[??] {Jo read \textcolor{orange}{all} or \textcolor{blue}{some} of the books. \hfill \splus{} $\vee$ \s}\label{ex:hd-scalar-s-w}
		\end{xlist}
	\end{exe}

	%\ex[] {Jo read all or only some of the books.}
	
	
	The asymmetry in (\ref{ex:hd-s}) has received several accounts (\cite{Singh2008a,Fox2018,Tomioka2021,HenotMortier2022} i.a.), all of which capitalize on the idea that (\ref{ex:hd-scalar-w-s}) can be rescued \textit{via} a local scalar implicature within the first disjunct (allowed by the covert operator \textit{exh}, \cite{Fox2007,Spector2008}); while (\ref{ex:hd-scalar-s-w}) cannot, due to an interaction between the first disjunct, and the licensing/timing of \textit{exh} in the second disjunct.
	
	In particular, \citet{Fox2018} suggest \textit{exh} should not be applied to an expression $E$ if it turns out to be Incrementally Weakening (abbreviated \textbf{IW}). Very roughly, \textit{exh} is IW in a sentence if it leads to an equivalent/weaker meaning no matter how the sentence is finished. The constraint is spelled out in (\ref{ex:economy}).
	
	\begin{exe}
		\ex\label{ex:economy} \textit{Economy Condition on Exhaustification (full version). } Let \textit{exh}$_C(A)$ be the exhaustification of $A$ given a set of alternatives $C$. *$S$[\textit{exh}$_C(A)$], if \textit{exh}$_C$ is incrementally weakening in $S$.
		\begin{xlist}
			\ex\label{ex:economy-gl-wk} {Let \textit{IE}$(A, C)$ the set of Innocently Excludable alternatives to $A$ that belong to $C$. An occurrence of \textit{exh}$_C$ is globally weakening in a sentence S[\textit{exh}$_C(A)$] if $\exists C': \text{IE}(A, C') \subset \text{IE}(A, C) \wedge S[\textit{exh}_{C'}(A)] \vDash S[\textit{exh}_C(A)]$}
			\ex\label{ex:economy-incr-wk} {An occurrence of \textit{exh} taking $A$ as argument is incrementally vacuous in $S$ if it is globally vacuous for every continuation of $S$ at point $A$.}
			\ex\label{ex:economy-cont} {$S'$ is a continuation of S at point $A$ if $S'$ can be derived from $S$ by replacement of constituents that follow $A$.}
			\ex\label{ex:economy-prec} {$Y$ follows $A$ if all the terminals of $Y$ are pronounced after those of $A$.}
		\end{xlist}
	\end{exe}
	A special case of interest is the following: if $C'$ is taken to be empty, $\textit{exh}_{C'}(A) = A$. This necessarily happens if $C$ is already a singleton, e.g. $C = \lbrace \forall \rbrace$: reducing it further to form $C'$ necessarily results in the empty set. An empty $C'$ then corresponds to the following subcase of (\ref{ex:economy-gl-wk}): \textit{exh} is globally weakening if deleting it from the sentence altogether leads to an equivalent or stronger meaning. \textit{exh} will in turn be IW if deleting it from the sentence leads to an equivalent or weaker meaning, \textit{no matter the continuation} after the point of deletion. This gives rise to the simplified constraint in (\ref{ex:economy-simple}).
	
	\begin{exe}
		\ex\label{ex:economy-simple} \textit{Economy Condition on Exhaustification (simplified version).} *$S$[\textit{exh}$_C(A)$], if \textit{exh}$_C$ is incrementally weakening in $S$.
			\begin{xlist}
				\ex {An occurrence of \textit{exh}$_C$ is globally weakening in a sentence S[\textit{exh}$_C(A)$] if $S[A] \vDash S[\textit{exh}_C(A)]$}
				\ex {cf. (\ref{ex:economy-incr-wk})}
				\ex {cf. (\ref{ex:economy-cont})}
				\ex {cf. (\ref{ex:economy-prec})}
		\end{xlist}
	\end{exe}
	
	
	We focus on this subcase here, given that the most salient set of innocently excludable alternatives to \textit{some} is already a singleton ($\lbrace \forall \rbrace$), whose only strict subset ($C'$) is thus $\emptyset$.\\
	
	Given IW, the contrast in (\ref{ex:hd-s}) then boils down to the fact \textit{exh} is not IW in the first disjunct of (\ref{ex:hd-scalar-w-s}) (cf. (\ref{ex:iw-hd-ws})), while it is in the second disjunct of (\ref{ex:hd-scalar-s-w}) (cf. (\ref{ex:iw-hd-sw})).
	
	
	\begin{exe}
		\ex\label{ex:iw-hd-scalar}
		\begin{xlist}
			\ex{$\exists\Gamma.$ exh(\s) $\Gamma$ $\equiv$ (\s $\wedge\neg$\splus) $\Gamma$ $\not\equiv$ \s{} $\Gamma$  (e.g., take $\Gamma$ to be empty)}\label{ex:iw-hd-ws}
			\ex{$\forall \Gamma.$ (\splus{} $\vee$ exh(\s)) $\Gamma$ $\equiv$ (\splus{} $\vee$ (\s $\wedge\neg$\splus)) $\Gamma$ $\equiv$ (\splus{} $\vee$ \s) $\Gamma$ }\label{ex:iw-hd-sw}
		\end{xlist}
	\end{exe}
	
	As a result, \textit{exh} can be applied in the first disjunct of (\ref{ex:hd-scalar-w-s}), and breaks the entailment between the two disjuncts (\textit{exh}($\exists$) = \sbna, and \sbna $\wedge \forall = \bot$). And because \textit{exh} cannot be applied to the second disjunct of (\ref{ex:hd-scalar-s-w}) due to being IW in this position, the problematic entailment between disjuncts remains and (\ref{ex:hd-scalar-w-s}) cannot be rescued from infelicity. This is illustrated in (\ref{ex:hd-s-exh}).
	
	\begin{exe}
		\ex\label{ex:hd-s-exh}
		\begin{xlist}
			\ex[] {Jo read exh(\textcolor{blue}{some}) or \textcolor{orange}{all} of the books. \hfill exh(\s) $\vee$ \splus\\
			Jo read \textcolor{blue}{some} \textbf{but not all} or \textcolor{orange}{all} of the books. \hfill (\s{} $\wedge$ $\neg\splus$) $\vee$ \splus}\label{ex:hd-scalar-w-s-exh}
			\ex[??] {Jo read \textcolor{orange}{all} or *exh(\textcolor{blue}{some}) of the books. \hfill \splus{} $\vee$ \s}\label{ex:hd-scalar-s-w-exh}
		\end{xlist}
	\end{exe}
	
	Note that (\ref{ex:hd-w-s}) cannot be rescued like (\ref{ex:hd-scalar-w-s}), either because \textit{Paris} is not a natural alternative to \textit{France} out-of-the blue, or because exhaustifying \textit{France} by-city would lead to a ``symmetry problem'' \citep{Kroch1972,Fox2007}; namely, it is impossible to negate all French cities and maintain consistency with the \textit{France} prejacent (and negating any subset of the French cities instead would be arbitrary). Both HDs in (\ref{ex:hd-ns}) are thus still predicted to be infelicitous.
	
	
	The criterion we used to determine infelicity here (entailment between disjuncts) assumes the traditional (non-explanatory) view of Hurford's Constraint. We will detail how the predictions of this view carry over in our QuD framework in CHapter \ref{chap:exh-incr}. Let us first see how changing scalar HDs into scalar Hurford Conditionals leads to an additional challenge, that will be the focus of this Chapter.
	
	\subsection{Hurford Conditionals}

	\subsubsection{The problem}
	Does the pattern exhibited by scalar HDs in (\ref{ex:hd-s}) extend to structures isomorphic to these HDs assuming material implication? \citet{Mandelkern2018} observed that an asymmetry arises in so-called Hurford Conditionals (henceforth \textbf{HCs}, cf. Chapter \ref{chap:hurford-sentences}), when the antecedent and consequent are \textit{not} natural scalemates, as in (\ref{ex:hc-ns}). Interestingly, we observe that the asymmetry \textit{disappears}\footnote{Some speakers I consulted reported that (\ref{ex:hc-scalar-w-ns}) was hard to make sense of in English (it is fine in my French). We discuss this \textit{caveat} towards the end of this Chapter.} in HCs involving scalemates, as shown in (\ref{ex:hc-s}). We call such structures \textbf{scalar HCs}.
	
	\begin{exe}
		\ex\label{ex:hc-ns}
		\begin{xlist}
			\ex[]{If Jo studied in \textcolor{blue}{France} she did not study in \textcolor{orange}{Paris}.\hfill \p{} $\rightarrow$ $\neg$\pplus}\label{ex:hc-w-ns}
			\ex[\#]{If Jo did not study in \textcolor{orange}{Paris} she studied in \textcolor{blue}{France}.\hfill $\neg$\pplus{} $\rightarrow$ \p}\label{ex:hc-ns-w}
		\end{xlist}
	\end{exe}
	
	
	%note: peter cannot make sense of the some then not all case...... which is weird bc this should be GOOD
	%peter could make sense of the not all then some case... which is predicted to be bad without forcing exh.
	
	\begin{exe}
		\ex\label{ex:hc-s}
		\begin{xlist}
			\ex{If Jo has read \textcolor{blue}{some} of the books she hasn't read \textcolor{orange}{all}.\hfill \s{} $\rightarrow$ $\neg$\splus}\label{ex:hc-scalar-w-ns}
			\ex{If Jo hasn't read \textcolor{orange}{all} of the books she has read \textcolor{blue}{some}.\hfill $\neg$\splus{} $\rightarrow$ \s}\label{ex:hc-scalar-ns-w}
		\end{xlist}
	\end{exe}

	
	
	HDs and HCs therefore pattern differently, in both the scalar and the non-scalar case. \citet{Kalomoiros2024} proposed a constraint called \textsc{Super Redundancy} accounting for (\ref{ex:hc-ns}), that we introduced in Chapter \ref{chap:hurford-sentences} and repeat here in (\ref{ex:sr}).
	
	\begin{exe}
		\ex \textsc{Super Redundancy}. A sentence $S$ is infelicitous if it contains a subconstituent $C$ combining with a binary operator, such that $(S)^-_C$ is defined and for all $D$, $(S)^-_C \equiv S_{Str(C, D)}$. In this definition, $(S)^-_C$ designates $S$ where $C$ got deleted. $Str(C, D)$ refers to a strengthening of $C$ with $D$, which commutes with negation ($Str(\neg\alpha, D) = \neg (Str(\alpha, D))$) and with binary operators ($Str(O(\alpha, \beta), D) = O(Str(\alpha, D), Str(\beta, D))$). $S_{Str(C, D)}$ designates $S$ where $C$ is replaced by $Str(C, D)$.\label{ex:sr}
	\end{exe}
	
	Let us briefly summarize how (\ref{ex:hc-ns}) is captured by \textsc{Super Redundancy}. (\ref{ex:hc-ns-w}) is Super Redundant (abbreviated \textbf{SR}), because any local strengthening of its antecedent (\textit{not Paris}) yields a conditional expression equivalent to its consequent (\textit{France}). This is proved in (\ref{ex:sr-hc-sw}). (\ref{ex:hc-w-ns}) on the other hand, is not SR: its antecedent (resp. consequent), can be strenghtened in such a way that the entire conditional becomes logically non-equivalent to its consequent (resp. antecedent). This is shown in (\ref{ex:sr-hc-ws}). SR can also cover the HDs in (\ref{ex:hd-ns}), and, together with IW, the scalar HDs in (\ref{ex:hd-s}).
	
	\begin{exe}
		\ex
		\begin{xlist}
			\ex {(\ref{ex:hc-ns-w}) is SR.\\
				C = $\neg$ \pplus. $\forall D.$ $\neg$(\pplus{} $\wedge$ $D$) $\rightarrow$ \p{} $\equiv$ (\pplus{} $\wedge$ $D$) $\vee$ \p{} $\equiv$ \p} \label{ex:sr-hc-sw}
			\ex {(\ref{ex:hc-w-ns}) is not SR.\\C = $\neg$ \pplus. Take $D = \top$. \p{} $\rightarrow$ $\neg$(\pplus $\wedge$ $D$) $\equiv$ \p{} $\rightarrow$ $\neg$(\pplus $\wedge$ $\top$) $\equiv$ \p{} $\rightarrow$ $\neg$\pplus{} $\not\equiv$ \p\\
				C = \p. Take $D = \bot$. (\p{} $\wedge$ $D$) $\rightarrow$ $\neg$\pplus{} $\equiv$ (\p{} $\wedge$ $\bot$) $\rightarrow$ $\neg$\pplus{} $\equiv$ $\bot$ $\rightarrow$ $\neg$\pplus{} $\equiv$ $\top$ $\not\equiv$ $\neg$\pplus} \label{ex:sr-hc-ws}
		\end{xlist} 
		
	\end{exe}
	
	What about (\ref{ex:hc-scalar-w-ns}) vs. (\ref{ex:hc-scalar-ns-w})? (\ref{ex:hc-s-w-ns-iw}) shows that that \textit{exh} is IW in the antecedent and the consequent of (\ref{ex:hc-scalar-w-ns}), whether the conditional is seen as material or as strict. (\ref{ex:hc-scalar-w-ns}) is therefore isomorphic to (\ref{ex:hc-w-ns}), and so is correctly predicted to be non-SR, like (\ref{ex:hc-w-ns}).
	
	
	
	
	\begin{exe}
		\ex\label{ex:hc-s-w-ns-iw}
		\begin{xlist}
			\ex\textit{exh} is IW in the antecedent of (\ref{ex:hc-scalar-w-ns}); material case.\\ $\forall \Gamma.$ exh(\s) $\rightarrow$ $\Gamma$ $\equiv$ $\neg$(\s$\wedge$$\neg$\splus) $\vee$ $\Gamma$ $\equiv$ $\neg$\s{} $\vee$ \splus{} $\vee$ $\Gamma$ $\Dashv$ $\neg$\s{} $\vee$ $\Gamma$ $\equiv$ \s{} $\rightarrow$ $\Gamma$
			\ex\textit{exh} is IW in the antecedent of (\ref{ex:hc-scalar-w-ns}); non-material case.\\ $\forall \Gamma.$ $\forall w: \text{exh}(\s)(w). \ \Gamma \equiv \forall w: \s(w) \wedge \neg\splus(w). \ \Gamma \Dashv \forall w: \s(w). \ \Gamma \equiv \s{} \rightarrow \Gamma$
			\ex\textit{exh} is IW in the consequent of (\ref{ex:hc-scalar-w-ns}); material case.\\
			$\forall \Gamma.$ (\s{} $\rightarrow$ exh($\neg$\splus)) $\Gamma$ $\equiv$ ($\neg$\s{} $\vee$ ($\neg$\splus $\wedge$ \s)) $\Gamma$ $\equiv$ ($\neg$\s{} $\vee$ $\neg$\splus) $\Gamma$ $\equiv$ (\s{} $\rightarrow$ $\neg$\splus) $\Gamma$
			\ex\textit{exh} is IW in the consequent of (\ref{ex:hc-scalar-w-ns}); non-material case.\\$\forall \Gamma.$ $\forall w: \s(w). \ \text{exh}(\neg\splus)(w) \equiv \forall w: \s(w). \ \neg\splus(w) \wedge \s(w) \Dashv \forall w: \s(w). \ \splus(w) \equiv \s{} \rightarrow \splus$
		\end{xlist}
	\end{exe}

	
	(\ref{ex:hc-s-ns-w-iw}) shows that this reasoning incorrectly extends to (\ref{ex:hc-scalar-ns-w}): \textit{exh} is IW in both the antecedent and the consequent of (\ref{ex:hc-scalar-ns-w}), so SR incorrectly predicts (\ref{ex:hc-scalar-ns-w}) to pattern like (\ref{ex:hc-ns-w}), i.e. to be infelicitous.
	
	\begin{exe}
		\ex\label{ex:hc-s-ns-w-iw}
		\begin{xlist}
			\ex\textit{exh} is IW in the consequent of (\ref{ex:hc-scalar-ns-w}); material case.\\ $\forall \Gamma.$ ($\neg$\splus{} $\rightarrow$ exh(\s)) $\Gamma$ $\equiv$ (\splus{} $\vee$ (\s$\wedge$$\neg$\splus)) $\Gamma$ $\equiv$ (\splus{} $\vee$ \s) $\Gamma$ $\equiv$ ($\neg$\splus{} $\rightarrow$ \s) $\Gamma$
			\ex\textit{exh} is IW in the consequent of (\ref{ex:hc-scalar-ns-w}); non-material case.\\ $\forall \Gamma.$ $\forall w: \neg\splus(w). \ \text{exh}(\s)(w) \equiv \forall w: \neg\splus(w). \ \s(w) \wedge \neg \splus(w) \equiv \forall w: \neg\splus(w). \ \s(w) \equiv$ $\neg$\splus $\rightarrow$ \s
			\ex\textit{exh} is IW in the antecedent of (\ref{ex:hc-scalar-ns-w}); material case.\\
			$\forall \Gamma.$ (exh($\neg$\splus) $\rightarrow$ \s{}) $\Gamma$ $\equiv$ ($\neg$($\neg$\splus $\wedge$ \s) $\vee$ \s{}) $\Gamma$ $\equiv$ (\splus $\vee$ $\neg$\s{} $\vee$ \s) $\Gamma$ $\Dashv$ ($\neg$\splus{} $\rightarrow$ \s) $\Gamma$
			\ex\textit{exh} is IW in the antecedent of (\ref{ex:hc-scalar-ns-w}); non-material case.\\$\forall \Gamma.$ $\forall w: \text{exh}(\neg\splus)(w). \ \s(w) \equiv \forall w: \neg\splus(w) \wedge \s(w). \ \s(w) \equiv \top \Dashv \splus \rightarrow \s$
		\end{xlist}
	\end{exe}

	\subsubsection{Exploring a potential solution}\label{sec:sr-iw-tentative-solution}
	At this point, one might want to revise IW, or SR. If SR is maintained and IW is assumed to be inactive in conditionals, then both HCs in (\ref{ex:hc-s}) would be correctly predicted to be felicitous, due to \textit{exh} being licensed in the consequent of (\ref{ex:hc-scalar-ns-w}). This is shown in (\ref{ex:sr-wo-iw}).
	
	\begin{exe}
		\ex\label{ex:sr-wo-iw} {(\ref{ex:hc-scalar-ns-w}) with \textit{exh} in the consequent is not SR.\\
		C = $\neg$ \splus. Take D = $\top$. $\neg$(\splus{} $\wedge$ $D$) $\rightarrow$ exh(\s) $\equiv$ \splus{} $\vee$ (\s{} $\wedge$ $\neg$\splus) $\equiv$ \s{} $\not\equiv$ exh(\s)\\
		C = exh(\s). Take D = $\bot$. $\neg$\splus{} $\rightarrow$ (exh(\s) $\wedge$ $D$) $\equiv$ $\neg$\splus{} $\rightarrow$ $\bot$ $\equiv$ \splus{} $\not\equiv$ $\neg$\splus}
	\end{exe}
	
	A possible argument against this view comes from a ``Long-Distance'', non-scalar variant of (\ref{ex:hc-scalar-ns-w}). At the end of Chapter \ref{chap:hurford-sentences}, we discussed two kinds of LDHDs derived from HDs by further disjoining the stronger disjunct with a proposition incompatible with the weaker one; and we observed that the infelicity of such LDHDs persists once the outer disjunction is changed into a conditional \textit{via} the \textit{or-to-if} tautology, as shown in (\ref{ex:ldhc}).
	
	This is problematic for the hypothesis that SR is the only constraint at stake in conditionals: under this view, and because \textit{exh} is inactive in the sentences in (\ref{ex:ldhc}), SR would be expected to rule them out beyond repair. But, if SR correctly rules out (\ref{ex:ldhc-pos}), it incorrectly rules in (\ref{ex:ldhc-neg}).
	
	\begin{exe}
		\ex \label{ex:ldhc}
		\begin{xlist}
			\ex[\#] {If Jo did not study in \textcolor{blue}{Europe}, she studied in \textcolor{orange}{France} or in \textcolor{green}{New York}.\\ $\neg$\p{} $\rightarrow$ (\pplus{} $\vee$ \r)}\label{ex:ldhc-pos}
			\ex[\#] {If Jo studied in \textcolor{blue}{France}, she did not study \textcolor{orange}{in Europe} or she studied in \textcolor{green}{Paris}.\\ $\neg$\textbf{\textcolor{blue}{q}} $\rightarrow$ (\textcolor{orange}{\textbf{q}$^+$} $\vee$ \r)
				with: \textbf{\textcolor{blue}{q}} := $\neg$\pplus; \textcolor{orange}{\textbf{q}$^+$} := $\neg$\p }\label{ex:ldhc-neg}
		\end{xlist}
	\end{exe}
	
	This suggests that our original problem in the scalar HC (\ref{ex:hc-scalar-ns-w}) cannot be easily alleviated by maintaining SR and relaxing IW, since in environments where IW plays no role, such as in (\ref{ex:ldhd-neg}) and (\ref{ex:ldhc-neg}), SR alone makes unexpected predictions.
	
	
	%also think about chap 3: p or p or q and scalar variants thereof. we want one involving a conditional, where exh is capital to ensure non SR.
	
	
	
	To capture the scalar HCs in (\ref{ex:hc-s}) and (\ref{ex:ldhc}) while retaining the right predictions for the HDs in (\ref{ex:hd-ns}), (\ref{ex:hd-s}) and the non-scalar HC in (\ref{ex:hc-ns}), we thus suggest to maintain IW, and propose an alternative to SR based on three ideas:
	\begin{itemize}
		\item Questions under Discussion (\textbf{QuD}, \cite{VanKuppevelt1995,Roberts1996}) are compositionally accommodated when processing out-of-the-blue declaratives (cf. previous chapters);
		\item QuD computation is constrained by \textsc{Q-Relevance} (cf. Chapter \ref{chap:hurford-sentences});
		\item scalemates may answer same-granularity QuDs, while non-scalemates with different levels of granularity cannot (new claim).
	\end{itemize}
	The scalar HCs in (\ref{ex:hc-s}) can then escape a violation of \textsc{Q-Relevance}, because their consequent can evoke a question of the form \textit{none, some but not all, or all?} that is fine-grained enough to ``fit'' a question introduced by their antecedent. In the non-scalar case, (\ref{ex:hc-w-ns}) can do the same (\textit{not Paris} evokes a proper subdivision of \textit{France}), but crucially not (\ref{ex:hc-ns-w}) (\textit{France} cannot evoke a proper subdivision of \textit{not Paris}).
	
	
	
	\section{Scalarity and accommodated QuDs}
	We use the two core ideas we entertained in the previous chapters of this thesis: that out-of-the-blue declaratives evoke the potential QuDs they may answer; and that the derivation of such implicit QuDs is compositional and (incrementally) constrained. In line with \citet{Katzir2015}'s insights, we take that  a sentence is odd if it is compatible with no reasonable implicit QuD. Chapter \ref{chap:hurford-sentences} already used this formalism to capture the non-scalar HD in (\ref{ex:hd-ns}) and the non-scalar HC in (\ref{ex:hc-ns}). We now focus on explaining the scalar HCs in (\ref{ex:hc-s}). The core claim we introduce in this section in order to capture (\ref{ex:hc-s}), is that scalemates \textit{may} evkoe similar QuDs, while non-scalemates like \textit{Paris} and \textit{France} cannot. Chapter \ref{chap:exh-incr} will cover the case of scalar HDs (\ref{ex:hd-s}) building on this assumption, while also presenting a possible alternative to IW.
	
	\subsection{Qtree recap}
	Let us briefly summarize the basis of the formalism presented in Chapter \ref{chap:hurford-sentences}. Building on \citet{Buring2003,Riester2019,Onea2016,Zhang2024}, we took QuDs to be trees (\textbf{Qtrees}), that have the Context Set (\textbf{CS}, \cite{Stalnaker1974}) as their root, and are such that each intermediate node is a subset of the CS, partitioned by its children nodes. Thus, the set of leaves of a Qtree forms a partition of the CS, and correspond to the standard denotation of questions (\citenp{Hamblin1958},\citenp{Groenendijk1999}). Any subtree rooted in $N$ can be seen as a conditional question, granted $N$. A proposition answers a Qtree if it can be identified with the union of a strict subset of the Qtree's nodes.
	
	\begin{figure}[H]
		\begin{multicols}{2}
			\begin{forest}
				[CS[A$_1$ [B$_1$][B$_2$]][A$_2$][A$_3$[B$_3$[C$_1$][C$_2$]][B$_4$]][A$_4$[B$_5$][B$_6$][B$_7$]]]
			\end{forest}
			\columnbreak
			\begin{small}

				The Tree on the left is a Qtree iff...
				\begin{itemize}
					\item $\lbrace A_1, A_2, A_3, A_4\rbrace$ partitions CS;
					\item $\lbrace B_1, B_2\rbrace$ partitions $A_1$;
					\item  $\lbrace B_3, B_4\rbrace$ partitions $A_3$;
					\item $\lbrace B_5, B_6, B_7\rbrace$ partitions $A_4$;
					\item $\lbrace C_1, C_2\rbrace$ partitions $B_3$.
				\end{itemize}
			\end{small}
		\end{multicols}
			\small
			It follows from this that...
			\begin{itemize}
				\item $\lbrace B_1, B_2, A_2, C_1, C_2, B_4, B_5, B_6, B_7\rbrace$ (leaves) partitions CS;
				\item $\lbrace B_1, B_2, A_2, B_3, B_4, B_4, B_5, B_6, B_7\rbrace$ partitions CS (because $\lbrace C_1, C_2\rbrace$ partitions $B_3$);
				\item $\lbrace B_1, B_2, A_2, C_1, C_2, B_4, A_4\rbrace$ partitions CS (because $\lbrace B_5, B_6, B_7\rbrace$ partitions $A_4$);
				\item etc.
			\end{itemize}
		\caption{Illustration of some Qtree properties.}
	\end{figure}
	
	Building on \citet{Katzir2015,HenotMortier2024a,HenotMortier2024b}, we take that any out-of-the-blue declarative sentence denoting a proposition \textit{p} gets paired with the set of salient Qtrees \textit{p} may answer. Such Qtrees additionally carry information about how $p$ answers them, in the form of specific nodes entailing $p$ (\textbf{verifying nodes}). We refer to the structure formed by Qtrees, along with their verifying nodes, as ``flagged Qtrees'' (or sometimes just Qtrees). The pairing between LF and flagged Qtrees is compositional, meaning, the flagged Qtrees evoked by a complex LF, are derived from the flagged Qtrees derived from its parts, and from how these parts combine. 
	
	
	\subsection{Qtrees of simplex LFs: scalar vs. non-scalar case}
	Chapter \ref{chap:hurford-sentences} defined the set of possible Qtrees evoked by a simplex LF $X$ denoting $p$. Roughly, we assumed that a Qtree for $X$ may be a depth-1 Qtree whose leaves denote $p$ and $\neg p$; a depth-1 Qtree whose leaves correspond to the Hamblin partition of the CS generated by $p$ and same-granularity alternatives to $p$; or a ``tiered'' Qtrees whose layers are each generated from a set of same-granularity alternatives to an alternative to $p$ entailed by $p$. We also assumed that in each case, Qtrees derived from simplex LFs get ``flagged'' by defining their verifying nodes as the set of nodes entailing $p$.
	
	
	\subsubsection{Defining the same-granularity relation}
	So far we took granularity as a primitive. We now submit that scalemates such as \textit{some} and \textit{all} \textit{may} be seen as same granularity alternatives to each other, while non-scalemates, like \textit{Paris} and \textit{France}, \textit{cannot} be considered being so, at least out-of-the blue. From this, it follows that \textit{some} and \textit{all} may answer the same QuD (partitioning the CS into the \textit{none}, \textit{some but not all}, and \textit{all} worlds), while \textit{Paris} and \textit{France} never do.
	
	At the intuitive level, the difference between \textit{some}/\textit{all} and \textit{France}/\textit{Paris} seems to be related to the symmetry problem \citep{Kroch1972,Fox2007} that arises with the latter kind of alternatives. If \textit{Paris} and all other cities are considered to be same-granularity alternatives, and if, on top of this, \textit{Paris} and \textit{France} are considered same-granularity, then by transitivity \textit{France} and any city in France should be considered same-granularity. This appears counter-intuitive, given that at a certain level of abstraction, all French cities together cover France. This intuition leads us to define same-granularity alternatives as in (\ref{ex:same-gran-alt}). (\ref{ex:same-gran-alt-unpacking}) clarifies some of the terms introduced in (\ref{ex:same-gran-alt}).
	
	\begin{exe}
		\ex {\textit{Set of same granularity alternatives to q. } Let $X$ be a LF denoting $p$ and $\mathcal{A}_{p, X}$ be the set of all possible alternatives to $p$, obtained by the replacement of focused material in $X$ by relevant, same-complexity and same-type constituents. Let $\mathcal{H}(\mathcal{A}_{p, X})$ be the Hasse diagram generated by $\vDash$ on $\mathcal{A}_{p, X}$, directed from top (logically stronger) to bottom (logically weaker). For any $q \in \mathcal{A}_{p, X}$, a set of same-granularity alternatives to $q$ ($\mathcal{A}_{p, X}^q$) is obtained by:
			\begin{enumerate}
				\item\label{ex:same-gran-same-level} (obligatory) adding all same-level alternatives to $q$ in $\mathcal{H}(\mathcal{A}_{p, X})$ to $\mathcal{A}_{p, X}^q$;
				\item\label{ex:same-gran-higher-level} (optional) for each level higher than $q$'s level in $\mathcal{H}(\mathcal{A}_{p, X})$ (from the lowest to the highest level), adding to $\mathcal{A}_{p, X}^q$ all the alternatives that are not yet covered by a subset of $\mathcal{A}_{p, X}^q$. A set $\mathcal{S}$ of sets covers another set $s$ is $\bigcup\mathcal{S} = s$.
				\item\label{ex:same-gran-lower-level} (optional) for each level lower than $q$'s level in $\mathcal{H}(\mathcal{A}_{p, X})$ (from the highest to the lowest level), adding to $\mathcal{A}_{p, X}^q$ the grand intersection of the maximal sets of alternatives that together do not cover any subset of $\mathcal{A}_{p, X}^q$.\footnote{Note that this relates to \citet{Fox2007}'s notion of Innocent Exclusion: the goal here is to non-arbitrarily \textit{include} (rather than \textit{exclude}) a subset of alternatives that together do not already cover a union of alternatives in $\mathcal{A}_{p, X}^q$. In particular, if the set of alternatives considered at a given level is symmetric w.r.t. some alternative already present in $\mathcal{A}_{p, X}^q$, then, none of these alternatives will be added to $\mathcal{A}_{p, X}^q$.}
			\end{enumerate} }\label{ex:same-gran-alt}
		The three steps are ordered, and steps (\ref{ex:same-gran-alt}.\ref{ex:same-gran-higher-level}-\ref{ex:same-gran-lower-level}) are optional.
	\end{exe}
	
	\begin{exe}
		\ex \label{ex:same-gran-alt-unpacking}
		\begin{xlist}
			\ex {\textit{Same-level nodes in a Hasse diagram.} $p$ and $q$ are same-level nodes in a Hasse diagram $\mathcal{H}$ if $\exists r \in \mathcal{H}. \ \exists n \in \mathbb{N}. \ p \ \text{\MVRightarrow}^n \ r \wedge q \ \text{\MVRightarrow}^n \ r$, where $\text{\MVRightarrow}^n$ represents $n$ iterations of the accessibility relation in $\mathcal{H}$.}
			\ex {\textit{Level in a Hasse diagram.} $\mathcal{L}$ is a level in a Hasse diagram $\mathcal{H}$ iff $\exists p \in \mathcal{H}. \ \mathcal{L} = \lbrace q \in \mathcal{H} \ | \ \text{$p$ and $q$ are same-level nodes in } \mathcal{H}\rbrace$}
			\ex {\textit{Level higher/lower than a node in a Hasse diagram.} A level $\mathcal{L}$ in a Hasse diagram $\mathcal{H}$ is higher than a node $p$ if $\exists q \in \mathcal{L}. \ p \ \text{\MVRightArrow}^* \ q$. It is lower than a node $p$ if $\exists q \in \mathcal{L}. \ q \ \text{\MVRightArrow}^* \ p$.}
		\end{xlist}
	\end{exe}
	
	
	Let us now see how these definitions works when considering same-granularity alternatives to LFs containing \textit{Paris}, \textit{some}, and \textit{all}.
	
	\subsubsection{Non-scalar items}
	Starting with \textit{Paris}, one should consider $\mathcal{A}_{\textit{Paris}}$ to be a set of locations organized in a Hasse diagram like Figure \ref{fig:paris-hasse}.
	
		\begin{figure}[H]
		\centering
		\begin{forest}
			[World[Europe[France[Île-de-France[Paris [Ier][...][XXème]][Versailles [...]]][Grand Ouest[...]][...]][UK[...]][...]][Asia[China[...]][India[...]][...]][...][Antarctica]]
		\end{forest}
		\caption{Hasse diagram generated by alternatives to \textit{Paris}. Entailment goes upward. Nodes that are vertically aligned are same-level.}
		\label{fig:paris-hasse}
	\end{figure}
	
	In this diagram, \textit{Paris} is at a level that typically involves other cities. Applying (\ref{ex:same-gran-alt}.\ref{ex:same-gran-same-level}) then adds all those cities to the set of same-granularity alternatives to \textit{Paris}; so at this point, $\mathcal{A}_{\textit{Paris}}^{\textit{Paris}}$ = $\lbrace \textit{Paris}, \textit{Versailles}, \textit{Lyon}, \textit{London} ...\rbrace$. Figure \ref{fig:paris-hasse} shows that some higher-level locations, like \textit{Antarctica}, may not be subdivided into cities. Applying (\ref{ex:same-gran-alt}.\ref{ex:same-gran-higher-level}) then adds \textit{Antarctica} to $\mathcal{A}_{\textit{Paris}}^{\textit{Paris}}$, since no subset of cities that are already part of $\mathcal{A}_{\textit{Paris}}^{\textit{Paris}}$ covers it. Countries like \textit{France} and \textit{Germany} cannot be added to $\mathcal{A}_{\textit{Paris}}^{\textit{Paris}}$ in the same way, because they are covered by a subset of cities that are already part of $\mathcal{A}_{\textit{Paris}}^{\textit{Paris}}$. So at this point, $\mathcal{A}_{\textit{Paris}}^{\textit{Paris}} = \lbrace \textit{Paris}, \textit{Versailles}, \textit{Lyon}, \textit{London}, ...  \textit{Antarctica}\rbrace$. Assuming any city comes with a set of districts that fully partition it (i.e. districts are symmetric w.r.t their respective cities), (\ref{ex:same-gran-alt}.\ref{ex:same-gran-lower-level}) applies vacuously, since for no city already present in $\mathcal{A}_{\textit{Paris}}^{\textit{Paris}}$ is it possible to non-arbitrarily add to $\mathcal{A}_{\textit{Paris}}^{\textit{Paris}}$ a set of districts that does not fully cover the given city. In sum, we derive that same-granularity alternatives to \textit{Paris} are typically cities, but may also involve intuitively coarser-grained alternatives that cannot be reasonably subdivided into cities (\textit{e.g.} \textit{Antarctica}). Crucially, no location that is subdivided into cities is part of this set.
	
	This reasoning easily extends to an intuitively coarser-grained alternative to \textit{Paris} like \textit{France}. Same-granularity alternatives to \textit{France} are typically countries, but may also involve intuitively coarser-grained alternatives that cannot be reasonably subdivided into countries (\textit{e.g.} \textit{Antarctica}). Just like districts (which tend to partition cities) could not be added to the set of same-granularity alternatives to \textit{Paris}, cities (which tend to partition countries) cannot be added to the set of same-granularity alternatives to \textit{France}. A consequence of this, is that non-scalemates like \textit{Paris} and \textit{France} have inherently distinct sets of same-granularity alternatives -- and in turn, will be predicted to give rise to inherently distinct sets of Qtrees.

	
	\subsubsection{Scalar items}
	Now turning to \textit{some} and \textit{all}. We assume that the set of alternatives for such items is typically made of $\forall$ (\textit{all}), $\exists$ (\textit{some}), and $\neg\exists$ (\textit{none}), but does not contain, e.g. $\neg\forall$ or $\exists\wedge\neg\forall$, because such logical meanings correspond to expressions (\textit{not all}, \textit{some but not all}) that are strictly more complex. So, $\mathcal{A}_{\textit{some}}=\mathcal{A}_{\textit{all}}=\lbrace \neg\exists, \exists, \forall\rbrace$. The resulting Hasse diagram for both \textit{some} and \textit{all} is given in Figure \ref{fig:some-all-hasse}.
	
	\begin{figure}[H]
		\centering
		\begin{forest}
			[$\exists$[$\forall$]]
		\end{forest}
		\begin{forest}
			[$\neg\exists$]
		\end{forest}
		\caption{Hasse diagram generated by alternatives to \textit{some}/\textit{all}. Entailment goes upward. Each node belongs to a different level.}
		\label{fig:some-all-hasse}
	\end{figure}
	
	In this diagram, $\exists$ and $\forall$ are at different levels, since $\forall$ strictly entails $\exists$. To build a set of same-granularity alternatives to $\exists$ ($\mathcal{A}_{\textit{some}}^{\textit{some}}$=$\mathcal{A}_{\textit{all}}^{\textit{some}}$), we start by applying (\ref{ex:same-gran-alt}.\ref{ex:same-gran-same-level}), which adds $\exists$ to $\mathcal{A}_{\textit{some}}^{\textit{some}}$/$\mathcal{A}_{\textit{all}}^{\textit{some}}$. Applying (\ref{ex:same-gran-alt}.\ref{ex:same-gran-higher-level}) is vacuous, since there is no higher level above $\exists$. Applying (\ref{ex:same-gran-alt}.\ref{ex:same-gran-lower-level}) then adds $\forall$ to $\mathcal{A}_{\textit{some}}^{\textit{some}}$/$\mathcal{A}_{\textit{all}}^{\textit{some}}$, because doing so is non-arbitrary (only possibility), and $\forall$ does not cover $\exists$ ($\forall$ is strictly contained in $\exists$). Note that the absence of $\exists\wedge\neg\forall$ from the Hasse diagram is crucial to derive this: had $\exists\wedge\neg\forall$ been present, $\exists\wedge\neg\forall$ and $\forall$ would have been symmetric w.r.t. $\exists$, and none of these alternatives could have been non-arbitrarily added to $\mathcal{A}_{\textit{some}}^{\textit{some}}$/$\mathcal{A}_{\textit{all}}^{\textit{some}}$. In sum, $\mathcal{A}_{\textit{some}}^{\textit{some}}$=$\mathcal{A}_{\textit{all}}^{\textit{some}} = \lbrace\exists\rbrace$ (by only applying step (\ref{ex:same-gran-alt}.\ref{ex:same-gran-same-level})) or $\lbrace\exists, \forall\rbrace$ (by applying all steps).\\
	
	To build a set of same-granularity alternatives to $\forall$ ($\mathcal{A}_{\textit{some}}^{\textit{all}}$=$\mathcal{A}_{\textit{all}}^{\textit{all}}$), we start by applying (\ref{ex:same-gran-alt}.\ref{ex:same-gran-same-level}), which adds $\forall$ to $\mathcal{A}_{\textit{some}}^{\textit{all}}$/$\mathcal{A}_{\textit{all}}^{\textit{all}}$. Applying (\ref{ex:same-gran-alt}.\ref{ex:same-gran-higher-level}) then adds $\exists$ to $\mathcal{A}_{\textit{some}}^{\textit{all}}$/$\mathcal{A}_{\textit{all}}^{\textit{all}}$, because $\forall$ does not cover $\exists$. Applying (\ref{ex:same-gran-alt}.\ref{ex:same-gran-lower-level}) is vacuous, because $\forall$ already forms the lowest level of the diagram. In sum, $\mathcal{A}_{\textit{some}}^{\textit{all}}$=$\mathcal{A}_{\textit{all}}^{\textit{all}} = \lbrace\forall\rbrace$ (by only applying step (\ref{ex:same-gran-alt}.\ref{ex:same-gran-same-level})) or $\lbrace\exists, \forall\rbrace$ (by applying all steps).\\
	
	
	We therefore derive that \textit{some} and \textit{all} \textit{may} give rise to the same set of same-granularity alternatives, namely, $\lbrace \exists, \forall\rbrace$. This will eventually predict that \textit{some} and \textit{all} may give rise to the same kind of Qtree, namely, a Qtree partitioning the CS into $\neg\exists$-, (\sbna)-, and $\forall$-worlds. Note however that \textit{some} and \textit{all} may also give rise to distinct sets of same-granularity alternatives, respectively $\lbrace \exists\rbrace$ and $\lbrace\forall\rbrace$ -- if we assume that only step (\ref{ex:same-gran-alt}.\ref{ex:same-gran-same-level}) is applied. Zooming out, definition (\ref{ex:same-gran-alt}) allowed to model a crucial distinction between non-scalemates like \textit{Paris} and \textit{France}, and scalemates like \textit{some} and \textit{all}: the former will never give rise to the same sets of same-granularity alternatives, while the latter can. Because Qtrees for simplex LFs were defined layer-by-layer based on the notion of same-granularity alternatives back in Chapter \ref{chap:hurford-sentences}, we derive that non-scalemates will never give rise to identical Qtrees, while scalemates may do so.
	
	\subsubsection{Deriving Qtrees for scalemates and non-scalemates}\label{sec:qtrees-scalemates-non-scalemates}
	
	Now that same-granularity alternatives are defined for scalar and non-scalar items, we are in a position to apply the recipe (\ref{ex:qtree-simplex-def}) from Chapter \ref{chap:hurford-sentences} to derive Qtrees for sentences like \textit{Jo studied in Paris} or \textit{Jo read some of the books}.
	
	Starting with the LF \textit{X$^+$ = Jo studied in Paris}. Following principle (\ref{ex:qtree-simplex-def}.\ref{pt:simplex-qtree-polar}), a ``polar'' Qtree can be built out of \textit{X}$^+$ from the partition $\lbrace$\textit{Paris}, $\neg$\textit{Paris}$\rbrace$. This is done in Figure \ref{tree:paris-polar}. Following principle (\ref{ex:qtree-simplex-def}.\ref{pt:simplex-qtree-wh}), one must generate a Hamblin partition out of a set of same-granularity alternatives to \textit{X}$^+$. We just saw that same-granularity alternatives to \textit{X}$^+$ form a set $\lbrace$\textit{Paris}, \textit{Nice}, \textit{London}, ...$\rbrace$ containing cities (and intuitively coarser-grained locations that are not partitioned by cities). This set happens to be equal to its Hamblin partition, given that the alternatives it contains are already mutually exclusive. Applying principle (\ref{ex:qtree-simplex-def}.\ref{pt:simplex-qtree-wh}) using this Hamblin partition then generates the Qtree in Figure \ref{tree:paris-wh}. Lastly, according to principle (\ref{ex:qtree-simplex-def}.\ref{pt:simplex-qtree-tiered}), \textit{X}$^+$ can give rise to a chain of entailing propositions of the form \textit{Jo studied in Paris}, \textit{Jo studied in France}, \textit{Jo studied in Europe} etc. Restricting ourselves to the \textit{Paris}-\textit{France} chain, a ``tiered'' Qtree can be created by generating Hamblin partitions from same-granularity alternatives to \textit{Paris} and \textit{France}. We just saw that the Hamblin partition obtained for \textit{Jo studied in Paris} takes the form $\lbrace$\textit{Paris}, \textit{Nice}, \textit{London}, ...$\rbrace$. Similarly for \textit{Jo studied in France}, the relevant Hamblin partition corresponds to country alternatives (and potentially intuitively coarser grained alternatives that are not subdivided by countries), i.e. $\lbrace$\textit{France}, \textit{UK}, ...$\rbrace$. Following principle (\ref{ex:qtree-simplex-def}.\ref{pt:simplex-qtree-tiered}), a ``tiered'' Qtree for \textit{X}$^+$ is then built by ``stacking'' the \textit{Paris} and \textit{France} partitions, as done in Figure \ref{tree:paris-tiered}. Of course, principle (\ref{ex:qtree-simplex-def}.\ref{pt:simplex-qtree-tiered}) may generate more than one ``tiered'' Qtree, e.g., a Qtree with a continent tier on top of a country tier. We omit these extra Qtrees for simplicity.
	
	\begin{figure}[H]
		\centering
		\begin{subfigure}[b]{.3\linewidth}
			\centering
				\scalebox{1}{
				\begin{forest}
					for tree={s sep=2mm, inner sep=0, l=0}
					[CS[\ofbox{Paris}][$\neg$Paris]]
			\end{forest}}
		\caption{}\label{tree:paris-polar}
		\end{subfigure}
		\hfill
		\begin{subfigure}[b]{.3\linewidth}
			\centering
			\scalebox{1}{
				\begin{forest}
					for tree={s sep=2mm, inner sep=0, l=0}
					[CS[\ofbox{Paris}][Nice][London][...]]
			\end{forest}}
			\caption{}\label{tree:paris-wh}
		\end{subfigure}\hfill
		\begin{subfigure}[b]{.3\linewidth}
			\centering
			\scalebox{1}{
				\begin{forest}
					for tree={s sep=2mm, inner sep=0, l=0}
					[CS[France[\ofbox{Paris}][Nice]][UK[London][...]][...]]
			\end{forest}}
			\caption{}\label{tree:paris-tiered}
		\end{subfigure}
		
		
		\caption{Qtrees for \textit{Jo studied in \textbf{\textcolor{orange}{Paris}}.}}\label{fig:city-partition}
	\end{figure}
	
	Constructing Qtrees for the LF \textit{X = Jo studied in France} follows a very similar line of reasoning. Following principle (\ref{ex:qtree-simplex-def}.\ref{pt:simplex-qtree-polar}), a ``polar'' Qtree can be built out of \textit{X} from the partition $\lbrace$\textit{France}, $\neg$\textit{France}$\rbrace$. This is done in Figure \ref{tree:france-polar}.
	
	
	
	Following principle (\ref{ex:qtree-simplex-def}.\ref{pt:simplex-qtree-wh}), one must generate a Hamblin partition out of a set of same-granularity alternatives to $X$. We just saw that same-granularity alternatives to \textit{X} form a set $\lbrace$\textit{France}, \textit{UK}, ...$\rbrace$ containing countries (and intuitively coarser-grained locations that are not partitioned by countries); and that this set happens to be equal to its Hamblin partition. Applying principle (\ref{ex:qtree-simplex-def}.\ref{pt:simplex-qtree-wh}) using this Hamblin partition then generates the Qtree in Figure \ref{tree:france-wh}. Lastly, according to principle (\ref{ex:qtree-simplex-def}.\ref{pt:simplex-qtree-tiered}), \textit{X} can give rise to a chain of entailing propositions of the form \textit{Jo studied in France}, \textit{Jo studied in Europe}, etc. For simplicity, and to remain consistent with how we dealt with \textit{X}$^+$=\textit{Jo studied in Paris}, we omit the tiered Qtrees generated from this kind of chain.
	
	\begin{figure}[H]
		\centering
		\begin{subfigure}[b]{.3\linewidth}
			\centering
			\begin{forest}
				[CS[\bfbox{France}][$\neg$France]]
			\end{forest}
			\caption{}\label{tree:france-polar}
		\end{subfigure}
		\qquad
		\begin{subfigure}[b]{.3\linewidth}
			\centering
			\begin{forest}
				[CS[\bfbox{France}][UK][...]]
			\end{forest}
			\caption{}\label{tree:france-wh}
		\end{subfigure}
		\caption{Qtrees for \textit{Jo studied in \textbf{\textcolor{blue}{France}}}.}
	\end{figure}
	
	We can now turn to the scalar case, with \textit{Y}$^+$=\textit{Jo read all of the books}, and \textit{Y}=\textit{Jo read some of the books}.
	
	\begin{figure}[H]
		\centering
		\begin{subfigure}[b]{.2\linewidth}
			\centering
			\begin{forest}
				[CS[Paris][$\neg$Paris]]
			\end{forest}
			\caption{If $\mathcal{A}_{\textit{Paris}} = \lbrace \textit{Paris}\rbrace$}
		\end{subfigure}
		\hfill
		\begin{subfigure}[b]{.3\linewidth}
			\centering
			\begin{forest}
				[CS[Paris][Lyon][Berlin][...]]
			\end{forest}
			\caption{If $\mathcal{A}_{\textit{Paris}} = \lbrace \textit{Paris}, \textit{Lyon}, \textit{Berlin}, ...\rbrace$}
		\end{subfigure}
		\hfill
		\begin{subfigure}[b]{.45\linewidth}
			\centering
			\begin{forest}
				[CS[France[Paris][Lyon][...]][Germany[Berlin][...]][...]]
			\end{forest}
			\caption{If $\mathcal{A}_{\textit{Paris}}^{\textit{Paris}} = \lbrace \textit{Paris}\rbrace$ and  $\mathcal{A}_{\textit{Paris}}^{\textit{France}} = \lbrace \textit{France}, \textit{Germany}, ...\rbrace$}
		\end{subfigure}
	\end{figure}
	\begin{figure}[H]
		\centering
		\begin{subfigure}[b]{.2\linewidth}
			\centering
			\begin{forest}
				[CS[$\exists$][$\neg\exists$]]
			\end{forest}
			\caption{If $\mathcal{A}_{\textit{some}} = \lbrace\exists\rbrace$}
		\end{subfigure}
		\hfill
		\begin{subfigure}[b]{.3\linewidth}
			\centering
			\begin{forest}
				[CS[$\neg\exists$][\sbna][$\forall$]]
			\end{forest}
			\caption{If $\mathcal{A}_{\textit{some}} = \lbrace \exists, \forall\rbrace$}
		\end{subfigure}
	\end{figure}
	\begin{figure}[H]
		\centering
		\begin{subfigure}[b]{.2\linewidth}
			\centering
			\begin{forest}
				[CS[$\forall$][$\neg\forall$]]
			\end{forest}
			\caption{If $\mathcal{A}_{\textit{all}} = \lbrace\forall\rbrace$}
		\end{subfigure}
		\hfill
		\begin{subfigure}[b]{.3\linewidth}
			\centering
			\begin{forest}
				[CS[$\neg\exists$][\sbna][$\forall$]]
			\end{forest}
			\caption{If $\mathcal{A}_{\textit{all}} = \lbrace \exists, \forall\rbrace$}
		\end{subfigure}
		\hfill
		\begin{subfigure}[b]{.3\linewidth}
			\centering
			\begin{forest}
				[CS[$\exists$[$\forall$][\sbna]][$\neg\exists$]]
			\end{forest}
			\caption{If $\mathcal{A}_{\textit{all}}^{\textit{all}} = \lbrace \exists, \forall\rbrace$ or $\lbrace\forall\rbrace$  and $\mathcal{A}_{\textit{all}}^{\textit{some}} = \lbrace \exists\rbrace$}
		\end{subfigure}
	\end{figure}
	
	
	
	
	
	
	
	Intuitively, this renders the intuition that non-scalemates like \textit{Paris} and \textit{France} will answer different kinds of questions -- finer-grained \textit{which city?} question (\ref{ex:which-city}), vs. coarser-grained \textit{which country?} question (\ref{ex:which-country}) -- while scalemates \textit{may} answer similar questions -- \textit{e.g.} \textit{how much/many?} (\ref{ex:how-many}).
	
	
	
	
	
	In (\ref{ex:which-city}), a hedges like \textit{all I know is that...} allows to shift the question and be less informative than originally expected (\ref{ex:which-city-aik}). Being more informative a
	
	\begin{exe}
		\ex {In which city does Jo study?}\label{ex:which-city}
		\begin{xlist}
			\ex[] {Jo studied in Paris.}
			\ex [\#] {Jo studied in France.}
			\ex [] {All I know is that Jo studied in France.}\label{ex:which-city-aik}
		\end{xlist}
		\ex {In which country does Jo study?}\label{ex:which-country}
		\begin{xlist}
			\ex[] {Jo studied in France.}
			\ex [??] {Jo studied in Paris.}
			\ex [\#] {All I know is that Jo studied in Paris.}
		\end{xlist}
		\ex {How many students passed the class?}\label{ex:how-many}
		\begin{xlist}
			\ex[] {All passed.}
			\ex [] {Some passed.}
			\ex [] {All I know is that some passed.}
		\end{xlist}
	\end{exe}
	
	One might argue that \textit{Paris} and \textit{France} may in fact answer the same, more general question: \textit{where?}. We think this kind of question can be coerced by the answerer into a more specific question (e.g. \textit{which city?}), depending on how informed they are. That kind of coercion does not seem to be needed in the case of \textit{how much/many?} questions answered by \textit{some} or \textit{all}.
	
		
	
	
	\iffalse
	. They can also be seen as belonging to different tiers,\textit{some} being less granular than \textit{all}.
	; and whose (potential) intermediate layers corresponds to the Hamblin partition of the CS generated by $q$ (an alternative to $p$ entailed by $p$) and same-granularity alternatives to $q$. Such tiered Qtrees can be interpreted as a stack of \textit{wh}-questions answered by $p$, and whose specificity increases from top to bottom.
	\begin{minipage}{.26\linewidth}
		\centering
		\begin{figure}[H]
			\centering
			\scalebox{.8}{
				\begin{forest}
					[CS[$\neg\exists$][$\exists$[\sbna][\ofbox{$\forall$}]]]
			\end{forest}}
			\caption{``Wh'' Qtree for \textit{Jo read \textit{\textcolor{orange}{all}} of the books}.}\label{fig:all-not-all-some-not-some}
		\end{figure}
	\end{minipage}\hfill
	\hfill
	\scalebox{.8}{
		\begin{forest}
			for tree={s sep=2mm, inner sep=0, l=0}
			[CS[FR[\ofbox{Paris}][Nice][...]][UK[London][...]][...]]
	\end{forest}}
	\fi
	
	According to this definition, \textit{Jo read all of the books} gets paired with a ``polar'' Qtree corresponding to whether or not she read all the books (cf. Fig. \ref{fig:all-not-all}); and a ``wh'' Qtree corresponding to whether she read none, only some, or all of the books (generated by Alt($\exists$) = $\lbrace \exists, \forall\rbrace$, cf. Fig. \ref{fig:none-sbna-all}). Same can be done for \textit{Jo read some of the books}, except the ``polar'' Qtree is different (cf. Fig. \ref{fig:some-not-some}). For \textit{Jo studied in Paris} (resp. \textit{France}), \textit{wh}-Qtrees are generated by city (resp. country) alternatives, cf. Fig. \ref{fig:city-partition} and \ref{fig:country-partition}. Verifying nodes are boxed.
	
	\hspace*{-6mm}\begin{minipage}{.25\linewidth}
		\centering
		\begin{figure}[H]
			\centering
			\scalebox{.8}{
				\begin{forest}
					for tree={s sep=2mm, inner sep=0, l=0}
					[CS[$\neg\forall$][\ofbox{$\forall$}]]
			\end{forest}}
			\caption{``Polar'' Qtree for \textit{Jo read \textbf{\textcolor{orange}{all}} of the books}}\label{fig:all-not-all}
		\end{figure}
	\end{minipage}\hfill
	\begin{minipage}{.3\linewidth}
		\centering
		\begin{figure}[H]
			\centering
			\scalebox{.8}{
				\begin{forest}
					for tree={s sep=2mm, inner sep=0, l=0}
					[CS[$\neg\exists$][\bfbox{\sbna}][\bfbox{\ofbox{$\forall$}}]]
			\end{forest}}
			\caption{``Wh'' Qtree for \textit{Jo read $\lbrace$\textbf{\textcolor{blue}{some}}/\textbf{\textcolor{orange}{all}}$\rbrace$ of the books}.}\label{fig:none-sbna-all}
		\end{figure}
	\end{minipage}\hfill
	\begin{minipage}{.25\linewidth}
		\centering
		\begin{figure}[H]
			\centering
			\scalebox{.8}{
				\begin{forest}
					for tree={s sep=2mm, inner sep=0, l=0}
					[CS[$\neg\exists$][\bfbox{$\exists$}]]
			\end{forest}}
			\caption{``Polar'' Qtree for \textit{Jo read \textbf{\textcolor{blue}{some}} of the books}}\label{fig:some-not-some}
		\end{figure}
	\end{minipage}
	
	
	
	\begin{minipage}{.4\linewidth}
		\centering
		\begin{figure}[H]
			\centering
			\scalebox{.8}{
				\begin{forest}
					for tree={s sep=2mm, inner sep=0, l=0}
					[CS[\bfbox{FR}][$\neg$FR]]
			\end{forest}}\hfill
			\scalebox{.8}{
				\begin{forest}
					for tree={s sep=2mm, inner sep=0, l=0}
					[CS[\bfbox{FR}][UK][...]]
			\end{forest}}
			\caption{Qtrees for \textit{Jo studied in \textbf{\textcolor{blue}{France}}.}}\label{fig:country-partition}
		\end{figure}
	\end{minipage}\vspace{3mm}
	
	\subsection{Getting compositional}
	Just like the meanings of simple sentences are incrementally composed, their sets of candidate Qtrees get incrementally combined. The Qtrees compatible with a negated LF $\neg X$, are Qtrees for $X$ in which the set of compatible nodes is ``flipped'' on a layer-by-layer basis. \textit{Jo did not read all of the books} is thus linked to the Qtrees in Fig. \ref{fig:all-neg} and \textit{Jo didn't study in Paris} to those in Fig. \ref{fig:city-partition-neg}. The Qtrees compatible a disjunctive LF $X \vee Y$, are all the Qtrees that result from the union of a tree for $X$, and a tree for $Y$. The union operation -- understood as union over sets of nodes, sets of edges, and sets of verifying nodes -- ensures that the Qtree of a disjunction addresses the QuDs evoked by \textit{both} disjuncts in parallel (\cite{Simons2001,Zhang2024}). \textit{Jo read some or all of the books} is therefore only compatible with Tree \ref{fig:none-sbna-all} because all the other unions obtained from of Trees \ref{fig:all-not-all}, \ref{fig:some-not-some} and \ref{fig:none-sbna-all} fail to generate proper Qtrees. The HDs (\ref{ex:hd-ns}) are compatible with no Qtree, because the Qtrees for \textit{Paris} and those for \textit{France} always subdivide the CS differently.\footnote{\citet{HenotMortier2024a,HenotMortier2024b} predict that (\ref{ex:hd-w-s}-\ref{ex:hd-s-w}) do create proper Qtrees, but that such Qtrees (paired with their LFs) are \textsc{Q-Redundant}.} The Qtrees compatible with a conditional LF $X\rightarrow Y$ are Qtrees for $X$, where each verifying node is replaced by its intersection with a Qtree for $Y$. Verifying nodes are inherited from the consequent Qtree (in line with the observations in (\ref{ex:depending-on})). (\ref{ex:hc-scalar-w-ns}) is then compatible with the Tree in Fig. \ref{fig:if-some-then-not-all}; (\ref{ex:hc-scalar-ns-w}), with Fig. \ref{fig:if-not-all-then-some}, (\ref{ex:hc-w-ns}) with Fig. \ref{fig:hc-non-scalar-w-ns} and (\ref{ex:hc-ns-w}) with Fig. \ref{fig:hc-non-scalar-ns-w}. We proceed to show that both trees associated with (\ref{ex:hc-ns-w}) violate some notion of relevance; while no trees associated with (\ref{ex:hc-scalar-w-ns}), (\ref{ex:hc-scalar-ns-w}), and (\ref{ex:hc-w-ns}) do. Roughly, the issue is that none of the trees evoked by (\ref{ex:hc-ns-w}) fully preserve the answer conveyed by its consequent (the \textit{France}-node); while those evoked by (\ref{ex:hc-scalar-w-ns}), (\ref{ex:hc-scalar-ns-w}) and (\ref{ex:hc-w-ns}) do.
	

		\begin{figure}[H]
			\centering
			\begin{subfigure}[b]{.3\linewidth}
				\centering
				\scalebox{1}{
					\begin{forest}
						for tree={s sep=2mm, inner sep=0, l=0}
						[CS[\fbox{$\neg\forall$}][$\forall$]]
				\end{forest}}
				\caption{}\label{fig:all-neg-polar}
			\end{subfigure}
			\begin{subfigure}[b]{.3\linewidth}
				\centering
				\scalebox{1}{
					\begin{forest}
						for tree={s sep=2mm, inner sep=0, l=0}
						[CS[\fbox{$\neg\exists$}][\fbox{\sbna}][$\forall$]]
				\end{forest}}
				\caption{}\label{fig:all-neg-wh}
			\end{subfigure}
			\begin{subfigure}[b]{.3\linewidth}
				\centering
				\scalebox{1}{
					\begin{forest}
						for tree={s sep=2mm, inner sep=0, l=0}
						[CS[$\exists$[$\forall$][\fbox{\sbna}]][$\neg\exists$]]
				\end{forest}}
				\caption{}\label{fig:all-neg-tiered}
			\end{subfigure}
			\caption{Qtrees for\textit{Jo did\textbf{n't} read \textbf{\textcolor{orange}{all}} of the books}, derived from Fig. \ref{fig:all-not-all}\&\ref{fig:none-sbna-all}}\label{fig:all-neg}
		\end{figure}

		\begin{figure}[H]
			\centering
			\begin{subfigure}[b]{.3\linewidth}
				\centering
				\scalebox{.8}{
					\begin{forest}
						for tree={s sep=2mm, inner sep=0, l=0}
						[CS[{Paris}][\fbox{$\neg$Paris}]]
				\end{forest}}
			\end{subfigure}
			\hfill
			\begin{subfigure}[b]{.3\linewidth}
				\centering
				\scalebox{.8}{
					\begin{forest}
						for tree={s sep=2mm, inner sep=0, l=0}
						[CS[{Paris}][\fbox{Nice}][\fbox{London}][\fbox{...}]]
				\end{forest}}
			\end{subfigure}
			\hfill
			\begin{subfigure}[b]{.3\linewidth}
				\centering
				\scalebox{.8}{
					\begin{forest}
						for tree={s sep=2mm, inner sep=0, l=0}
						[CS[France[Paris][\fbox{Lyon}][\fbox{...}]][UK[\fbox{London}][\fbox{...}]]]
				\end{forest}}
			\end{subfigure}
			\caption{Qtrees for \textit{Jo did\textbf{n't} study in \textbf{\textcolor{orange}{Paris}}}, derived from Fig. \ref{fig:city-partition}.}\label{fig:city-partition-neg}
		\end{figure}

		\begin{figure}[H]
			\centering
			\scalebox{.8}{
				\begin{forest}
					for tree={s sep=2mm, inner sep=0, l=0}
					[CS[$\neg\exists$][$\exists$[\fbox{\sbna}][$\forall$]]]
			\end{forest}}
				\scalebox{.8}{
				\begin{forest}
					for tree={s sep=2mm, inner sep=0, l=0}
					[CS[$\neg\exists$][\fbox{\sbna}][$\forall$]]
			\end{forest}}
			\caption{Qtrees compatible with (\ref{ex:hc-scalar-w-ns}) derived from Fig. \ref{fig:some-not-some}\&\ref{fig:all-neg-polar}/\ref{fig:all-neg-wh}}\label{fig:if-some-then-not-all}%=\textit{If Jo read \textbf{\textcolor{blue}{some}} of the books she has\textbf{n't} read \textbf{\textcolor{orange}{all}}}
		\end{figure}

		\begin{figure}[H]
			\centering
			\scalebox{.8}{
				\begin{forest}
					for tree={s sep=2mm, inner sep=0, l=0}
					[CS[$\neg\forall$[$\neg\exists$][\fbox{\sbna}]][$\forall$]]
			\end{forest}}
			\scalebox{.8}{
				\begin{forest}
					for tree={s sep=2mm, inner sep=0, l=0}
					[CS[$\neg\exists$][\fbox{\sbna}][$\forall$]]
			\end{forest}}
			\caption{Qtrees compatible with (\ref{ex:hc-scalar-ns-w}) derived from Fig. \ref{fig:all-neg-polar}\&\ref{fig:none-sbna-all}}\label{fig:if-not-all-then-some}%=\textit{If Jo hasn't read \textbf{\textcolor{orange}{all}} of the books she has read \textbf{\textcolor{blue}{some}}}
		\end{figure}

		\begin{figure}[H]
			\centering
			\scalebox{.8}{
				\begin{forest}
					for tree={s sep=2mm, inner sep=0, l=0}
					[CS[FR[Paris][\fbox{Nice}][\fbox{Lyon}][\fbox{...}]][$\neg$FR/UK]]
			\end{forest}}
			\caption{Qtree for (\ref{ex:hc-w-ns}), derived from Fig. \ref{fig:country-partition}\&\ref{fig:city-partition-neg}.}\label{fig:hc-non-scalar-w-ns}
		\end{figure}
	
	
		\begin{figure}[H]
			\centering
			\scalebox{.8}{
				\begin{forest}
					for tree={s sep=2mm, inner sep=0, l=0}
					[CS[$\neg$Paris[\fbox{\begin{tabular}{c}
							FR$\wedge$$\neg$Paris\\
							\textcolor{red}{$\subset$ FR}
					\end{tabular}}][UK/$\neg$FR]][Paris]]
			\end{forest}}
			\scalebox{.8}{
				\begin{forest}
					for tree={s sep=2mm, inner sep=0, l=0}
					[CS[Paris][\fbox{\begin{tabular}{c}
							FR$\wedge$Nice\\
							\textcolor{red}{$\subset$ FR}
					\end{tabular}}][London][...]]
			\end{forest}}
			\caption{Qtree for (\ref{ex:hc-ns-w}), derived from Fig. \ref{fig:city-partition-neg}\&\ref{fig:country-partition}.}\label{fig:hc-non-scalar-ns-w}
		\end{figure}
	
	\section{Capturing scalar HCs via Q-Relevance}
	Chapter \ref{chap:hurford-sentences} defined \textsc{Q-Relevance} as a constraint on Qtree computation: when combining Qtrees incrementally, none of the verifying nodes of the input Qtree should be cut across (i.e. be strictly entailed by some node) in the output Qtree. The constraint is repeated in (\ref{ex:q-relevance}).
	
	
	\begin{exe}
		\exr{ex:q-relevance} {\textsc{Q-Relevance}.  Let $X$ and $Y$ be LFs and let $Qtrees(X)$ and $Qtrees(Y)$ be the sets of Qtrees compatible with $X$ and $Y$. Let $\circ$ be a Qtree-level operation, e.g. $\neg$, $\vee$, or $\rightarrow$. Let $C$ be a non-empty partial LF (incremental context). Two cases:
			\begin{itemize}
				\item $C=\circ$, with $\circ$ a unary operation. For any $T \in Qtrees(X)$, $\circ T$ is \textsc{Q-Relevant} with respect to $\circ X$ iff $\forall N \in \mathbb{N}^+(T). \ \neg\exists N' \in \mathbb{N}(\circ T). \ N' \subset N$.
				\item $C = X \circ$, with $\circ$ a binary operation. For any $T \in Qtrees(Y)$, $T_X \circ T_Y$ is \textsc{Q-Relevant} with respect to $X \circ Y$ iff $\forall N \in \mathbb{N}^+(T_Y). \ \neg\exists N' \in \mathbb{N}(T_x \circ T_Y). \ N' \subset N$.
		\end{itemize}}
	\end{exe}
	
	\begin{exe}
		\ex {\textsc{Q-Relevance} \textit{(applied to conditionals)}.  Let $X$ and $Y$ be LFs and let $Qtrees(X)$ and $Qtrees(Y)$ be the sets of Qtrees compatible with $X$ and $Y$. For any $T \in Qtrees(Y)$, $T_X \rightarrow T_Y$ is \textsc{Q-Relevant} with respect to $X \rightarrow Y$ iff $\forall N \in \mathbb{N}^+(T_Y). \ \neg\exists N' \in \mathbb{N}(T_x \rightarrow T_Y). \ N' \subset N$.}
		\label{ex:q-relevance-conditional}
	\end{exe}
	
	This allowed to account for the contrast in (\ref{ex:hc-ns}). Let us briefly summarize the argument. (\ref{ex:hc-w-ns}) corresponds to the Qtree in Fig. \ref{fig:hc-non-scalar-w-ns}, which is obtained from a country-level antecedent Qtree and a city-level consequent Qtree; therefore, all verifying leaves of the consequent (city nodes different from Paris) are contained in some leaf of the antecedent Qtree, and can thus ``fit'' into the output Qtree without being cut across. \textsc{Q-Relevance} is thus satisfied.  (\ref{ex:hc-ns-w}) corresponds to the Qtrees in Fig. \ref{fig:hc-non-scalar-ns-w}, which are obtained from a city-level antecedent Qtree and a country-level consequent Qtree; in such trees, the \textit{France} verifying leaves are always cut across, either by \textit{not Paris}, or by individual city-nodes different from \textit{Paris}. \textsc{Q-Relevance} is thus violated.
	
	The same kind of reasoning shows that the Qtrees corresponding to (\ref{ex:hc-scalar-w-ns}) and (\ref{ex:hc-scalar-ns-w}), in resp. Fig. \ref{fig:if-some-then-not-all} and \ref{fig:if-not-all-then-some}, verify \textsc{Q-Relevance}. Starting with Qtree \ref{fig:if-some-then-not-all}: it can be built by incrementally combining  Qtree \ref{fig:some-not-some} (antecedent Qtree), with Qtree \ref{fig:all-neg-wh} (consequent Qtree). Qtree \ref{fig:all-neg-wh} has $\neg\exists$ and $\exists\wedge\neg\forall$ as verifying nodes; in the output Qtree \ref{fig:if-some-then-not-all}, both nodes are fully preserved. So (\ref{ex:hc-scalar-w-ns}) is compatible with a Qtree and is thus felicitous. As for (\ref{ex:hc-scalar-ns-w}), its Qtree \ref{fig:if-not-all-then-some} can be built by incrementally combining  Qtree \ref{fig:all-not-all} (antecedent Qtree), with Qtree \ref{fig:all-neg-wh} (consequent Qtree). Qtree \ref{fig:all-neg-wh} has $\neg\exists$ and $\exists\wedge\neg\forall$ as verifying nodes; in the output Qtree \ref{fig:if-some-then-not-all}, both nodes are fully preserved. So (\ref{ex:hc-scalar-ns-w}) is compatible with a Qtree and is thus felicitous. In brief, (\ref{ex:hc-ns-w}) and (\ref{ex:hc-w-ns}) are both rescued by the fact their consequent can evoke a Qtree whose verifying nodes are fine-grained enough to properly ``fit'' the structure already introduced by the antecedent Qtree.
	
	
	\subsection{Interim conclusion}
	We proposed an account of (scalar) HCs exploiting the intuitive idea that conditionals evoke ``restricted'' questions whose composition is constrained by the new notion of relevance presented back in Chapter \ref{chap:hurford-sentences}, \textsc{Q-Relevance}. The contrast between scalar and non-scalar HCs was thus captured, not \textit{via} \textit{exh} \textit{per se}, but instead by appealing to how scalar vs. non-scalar pairs of items differ information-structurally. Specifically, it was assumed scalar items could evoke fine-grained enough questions (generated by their scalemates) out-of-the-blue, while non-scalar items with different granularities could not.\\
	
	
	Before moving on to more complex cases in which scalarity and \textsc{Q-Relevance} also appear relevant(!), let us discuss the felicity profile of the scalar HCs in (\ref{ex:hc-s}), repeated below.
	
	\begin{exe}
		\exr{ex:hc-s}
		\begin{xlist}
			\ex{If Jo has read \textcolor{blue}{some} of the books she hasn't read \textcolor{orange}{all}.\hfill \s{} $\rightarrow$ $\neg$\splus}
			\ex{If Jo hasn't read \textcolor{orange}{all} of the books she has read \textcolor{blue}{some}.\hfill $\neg$\splus{} $\rightarrow$ \s}
		\end{xlist}
	\end{exe}
	
	In consulting with various speakers, judgments for (\ref{ex:hc-scalar-w-ns}) and (\ref{ex:hc-scalar-ns-w}) varied quite a bit. In particular, some speakers reported that (\ref{ex:hc-scalar-w-ns}) was hard to make sense of. This potential infelicity appears problematic for all accounts of Hurford Sentences -- in particular the current account, and \citet{Kalomoiros2024}'s SR. Here is however the sketch of a solution within the current framework. Recall that \textsc{Q-Relevance} imposes that some QuD evoked by the consequent of a conditional ``fit'' the information structure already introduced by the antecedent. One noticeable difference between (\ref{ex:hc-scalar-w-ns}) and (\ref{ex:hc-scalar-ns-w}), is that (\ref{ex:hc-scalar-w-ns}), unlike (\ref{ex:hc-scalar-ns-w}), features a \textit{negated} scalemate within its consequent. So far, our model of accommodated QuDs was assumed to handle negation quite transparently; specifically, we made the assumption that negation preserves Qtree structure, and only affects verifying nodes. But this might be too simplistic, and does not account for the intuition that negated expressions (e.g. \textit{not all}) may more saliently evoke ``polar'' QuDs (e.g. $\forall$/$\neg\forall$) as opposed to other QuDs (e.g. $\forall$/$\exists\wedge\neg\forall$/$\neg\exists$). If this is the case, then \textit{not all} in (\ref{ex:hc-scalar-w-ns}) may be less likely to evoke the kind of tripartite Qtree that rescued both scalar HCs in (\ref{ex:hc-s}). When combined with an antecedent QuD for \textit{some}, the polar QuD evoked by \textit{not all} then ends up violating \textsc{Q-Relevance}. The subtleness of the subsequent infelicity may be explained by the fact that negated expression \textit{preferentially} (but not always) evoke polar Qtrees.\\
	
	This observation can be related to informativity: uttering $\neg p$ when the question is \textit{whether p?}, is maximally informative, because it identifies one single cell -- the $\neg p$-cell. Uttering $\neg p$ when the questions is e.g. \textit{p, q, or r?}, is underinformative, because it does \textit{not} identify a single cell. To account for this, one might want to say that Qtrees ar ranked according to how well they are addressed by the assertion evoking them -- Qtree with smaller sets of verifying nodes should be preferred.
	
	The last section of the Chapter focuses on extensions of the current account, and in particular, explores predictions of \textsc{Q-Relevance} together with the intuition that scalemates may answer the same QuD.
	
	
	
	
	
	\iffalse
	
	\ex.[\Vref{ex:hd-s}{$'$}]
	\a.{Jo read (only) \textcolor{blue}{some} (but not all) or \textcolor{orange}{all} of the books.}\label{ex:hd-scalar-w-s-repaired}
	\b.{Jo read \textcolor{orange}{all} or ${}^\#$(only) \textcolor{blue}{some} ${}^\#$(but not all) of the books.}\label{ex:hcd-scalar-s-w-repaired}
	
	\ex.[\Vref{ex:hc-s}{$'$}]
	\a.{If Jo read (${}^\#$only) \textcolor{blue}{some} (${}^\#$but not all) of the books she hasn't read \textcolor{orange}{all}.}\label{ex:hc-scalar-w-ns-repaired}
	\b.{If Jo hasn't read \textcolor{orange}{all} of the books she's read (${}^\#$only) \textcolor{blue}{some} (${}^\#$but not all).}\label{ex:hc-scalar-ns-w-repaired}
	
	\fi
	
	
	

	
	also talk about negated HDs...
	Jo did not read all of the book or she did not read some of them
	<=>Jo read some but not all or she read none
	==> should be ok but is not
	
	
	The case of Long-Distance scalar  talk plans to dive into
	\noindent\textit{Context: Cafeteria Xor's meal plan is all you can eat starter XOR main dish XOR desserts.}
	\begin{exe}
		\ex {If Jo didn't have all starters or the main dish then she had some starters. \hfill $\neg$(\splus$\vee$\r)$\rightarrow$\s}
		\ex {?If Jo had some starters then she didn't have all starters or the main dish.\hfill \s$\rightarrow$$\neg$(\splus$\vee$\r)}
	\end{exe}
	
	if not all of the S or the main dish then some of the S fine
	if some of the S then not all of the S or the main dish sounds trivial but fine
	==> exh vacuous there (at least under material implication... just use commutativity and the fact exh is vacuous under neg)
	==> should pattern like 11 and 12... not the case!
	==> kalomoiros predicts them correctly to be fine
	
	
	(1) m has read some of the books if not all of them fine 
	(2) m has not read all of the books, if she has *(even) read some of them badish should be fine if no exh
	
	what do linear fs say
	(1) can be parsed as
	m has read sbna of the books if not all of them
	if not all then sbna not super redundant
	
	(2) must be parsed as 
	if some then not all
	analog to if france then not paris should be good
	
	
	what do hierarchical fs say
	(1) can be parsed as
	m has read some of the books if not all of them
	if not all then some not super redundant
	
	(2) must be parsed as 
	if some then not all
	analog to if france then not paris should be good
	
	
	
	m did not study in paris, if she studied in france fine
	m studied in france, if she did not study in p bad
	==> with non scalar hc reversal did not affect judgment
	
	
	\begin{exe}
		\ex{Jo did not study in Paris, if she studied in France.}
		\ex{Jo studied in France, if she did not study in Paris.} still bad
		\ex{?Jo has not read all of the books, if she has read some.}
		\ex{Jo has read some of the books, if she has not read all.}
	\end{exe}
	5 - not paris then france
	(not paris or not D) then france
	(paris and D) or france === france
	7, no exh - if not all then some
	(not all or not d) then some
	(all and d) or some === some 
	7, with exh - if not all then sbna
	(not all or not d) then sbna
	(all and d) or sbna =/= sbna
	==> having exh makes 7 not super redundant
	
	what about 6 with exh?
	some then (not all and some)
	(some and D) then (not all and some)
	not some or not D or (sbna)  =/= sbna 
	
	(some) then (sbna and D)
	not some or (sbna and D) =?= not some
	=> having exh makes 6 not super redundant too!
	
	if we buy super redundancy, then we have to say something about exh-licensing
	6 == not p or not p+
	not p or (not p+ and p)
	not p or not p+ and not p or p
	not p or not p+
	==> exh vacuous
	7 == p+ or p
	p+ or (p and not p+)
	(p+ or p) and (p+ or not p+)
	p+ or p
	==> exh vacuous
	
	All I have to do is update exh-licensing to make it ok in conditionals
	

